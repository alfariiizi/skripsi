\chapter{TINJAUAN PUSTAKA}\label{tipus}
Pemelajaran mendalam (\emph{deep learning}) sebagai sebuah pemahaman telah lahir sejak tahun 1940an. Namun pemelajaran mendalam sebagai istilah yang tren memang baru muncul belakangan. Setidaknya ada tiga gelombang pengembangan pemelajaran mendalam yang ditandai dengan nomenklaturnya \citep{goodfellow_bengio_courville_2016}. Gelombang pertama (1940-1960an), istilah pemelajaran mendalam dikenal dengan \emph{cybernetics} yang merupakan pengembangan dari teori pola pemelajaran biologis \citep{mcculloch_pitts_1943, morris_1999} dan implementasi pertama pada model \emph{perceptron} \citep{rosenblatt_1958}.

Pada gelombang kedua (1980-1995), dikenal dengan istilah \emph{connectionism} yang menggunakan pendekatan \emph{connectionist} dengan gebrakan baru yaitu perambatan mundur (\emph{back-propagation}) untuk melatih jaringan saraf (\emph{neural network}) dengan satu atau dua lapisan tersembunyi \citep{rumelhart_hinton_williams_1986}. Dan akhirnya gelombang ketiga atau masa kontemporer dimulai pada sekitar tahun 2006 hingga sekarang yang dikenal dengan \emph{deep learning} \citep{hinton_osindero_teh_2006, NIPS2006_5da713a6}.

Ilmu saraf memberikan wawasan yang banyak mengenai cara bekerja sebuah jaringan saraf biologis untuk dikembangkan ke dalam algoritma jaringan saraf buatan untuk belajar memecahkan beragam tugas yang berbeda \citep{goodfellow_bengio_courville_2016}. Salah satunya adalah penjelasan dari ilmuwan saraf bagaimana musang dapat belajar melihat dengan daerah pemrosesan pendengaran di otaknya jika otaknya diatur ulang untuk mengirimkan sinyal visual ke area tersebut \citep{von_melchner_pallas_sur_2000}. Hal ini kemudian menunjukkan bahwa mayoritas otak mamalia menggunakan algoritma tunggal untuk memecahkan kebanyakan tugas yang berbeda. Dari cara kerja tersebut, terinspirasilah ide utama untuk memiliki banyak unit komputasional yang menjadi pintar melalui interaksi dengan unit lain \citep{goodfellow_bengio_courville_2016}.

Neocognitron \citep{fukushima_1980} memperkenalkan model arsitektur yang sangat bagus untuk memroses gambar yang terinspirasi dari sistem visual mamalia dan yang kemudian menjadi cikal bakal ide untuk pengembangan jaringan saraf konvolusi modern (\emph{modern convolutional neural network} \citep{726791, lecun_kavukcuoglu_farabet_2010}.

Penyelesaian persamaan diferensial parsial menggunakan metode jaringan saraf buatan dimulai sejak tahun 1990an yang dimulai oleh \cite{lee_kang_1990} dengan artikelnya yang berjudul '\emph{Neural algorithm for solving differential equation}' yang terbit di \emph{Journal of Computational Physics}. Kemudian pada dekade yang sama muncul beberapa artikel yang membahas tentang topik ini, yaitu oleh Dissanayake dan Phan-Tien pada tahun 1994 dan Lagaris, dkk. pada tahun 1998. 

Dalam \citep{lagaris1998}, solusi persamaan diferensial dalam koordinat kartesian yang coba diselesaikan merupakan penjumlahan dari dua bagian. Bagian yang pertama merupakan permasalahan syarat batas atau syarat awal dan mengandung parameter yang tidak dapat diubah-ubah. Dan bagian yang kedua adalah bagian yang tidak memiliki hubungan dengan bagian syarat awal/syarat batas dan mengandung perambatan maju dari jaringan saraf buatan. Pengaplikasian metode ini dapat digunakan pada persamaan diferensial biasa tunggal, persamaan diferensial biasa berpasangan, dan persamaan diferensial parsial. Metode ini dikomparasikan dengan solusi dari metode numerik Galerkin \emph{finite element}.

Usaha-usaha awal penggunaan jaringan saraf buatan pada penyelesaian persamaan diferensial parsial ini sangat memanfaatkan algoritma perambatan mundur karena memberikan metode yang sangat akurat untuk menghitung ralat. Namun usaha-usaha ini memiliki kendala yang kurang lebih sama, yaitu hanya mampu menangani fungsi ruas kanan dan syarat batas tertentu saja.

Dengan perkembangan kekuatan komputasi pada tahun 2000an, dimungkinkan untuk membuat model yang lebih rumit, dan parameter yang lebih banyak, serta jumlah lapisan (\emph{layer}) yang lebih banyak. \cite{Smaoui2004} menggunakan jumlah daya komputasional yang lebih besar untuk menginvestigasi model yang lebih mendalam. menggunakan \emph{multilayer perceptron} untuk memprediksi dinamika dari dua persamaan diferensial parsial nonlinear menggunakan dekomposisi Kahunen-Loeve dan jaringan saraf buatan. \cite{baymani2010} mengembangkan jaringan saraf buatan untuk penyelesaian permasalahan Stokes. Permasalahan Stokes campuran ditransformasi ke dalam tiga permasalahan Poisson koordinat kartesian yang kemudian diselesaikan untuk mendapatkan hasil dari permasalahan Stokes tersebut. Hasil yang didapat dari metode jaringan saraf buatan ini kemudian dikomparasikan dengan metode numerik dari penelitian lainnya dan hasil eksaknya. Dari penelitian tersebut didapatkan bahwa pendekatan jaringan saraf tiruan yang baru memberikan hasil yang memiliki akurasi yang lebih tinggi dan jumlah parameter yang digunakan lebih sedikit dari model konvensional.

\cite{DBLP:journals/corr/abs-1711-10561} memperkenalkan jaringan saraf terinformasi fisika (\emph{physics informed neural networks} (PINN)), yaitu jaringan saraf yang dilatih untuk menyelesaikan tugas-tugas pembelajaran yang diawasi dengan tetap memperhatikan hukum fisika tertentu yang dijelaskan oleh persamaan diferensial parsial nonlinier umum. Pengembangan tersebut berada pada koridor untuk memecahkan dua masalah utama yaitu solusi berbasis data dan penemuan berbasis data dari persamaan diferensial parsial.  

Seiring dengan perkembangan model \emph{deep learning}, suatu terobosan dalam penyelesaian persamaan diferensial parsial menggunakan jaringan saraf buatan kemudian ditemukan. Para peneliti kemudian mengembangkan model jaringan saraf buatan konvolusi (\emph{convolutional neural network}/ CNN) dengan hipotesis bahwa dengan menggunakan CNN, parameter yang digunakan akan lebih sedikit sehingga akan banyak melakukan penghematan sumber daya komputer serta akan memberikan hasil yang lebih akurat. Selain itu, menurut \cite{Li_Li_Gao}, CNN memiliki kemampuan yang lebih baik untuk mengenali masukan gambar.

\cite{shan_2020_study} mengembangkan model penyelesaian persamaan Poisson menggunakan CNN untuk memprediksi potensial listrik  dengan variasi pada eksitasi dan distribusi permitivitas pada bidang kartesian 2 dimensi dan 3 dimensi. Penelitian ini menggunakan desain \emph{cost function} yang dikustomisasi dan data yang didapat dari metode numerik beda-hingga. Model yang dikembangkan dapat melakukan peforma yang cukup efektif dibandingkan dengan metode numerik yang digunakan untuk membentuk data latih, dengan rata-rata ralat prediksi kurang dari 3\%.

Perkembangan yang cukup signifikan dilakukan oleh \cite{Ozbay2021} dengan mengembangkan arsitektur CNN untuk menyelesaikan persamaan Poisson 2 dimensi pada koordinat kartesian dengan variasi resolusi yang diberikan oleh ruas kanan persamaan, sembarang syarat batas, dan variasi parameter grid. Permasalahan syarat batas diselesaikan dengan metode baru, mendekomposisikan persaman Poisson yang asli ke dalam satu persamaan Poisson homogen dan empat submasalah Laplace nonhomogen. Model yang dikembangkan terbukti dapat mengungguli model dengan jaringan saraf tiruan konvensional dan dapat memprediksi dengan rata-rata presentase ralat di bawah 10\%. 

\cite{cheng2021using} menyelesaikan persamaan Poisson 2 dimensi dengan syarat batas Dirichlet nol menggunakan lapisan konvolusi secara penuh dengan arsitektur U-Net \citep{DBLP:journals/corr/RonnebergerFB15} yang didefinisikan dengan variasi pada jumlah percabangan, kedalaman, dan medan reseptif. Pada penelitian tersebut didapati bahwa medan reseptif memiliki peran yang penting dalam menangkap struktur topologis dari masukan pada tiap lapisan. Untuk menghitung syarat batas dan interior, digunakan dua syarat batas, yaitu \emph{dichlet loss} dan \emph{inside loss} serta syarat batas alternatif yaitu \emph{laplacian loss}. Untuk pelatihan digunakan dataset dengan distribusi muatan acak dan dataset dengan dengan distribusi mengikuti aturan deret Fourier.

Sebagai domain fisis, dalam penelitian ini akan digunakan model pendorong Hall (\emph{Hall thruster}) sebagai perwakilan masalah fisis perhitungan potensial listrik di koordinat silinder. \cite{braga_miranda_2019} dari Laboratorium Fisika Plasma Universitas Brasil (LFP-UnB) sejak 2004 sedang mengembangkan pendorong Hall yang disebut PHall yang memiliki perbedaan dalam dimensi kanal, parameter operasi, dan mekanisme pembentukan medan magnet yang dibutuhkan dengan SPT-100. SPT-100 dijadikan sebagai patokan dan pembanding dalam pembangunannya. Model 2 dimensi, spesifikasi serta, gambaran simulasi tentang SPT-100 dijelaskan dalam karya mereka.

Penelitian ini akan memecahkan masalah persamaan diferensial parsial berupa persamaan Poisson yang diterapkan pada domain fisis pedorong Hall SPT-100 pada koordinat silinder dua dimensi menggunakan metode jaringan saraf konvolusi dengan \emph{ground truth} yang dihitung menggunakan metode numerik Gauss-Seidel.