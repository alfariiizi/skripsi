\chapter{PENDAHULUAN}
\pagenumbering{arabic}
\setcounter{page}{1}
\section{Latar Belakang Masalah}
\label{latarbelakang}

% [Komputer]

Penggunaan komputer tidak bisa lepas dari kehidupan manusia saat ini. Hampir
semua manusia mempunyai komputer. Komputer yang digunakan ini berbentuk smartphone, laptop, hingga
komputer berbentuk \emph{microcontroller} seperti Arduino atau
\emph{microprocessor} seperti Raspberry Pi. Kebergantungan manusia akan
komputer ini disebabkan karena tingginya kecepatan komputer dalam melakukan
komputasi. Dari komputasi sederhana seperti pada aplikasi kalkulator,
hingga komputasi kompleks seperti \emph{event handler} pada aplikasi mobile.

% [CPU]

Kecepatan komputer dalam melakukan proses komputasi Tinggi disebabkan karena
arsitektur dari komputer sendiri yang mempunyai komponen utama bernama
\emph{Central Processing Unit} atau \emph{CPU}. CPU ini bertindak sebagai
\emph{"Control Center"} yang mana menginterpretasi, memproses, dan mengeksekusi
instruksi dari software dan hardware \citep{armWhatCentralProcessing2023}. Hal
ini menyebabkan pemilihan CPU menjadi hal yang krusial ketika ingin mendapatkan
proses komputasi yang cepat.

% [Komputasi Numerik]

Manfaat dari munculnya komputer tidak hanya dirasakan oleh masyarakat umum,
namun juga dirasakan oleh akademisi, peneliti, maupun pekerja di suatu
industri. Banyak aplikasi yang jauh lebih mudah dikerjakan jika dilakukan
menggunakan komputer. Bahkan saat ini muncul bidang bidang baru yang
menggunakan komputer sebagai dasar dari suatu komputasi, seperti \emph{Data Science},
\emph{Data Analyst}, \emph{Machine Learning Engineering}, dan masih
banyak lagi. Bidang - bidang tersebut sangat bergantung pada proses komputasi
yang cepat, sehingga memerlukan tinggi kecepatan CPU dalam melakukan kalkulasi.

Dalam lingkup penelitian, khusus nya penelitian di bidang metode numerik,
proses kalkulasi yang cepat dapat memudahkan para peneliti untuk mendapatkan
hasil yang diinginkan. Untuk itu, dengan adanya komputer, para peneliti dapat
menghasilkan metode numerik baru yang lebih berat dalam proses komputasi, namun
memiliki nilai yang lebih akurat. Proses komputasi ini juga pastinya sangat
bergantung pada CPU yang digunakan. Sehingga secara tidak langsung penggunaan
CPU juga mempengaruhi akurasi suatu kalkulasi berdasarkan suatu metode yang
digunakan.

% [CPU]

Pada komponen CPU, terdapat inti atau \emph{core} yang mana umumnya berjumlah
kurang dari 10. Salah satu hal yang mempengaruhi kecepatan komputasi CPU jumlah inti yang ada di dalam CPU. Namun, CPU sendiri juga memiliki aturan dalam banyak
nya inti yang digunakan. CPU didesain untuk menyelesaikan permasalahan
komputasi dari yang mudah hingga kompleks. Berdasarkan
\cite{kukunasChapterIntelPentium2015}, CPU sendiri mempunyai desain arsitektur
\emph{Complex Instruction Set Computer} atau \emph{CISC} untuk menyelesaikan komputasi
yang kompleks. Kemudahan dalam pemberian instruksi ini yang menjadi alasan
mengapa CPU sengaja menggunakan inti yang sedikit.

Proses kalkulasi yang memuat banyak data atau perhitungan, umumnya membutuhkan
inti CPU yang banyak agar hasil lebih cepat diperoleh. Namun, dikarenakan inti
CPU berjumlah sedikit, kecepatan kalkulasi ini tidak dapat dimaksimalkan lagi
pada suatu perangkat. Beberapa solusi lain agar kalkulasi dijalankan pada inti
yang lebih banyak, diantara nya adalah dengan menyewa penyedia server kalkulasi
seperti \emph{Google Colab}. Kekurangan menyewa penyedia server kalkulasi ini
yang pasti adalah perlu nya biaya tambahan ketika ingin menambah inti yang akan
digunakan. Alternatif lain yang dapat dilakukan supaya mendapat inti yang
banyak dengan biaya yang gratis adalah dengan menggunakan komponen lain dari
perangkat komputer yang bernama GPU.

% [GPU]

GPU atau \emph{Graphics Processing Unit} merupakan pemrosesan kalkulasi yang
ditujukan untuk me-\emph{render} suatu gambar atau \emph{3D computer graphics}
ke layar perangkat komputer \citep{armWhatGraphicsProcessing2023}. Pada
kebanyakan kasus, penggunaan GPU ini akan sangat berpengaruh ketika memainkan
game yang mempunyai tampilan yang baik, seperti model di gamenya mempunyai
jumlah \emph{mesh} yang banyak atau juga bahkan pencahayaan di game mempunyai
metode yang membutuhkan kalkulasi yang tinggi. Mampunya GPU untuk melakukan
kalkulasi yang berat ini disebabkan karena banyaknya inti yang ada di GPU. Beda
nya jumlah inti pada CPU dan GPU disebabkan karena perbedaan fungsi antara
kedua komponen tersebut. CPU lebih ditujukan pada kalkulasi yang kompleks
dengan jumlah kalkulasi yang lebih sedikit, sedangkan GPU lebih ditujukan untuk
kalkulasi yang tidak kompleks dengan jumlah kalkulasi yang banyak
\citep{intelCPUVsGPU2023}. Meskipun GPU mempunyai fungsi dasar untuk melakukan
\emph{rendering} ke layar perangkat komputer, namun GPU sendiri juga bisa
digunakan untuk melakukan kalkulasi tanpa perlu nya melakukan \emph{rendering}
ke layar komputer. Penggunaan GPU untuk melakukan komputasi umum ini biasaya
disebut dengan \emph{General Purpose GPU} atau GPGPU
\citep{gigabyteWhatGPGPUWhy2023}. Penggunaan GPGPU ini ditujukan agar tercapai
nya \emph{High Performance Computing} atau HPC yang mana merupakan komputasi
yang mampu melakukan perhitungan pada kecepatan tinggi
\citep{gigabyteHPCHighPerformance2023}.

% [Julia]

Penggunaan GPGPU pada dasarnya mempunyai kekurangan yang membuat banyak
pengguna memilih alternatif lain daripada GPGPU, yakni kompleks nya
implementasi ketika ingin menggunakan GPGPU. Implementasi yang kompleks ini
muncul karena GPU mempunyai jumlah inti yang banyak. Beberapa \emph{Application
	Programming Interface} atau API \citep{evansonExplainerWhatAPI2021} yang dapat
membuat penggunaan GPGPU lebih mudah, diantara nya adalah \emph{Vulkan API},
\emph{Direct3D}, dan \emph{OpenGL}
\citep{khairySurveyArchitecturalApproaches2019}. Namun, kedua API tersebut pada
dasarnya merupakan \emph{Graphics API} yang dituliskan dalam bahasa C/C++ yang
mana membutuhkan penyesuaian lagi jika ingin digunakan untuk proses kalkulasi
tanpa \emph{rendering}. Ditambah lagi, bahasa C/C++ mempunyai \emph{syntax} dan
paradigma yang relatif lebih susah jika dibandingkan dengan bahasa lain seperti
Python. Salah satu alternatif bahasa pemrograman yang dapat digunakan adalah
Julia. Julia mempunyai \emph{syntax} dan paradigma yang tidak terlalu kompleks seperti
Python, dan memiliki performa yang baik seperti C++
\cite{bezansonJuliaMicroBenchmarks2023}. Kombinasi \emph{syntax} dan performa inilah
yang membuat Julia menjadi bahasa pemrograman paling cocok untuk implementasi
GPGPU pada saat ini.

Berdasarkan \cite{jeffbezansonJuliaProgrammingLanguage2024}, Julia mempunyai ekosistem yang banyak ketika digunakan untuk menyelesaikan permasalahan komputasi ilmiah secara umum. Salah satu permasalahan komputasi adalah permasalahan pencarian nilai eigen dalam fisika kuantum. Untuk itu, penggunaan Julia ini juga diharapkan dapat mempermudah dalam melakukan implementasi penyelesaian permasalahan pencarian nilai eigen, dan diharapkan juga diperoleh nilai yang akurat ketika dibandingkan dengan nilai anlitik nya.

\section{Rumusan Masalah}
Berdasarkan latar belakang yang telah dipaparkan, maka rumusan masalah yang
diangkat pada penelitian ini adalah sebagai berikut:
\begin{enumerate}
	\item Bagaimana cara kerja komputasi sistem series pada CPU dan komputasi sistem paralel pada GPU dengan menggunakan
	      Bahasa Julia

	\item Bagaimana performa yang dihasilkan dari komputasi secara \emph{parallel}
	      dibandingkan dengan komputasi secara \emph{series}
\end{enumerate}

\section{Tujuan Penelitian}
\begin{enumerate}
	\item Mempelajari cara kerja komputasi sistem paralel pada GPU dengan menggunakan Bahasa Julia
	      untuk menyelesaikan permasalahan pada metode komputasi

	\item Membandingkan performa komputasi sistem paralel pada GPU dengan komputasi sistem series pada CPU
\end{enumerate}

\section{Batasan Masalah}
\begin{enumerate}
	\item Pada kajian ini, sistem series dan paralel menggunakan bahasa yang sama, yakni
	      Bahasa Julia, serta kajian ini tidak akan membandingkan performa sistem series
	      dan paralel pada bahasa lain. Tidak menggunakan bahasa lain, seperti Python, dikarenakan Julia sudah mendukung sistem \emph{High Performance Computing} (HPC), sedangkan Python belum mendukung sistem HPC. Tidak menggunakan bahasa yang mendukung sistem HPC, seperti C/C++, dikarenakan Julia mendukung berbagai macam komputasi ilmiah, sedangkan C/C++ belum mendukung berbagai macam komputasi ilmiah. Serta, pada \href{https://julialang.org/}{Website Julia}, secara eksplisit dijelaskan bahwa fitur utama Julia adalah Komputasi Paralel pada GPU.

	\item Perbandingan performa komputasi series dan komputasi parelel dilakukan dalam
	      lingkup permasalahan fisika berbasis matriks.
\end{enumerate}

\section{Manfaat Penelitian}
\begin{enumerate}
	\item Diharapkan penelitian ini dapat membawa pandangan baru bahwa Bahasa Julia
	      mempunyai penulisan kode yang sederhana ketika dijalankan di sistem series
	      maupun sistem paralel

	\item Diharapkan hasil penelitian ini dapat menunjukkan komputasi paralel dengan
	      menggunakan Bahasa Julia mempunyai performa yang lebih baik seiring
	      meningkatnya ukuran data, daripada komputasi series
\end{enumerate}
