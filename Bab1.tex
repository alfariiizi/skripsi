\chapter{PENDAHULUAN}
\pagenumbering{arabic}
\setcounter{page}{1}
\section{Latar Belakang Masalah}
\label{latarbelakang}

Dalam kehidupan kita sehari-hari, perubahan adalah sebuah keniscayaan. Perubahan
yang terjadi bisa sangat beragam bentuknya, mulai dari perubahan yang terjadi sangat
lambat dan dapat diamati, misalnya perubahan tinggi badan manusia dari tahun ke
tahun; perubahan yang terjadi secara cepat yang masih dapat diamati, seperti
halnya perubahan kecepatan kendaraan; hingga perubahan yang sangat cepat, dan
sulit untuk diamati, seperti perubahan properti dari partikel.

Test test \cite{alfarizi_2023_powerranger}, and others.

Dalam ilmu terapan (misalnya Ilmu ekonomi \citep{norberg_1995}) dan ilmu alam (misalnya
Biologi \citep{culshaw_ruan_2000}), perubahan dimodelkan secara matematis dengan
menggunakan persamaan diferensial. Persamaan diferensial menggambarkan perubahan
yang terjadi dari suatu persamaan terhadap suatu variabel (biasanya variabel waktu).
Melalui persamaan diferensial, sebuah perubahan dapat digambarkan secara akurat.
Maka dari itu, persamaan diferensial sangat jamak digunakan di berbagai bidang
keilmuan dan teknologi.

Dalam bidang Fisika, persamaan diferensial memiliki kedudukan yang sangat penting.
\cite{strogatz_2020_infinite} dalam Infinite Powers menyatakan bahwa sejak (penemuan
Kalkulus oleh) Newton, umat manusia sadar bahwa seluruh hukum fisika dinyatakan dalam
bahasa persamaan diferensial. Pernyataan Strogatz ini bukanlah suatu hal yang
berlebihan. Karena nyatanya kebanyakan konsep pada dalam bidang Fisika menggunakan
persamaan diferensial.

Mayoritas fenomena Fisika, entah dalam domain dinamika fluida, kelistrikan,
magnetik, mekanika, optik, ataupun aliran panas, dapat dideskripsikan secara
umum menggunakan persamaan diferensial parsial (\emph{partial differential
equation}(PDE)), dan sesungguhnya, sebagian besar dari matematika fisika adalah
PDE \citep{farlow_2012_partial}.

Jika suatu keadaan yang akan dimodelkan memiliki lebih dari satu faktor, maka
digunakan persamaan diferensial parsial. Persamaan diferensial parsial merupakan
suatu relasi/persamaan yang melibatkan satu atau dua derivatif dan suatu fungsi.
Biasanya, fungsinya menggambarkan kuantitas dan derivatif menggambarkan rasio dari
perubahan, dan persamaan diferensial adalah hubungan dari kedua komponen
tersebut.

Persamaan Poisson adalah salah satu bentuk PDP yang umum digunakan. Persamaan Poisson
digunakan untuk menjelaskan berbagai perubahan pada berbagai kuantitas fisika,
seperti misalnya potensial gravitasi, potensial elektrostatis, dan sebagainya
pada daerah yang memiliki muatan, massa, atau sumber lain seperti panas atau
fluida \citep{boas_2006_mathematical}. Dalam bidang komputasi dinamika fluida, persamaan
Poisson digunakan untuk menghitung koreksi dari tekanan medan untuk memastikan inkompresibilitas
medan kecepatan \citep{Ozbay2021}.

Pada koordinat kartesian, telah ada usaha yang berhasil dalam mencari solusi yang
cukup akurat dan metode paralel untuk persamaan Poisson, misalnya menggunakan
metode Fourier \citep{cohl_1999}. Situasi demikian menjadi tidak mudah pada
koordinat silinder, sebab variasi yang tidak konstan pada elemen matriks yang dihasilkan
dari diskretisasi terbatas pada persamaan Poisson silinder. Hal ini yang membuat
metode langsung Fourier pada koordinat silinder menjadi tidak mungkin \citep{cohl_1999}.

Karena pentingnya persamaan diferensial, timbul banyak metode untuk solusi
persamaan diferensial, contohnya metode Runge-Kutta, metode Predictor- Corrector,
metode Beda-Hingga, metode Elemen-Batas, metode Splines, dan metode lainnya.
Metode-metode tersebut membutuhkan diskretisasi dari domain yang ada ke dalam
sejumlah elemen hingga sehingga fungsi tersebut didekati secara lokal \citep{kumar_yadav_2011}.
Walaupun metode-metode ini memberikan perkiraan yang baik pada solusi, metode-metode
tersebut membutuhkan diskretisasi domain melalui \emph{meshing} yang merupakan
tantangan untuk permasalahan pada dua atau lebih dimensi \citep{kumar_yadav_2011}.
Selain itu, turunan solusi perkiraan bersifat diskontinu dan dapat berdampak serius
pada stabilitas solusi. Lebih jauh lagi, untuk mendapatkan akurasi solusi yang memuaskan,
mungkin perlu untuk mengubah \emph{mesh} yang secara signifikan meningkatkan upaya
komputasi \citep{kumar_yadav_2011}. Perkiraan solusi PDP dengan metode-metode tersebut
dapat dilakukan secara lebih menguntungkan dengan menggunakan pendekatan pemelajaran
mendalam (\emph{deep learning}).

Dalam penelitian ini akan diperkenalkan metode penyelesaian persamaan Poisson
dalam koordinat silinder 2 dimensi pada syarat batas Dirichlet dengan paradigma
pemrograman pembelajaran mesin dengan metode pemelajaran mendalam yaitu jaringan
saraf tiruan konvolusi (\emph{convolutional neural network} (CNN)). Hipotesis
yang dibangun untuk melakukan penelitian ini dibangun dari pernyataan
\cite{moroney_2022} bahwa pemelajaran mesin adalah paradigma pemrograman yang
akan menghasilkan aturan (\emph{rules}) setelah belajar dari jawaban (\emph{answer})
dan data. Program yang sudah melakukan pemelajaran ini akan dapat melakukan prediksi
dengan relatif lebih cepat secara signifikan dibandingkan dengan metode numerik iteratif
lainnya yang memberikan \emph{answer} lewat \emph{rules} dan data yang diberikan
oleh pemrogram. Lebih khusus mengenai CNN, menurut \cite{Li_Li_Gao}, CNN lebih
baik dalam hal menangani masukan gambar daripada jaringan saraf (\emph{neural
network}) reguler.

Koordinat silinder digunakan pada penelitian ini dikarenakan penulis melihat peluang
pada penerapan perhitungan potensial listrik pada bagian penggerak wahana luar angkasa
berbasis plasma, yaitu \emph{Hall thruster}. Menurut \cite{cohl_1999}, dari sudut
pandang matematika, ada dua metode untuk menghitung potensial listrik: dengan menyelesaikan
persamaan diferensial parsial (contohnya persamaan Poisson), atau menyelesaikan
permasalahan persamaan integral (contohnya metode Green). Sebagai perwakilan sistem
fisis dari pendorong Hall, akan digunakan referensi dimensi domain pendorong
Hall dengan model SPT-100 (\emph{Stationary Plasma Thruster-100)}. Pemilihan SPT-100
menjadi model referensi karena model SPT-100 sudah cukup mapan dan memiliki sejarah
penggunaan yang cukup banyak di berbagai proyek ruang angkasa, sehingga model
ini menjadi patokan (\emph{benchmark}) bagi pendorong \emph{Hall} yang sedang
dibangun \citep{braga_miranda_2019}.

Model penyelesaian persamaan Poisson menggunakan pemelajaran mendalam di koordinat
silinder 2 dimensi yang ditawarkan pada penelitian ini juga akan menginvestigasi
manfaat dan keuntungan yang dihasilkan dari model ini dibandingkan dengan metode
numerik Gauss-Seidel yang juga merupakan metode untuk pengambilan data latih dan
data uji untuk model ini.

\section{Rumusan masalah}
Berdasarkan latar belakang yang telah dipaparkan pada subbab \ref{latarbelakang},
maka rumusan masalah yang diangkat pada penelitian ini adalah:
\begin{enumerate}
  \item Bagaimana cara kerja algoritma CNN dalam perhitungan persamaan Poisson dalam
    koordinat silinder dua dimensi?

  \item Bagaimana unjuk kerja algoritma CNN dibandingkan metode numerik Gauss-Seidel
    dalam penyelesaian persamaan Poisson dalam koordinat silinder 2 dimensi?
\end{enumerate}

\section{Tujuan dan Manfaat Penelitian}
\subsection{Tujuan Penelitian}
\begin{enumerate}
  \item Mempelajari cara kerja algoritma CNN dalam pemecahan masalah regresi serta
    variabel yang dapat mengoptimasi pada perhitungan permasalahan Poisson

  \item Mengetahui peforma algoritma CNN dalam penyelesaian permasalahan Poisson
    pada koordinat silinder 2 dimensi
\end{enumerate}

\subsection{Manfaat Penelitian}
\begin{enumerate}
  \item Diharapkan penelitian ini dapat digunakan pada perhitungan potensial listrik
    pada koordinat silinder, utamanya model koaksial seperti \emph{hall thruster}

  \item Diharapkan hasil penelitian ini dapat menjadi nilai tebakan awal bagi metode
    iteratif seperti Gauss-Seidel untuk mempercepat metode tersebut

  \item Diharapkan penelitian ini dapat menambah wawasan metode perhitungan pada
    dunia Fisika dengan menggunakan metode pembelajaran mendalam

  \item Diharapkan penelitian ini dapat menjadi acuan awal dalam penelitian dan pengembangan
    model CNN pada koordinat silinder 2 dimensi
\end{enumerate}