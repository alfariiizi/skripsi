\chapter{KESIMPULAN DAN SARAN}
\section{Kesimpulan}
Dari penelitian ini, dapat disimpulkan
\begin{enumerate}
	\item Cara kerja komputasi di sistem paralel oleh GPU adalah dengan menjalankan program di CPU, kemudian mengirimkan data dari CPU ke GPU, GPU melakukan komputasi pada data yang diterima, kemudian hasil kalkulasi akan dikirimkan dari GPU ke CPU. Untuk mengatur supaya CPU menunggu kalkulasi GPU, perlu ditambahkan \cw{@async} pada kode Julia nya.
	\item Pada operasi yang sederhana dan pada ukuran matriks yang kecil, eksekusi sistem series oleh CPU dapat lebih cepat daripada eksekusi sistem paralel oleh GPU. Namun, pada operasi yang kompleks dan pada ukuran matriks yang besar, eksekusi sistem paralel oleh GPU dapat lebih cepat daripada eksekusi sistem series oleh CPU.
\end{enumerate}

% \section{Saran}
% \begin{enumerate}
% \item Dilakukan \textit{data cleaning} sebelum menggunakan data latih dan data uji
% \item Membuka kemungkinan pada jenis/bentuk arsitektur lain, khususnya pemanfaatan \textit{transfer learning} dalam pembentukan arsitektur.

% \item Penulis hanya mempunyai RAM yang terbatas, sehingga hasil pengujian pada ukuran matriks yang lebih besar masih belum bisa dilakukan
% \end{enumerate}
