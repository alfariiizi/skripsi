\documentclass{skripsiactugm}
\usepackage{chapterbib}
\usepackage{setspace}
\graphicspath{{./gambar/}}
\usepackage{svg}
\usepackage{empheq}
\usepackage{wrapfig}
\usepackage{lipsum}
\usepackage{array}
\newcolumntype{M}[1]{>{\centering\arraybackslash}m{#1}}
\usepackage[utf8]{inputenc}
\usepackage{tabularx}
\usepackage{pdflscape}
% \usepackage[table]{xcolor}
\usepackage{mathtools}
\usepackage{listings}
\usepackage{caption}
\usepackage{jlcode}
\usepackage{booktabs}

% \usepackage[table,xcdraw]{xcolor}
\usepackage{lscape}
\usepackage{longtable}

\newcounter{pycodecounter}
\newcounter{cppcodecounter}

\lstnewenvironment{mypythoncode}[1][]{
	\refstepcounter{pycodecounter}
	\lstset{ language=Python, basicstyle=\small\ttfamily, numbers=left, numberstyle=\tiny,
		stepnumber=1, numbersep=5pt, backgroundcolor=
		\color{gray!10}
		, showspaces=false, showstringspaces=false, showtabs=false, frame=single, tabsize=4,
		breaklines=true, breakatwhitespace=true, escapeinside={(*@}{@*)}, captionpos=b, caption={Kode~\thepycodecounter: #1}
	} }{}

\lstnewenvironment{mycppcode}[1][]{
	\refstepcounter{cppcodecounter}
	\lstset{ language=C++, basicstyle=\small\ttfamily, numbers=left, numberstyle=\tiny,
		stepnumber=1, numbersep=5pt, backgroundcolor=
		\color{gray!10}
		, showspaces=false, showstringspaces=false, showtabs=false, frame=single, tabsize=4,
		breaklines=true, breakatwhitespace=true, escapeinside={(*@}{@*)}, captionpos=b, caption={Kode~\thepycodecounter: #1}
	} }{}

\captionsetup[lstlisting]{labelformat=empty}
%-----------------------------code untuk lampiran---------------------------
\lstnewenvironment{lampiranpythoncode}[1][]{
	\refstepcounter{pycodecounter}
	\lstset{ language=Python, basicstyle=\small\ttfamily, numbers=left, numberstyle=\tiny,
		stepnumber=1, numbersep=5pt, backgroundcolor=
		\color{gray!10}
		, showspaces=false, showstringspaces=false, showtabs=false, frame=single, tabsize=4,
		breaklines=true, breakatwhitespace=true, escapeinside={(*@}{@*)}, captionpos=b,
		% caption={Python Code~\thepycodecounter: #1}
	} }{}

\lstnewenvironment{lampirancppcode}[1][]{
	\refstepcounter{cppcodecounter}
	\lstset{ language=C++, basicstyle=\small\ttfamily, numbers=left, numberstyle=\tiny,
		stepnumber=1, numbersep=5pt, backgroundcolor=
		\color{gray!10}
		, showspaces=false, showstringspaces=false, showtabs=false, frame=single, tabsize=4,
		breaklines=true, breakatwhitespace=true, escapeinside={(*@}{@*)}, captionpos=b,
		% caption={C++ Code~\thecppcodecounter: #1}
	} }{}

\captionsetup[lstlisting]{labelformat=empty}
%-----------------------------------------------------------------
%Disini awal masukan untuk data Skripsi (ISI SESUAI DENGAN DATA ANDA!)
%-----------------------------------------------------------------
\titleind{Pemanfaatan General Purpose GPU dengan Bahasa Pemrograman Julia untuk Komputasi Berunjuk Kerja Tinggi}

\titleeng{Utilization of General Purpose GPUs with the Julia Programming Language for High Performance Computing}

\fullname{Moh Rizal Alfarizi}
\NIM{19/445591/PA/19415}
\yearenrollment{2019}
\examdate{28 Juni 2024}
\degree{Sarjana}
\yearsubmit{2024}
\firstsupervisor{Prof. Drs. Pekik Nurwantoro, M.S., Ph.D.}
% \secondsupervisor{Ahsani Hafizhu Shali, S.Si., M.Sc.}
\firstexaminer{Dr. Eko Sulistya, M.Si.}
\secondexaminer{Sholihun, S.Si., M.Sc., Ph.D.Sc}
% \thirdexaminer{Nama Anggota Tim Penguji}

\begin{document}

\cover

% \titlepage <-- activate this

\approvalpage \declarepage
%Setelah Anda selesai Ujian Akhir Skripsi, scan halaman pengesahan yang telah ditandatangani dosen pembimbing dan penguji, scan juga halaman pernyataan yang telah Anda tandatangani. Selanjutnya simpan hasil scan tersebut dalam bentuk PDF dengan nama pengesahanskripsi.pdf dan pernyataan.pdf (file disimpan di folder yang sama dengan folder dimana file Skripsi.tex tersimpan)
%Hilangkan karakter % sebelum perintah "\approvalpagescan" dan "\declarepage" di bawah ini, dan tambahkan karakter % sebelum perintah "\approvalpage" di atas.
%\approvalpagescan
%\declarepagescan

%-----------------------------------------------------------------
%Halaman Persembahan (ISI SESUAI DENGAN DATA ANDA!)
%-----------------------------------------------------------------
% \acknowledment
% \begin{flushright}
% \Large\emph\cal{Untuk Mama dan Papa}
% \end{flushright}
%-----------------------------------------------------------------
%-----------------------------------------------------------------

%-----------------------------------------------------------------
%Halaman Motto (ISI SESUAI DENGAN DATA ANDA!)
%-----------------------------------------------------------------
% \motto
% "...carilah maka kamu akan mendapat..."
%-----------------------------------------------------------------
%-----------------------------------------------------------------

%-----------------------------------------------------------------
%Disini awal masukan untuk Prakata (ISI SESUAI DENGAN DATA ANDA!)
%-----------------------------------------------------------------
\preface

Transkrip.xyz adalah sebuah perusahaan rintisan (\textit{startup}) yang berbasis
di Sleman, Indonesia, yang menyediakan jasa transkripsi audio ke teks menggunakan
teknologi kecerdasan buatan (\textit{artificial intelegence})LLM (\textit{large
	language model}) dengan rata-rata hingga 20.000 pengguna. Uniknya, perusahaan
ini tidak diinisiasi oleh seorang dengan latar belakang teknologi atau ilmu
komputer, melainkan diinisiasi dan dikembangkan baik segi teknis maupun bisnisnya
oleh seorang dari latar belakang ilmu Filsafat. Hal yang melatarbelakangi
terbentuknya perusahaan ini tentu bukan berasal dari kemampuan \textit{founder}-nya,
melainkan berasal dari kepekaan dalam melihat kebutuhan masyarakat dan
perkembangan teknologi sekarang.

Belakangan banyak orang latah mengenai perkembangan pemelajaran mesin (\textit{machine
	learning}) dan implementasinya pada kecerdasan buatan. Berbagai pekerjaan yang
sebelumnya dilakukan oleh manusia kini dapat diotomasi mengggunakan
pemelajaran mesin. Sebut saja beberapa pekerjaan yang sebelumnya kita sebagai manusia
saja susah melakukannya, seperti pendektesian objek, pengenalan wajah,
transkripsi audio, dan masih banyak lagi. Kemudian sekarang muncul pertanyaan
mengenai bagaimana teknologi pemelajaran mesin dapat memiliki kontribusi pada
pekerjaan di bidang Fisika, atau bahkan kalau mau lebih jauh lagi, seberapa
mampu bidang Fisika memberikan kontribusi pada bidang pemelajaran mesin.

Skripsi ini merupakan salah satu upaya untuk menjawab pertanyaan pertama
tersebut. Perhitungan persamaan Poisson secara numerik membutuhkan upaya yang sangat
besar. Hal ini yang kemudian menjadi pendorong utama penelitian ini. Pertanyaan
utamanya adalah bagaimana apabila pemecahan persamaan Poisson didekati dengan paradigma
pemrograman pemelajaran mesin: pemrogram tidak membuat aturan untuk memecahkan
masalah tersebut, melainkan hanya memberikan data yang sangat banyak (bisa
dianggap sebagai \textit{big data}) dan kemudian komputer sendiri yang akan
membuat aturan tersebut. Pemahaman seperti ini yang kiranya tumbuh dalam diri tiap
fisikawan, bahwa permasalahan fisika dapat dipecahkan secara lebih cepat dan
akurat. Namun hendaknya pemanfaatan pemelajaran mesin pada bidang Fisika ini harus
terus dikaji dan dipikirkan terus-menerus mengenai pemanfaatan hasilnya.

Seperti halnya Transkrip.xyz, para pihak yang bergelut di bidang Fisika juga
sebaiknya mampu melihat permasalahan apa saja yang dapat diselesaikan secara
lebih efisien dan efektif dengan penggunaan pemelajaran mesin. Dan lebih jauh lagi,
kiranya mampu untuk memberikan kontribusi dari bidang Fisika untuk
perkembangan teknologi ini.

Penulisan skripsi sarjana ini membuahkan banyak cerita, memberikan banyak kenangan
dan kenalan, serta memberikan banyak pelajaran penting. Saya merasa bertumbuh dan
berkembang baik secara kepribadian maupun pengetahuan keilmuannya melalui proses
ini. Maka, saya sangat ingin berterima kasih pada para pihak yang telah membantu
saya menyelesaikan karya ini:

\begin{itemize}
	\item Kedua orangtua, Ibu Cisca Ifke Meyke Langi dan Bapak Daniel Bonifacius
	      Laluyan untuk kesabaran dan keyakinan akan setiap pilihan yang saya ambil.
	      Dukungan dan doa mereka selalu menyertai langkah ini sampai manapun.

	\item Kakak serta adik saya, Diakon Brian Johnathan Laluyan, MSF dan Jennifer
	      'Jenglot' Elisabeth Laluyan, yang dengan caranya masing-masing menunjukkan
	      pengertian dan dukungan mereka masing-masing. Semoga lancar selalu tugas dan
	      pendidikan kalian.

	\item Pembimbing utama saya, Bapak Dr. Iman Santoso, S.Si, M.Sc., yang selalu
	      dengan penuh kesabaran dan ketelitian memberikan masukan dan mengusahakan yang
	      terbaik untuk saya.

	\item Pembimbing praktisi dari BRIN, Sleman, Bapak Ahsani Hafizhu Shali, S.Si.,
	      M.Sc. yang penuh dedikasi tinggi dan kesabaran membimbing saya,
	      mengajarkan banyak hal baru, dan mendorong saya untuk berbuat lebih.
	      Sukses untuk jenjang pendidikan selanjutnya.

	\item Kawan sekaligus lawan, pengagum sekaligus panutan, yang mencinta dan yang
	      dicinta, Kimi Bulan Rumondang Sianipar, S.H. yang selalu sabar dan penuh
	      kasih memberikan arahan akademis, spiritual, tempat makan enak, hingga cara
	      melepas stres. Selalu sukses untuk tugas selanjutnya, \textit{hasiani}.

	\item Kawan-kawan tempat bertumbuh, bekerja, bermain, curhat, dan mencoba hal
	      baru, para \textit{co-worker} di Transkrip.syz: Mas Unies, Yajid, dan
	      Haritz. Semoga lancar segala urusan transkripsi ini.

	\item Para \textit{comrades} di UKM biasa-biasa saja, seluruh awak BPPM Balairung
	      UGM yang menjadi tempat bertumbuh dan percaya bahwa manusia selalu punya pilihan
	      yang tak terbatas, tapi sebaiknya memilih kebenaran.

	\item Kawan-kawan pendamping di sisa akhir masa perkuliahan, kawan-kawan
	      \textit{nongkrong} di PPM: Darren, Carina, dan Marvy. Terima kasih untuk masukan
	      teknis, nonteknis, lawakan, dan spiritual. Semoga indah jalan kalian kedepannya.

	\item Calon rekan sejawat, kawan-kawan fisikawan Fisika UGM 2019 yang selalu
	      bersedia mendukung dan memberikan masukan yang sangat berarti mengenai
	      akademik, saya sebutkan secara khusus: Fariz, tanpanya skripsi ini akan selesai
	      mungkin 6 bulan lagi; Lucky, yang memberikan masukan mengenai LaTex; Yehez,
	      yang menyelamatkan saya di menit-menit akhir penulisan skripsi, yang juga
	      membuka jalan saya berjumpa dengan semangat akademis lewat magang di BRIN,
	      serta kawan kos.
\end{itemize}
Tanpa mereka ini mustahil karya ini selesai. Sekali lagi terima kasih
berlimpah untuk kalian semua, semoga dilancarkan semua urusan kalian. \textit{Te
	amo}.

Akhirnya, kiranya karya tulis ini dapat berguna bagi para pembaca. Semua yang tertulis
di sini merupakan tanggung jawab saya. Apabila di kemudian nanti terdapat
pertanyaan atau sanggahan, \textit{i'm only one call away}.
\vspace{0.8cm}

\begin{tabular}{p{7cm}c}
	 & Yogyakarta, 9 Oktober 2023 \\
	 &                            \\
	 &                            \\
	 & Samuel Johanes
\end{tabular}
%-----------------------------------------------------------------
%-----------------------------------------------------------------

%-----------------------------------------------------------------
%Daftar Isi (TIDAK PERLU DIUBAH)
%-----------------------------------------------------------------
\newpage
\phantomsection
\addcontentsline{toc}{chapter}{DAFTAR ISI}
\makeatletter
\renewcommand{\l@chapter}[2]{\ifnum \c@tocdepth >\z@\addpenalty\@secpenalty\addvspace{0em}
		\setlength{\@tempdima}{1.4em}
		\begingroup\parindent \z@ \rightskip \@pnumwidth\parfillskip -\@pnumwidth\leavevmode
		\bfseries\advance\leftskip\@tempdima\hskip -\leftskip#1\nobreak\ \leaders\hbox{$\m@th\mkern \@dotsep mu\hbox{.}\mkern \@dotsep mu$}\hfil\nobreak\hb@xt@\@pnumwidth{\hss #2}\par\endgroup\fi}
\makeatother
\begin{singlespacing}
	\tableofcontents
\end{singlespacing}
%-----------------------------------------------------------------
%-----------------------------------------------------------------

%-----------------------------------------------------------------
%Daftar Tabel (JIKA DIPERLUKAN, HAPUS KARAKTER % SEBELUM \newpage dan \begin)
%-----------------------------------------------------------------
\newpage
\phantomsection
\addcontentsline{toc}{chapter}{DAFTAR TABEL}
\begin{singlespacing}
	\listoftables
\end{singlespacing}
%-----------------------------------------------------------------
%-----------------------------------------------------------------

%-----------------------------------------------------------------
%Daftar Gambar (JIKA DIPERLUKAN, HAPUS KARAKTER % SEBELUM \newpage dan \begin)
%-----------------------------------------------------------------
\newpage
\phantomsection
\addcontentsline{toc}{chapter}{DAFTAR GAMBAR}
\begin{singlespacing}
	\listoffigures
\end{singlespacing}
%-----------------------------------------------------------------
%-----------------------------------------------------------------

%-----------------------------------------------------------------
%Daftar Lambang (ISI SESUAI DENGAN DATA ANDA!)
%-----------------------------------------------------------------
% \lambang
% \begin{tabular}{cp{10.5cm}}
% %sebagai contoh
%   $Q_i$ &: matriks pembobot \emph{state} subsistem $i$\\
%   $R_i$ &: matriks pembobot masukan subsistem $i$

% \end{tabular}
%-----------------------------------------------------------------
%-----------------------------------------------------------------

%-----------------------------------------------------------------
%Disini awal masukan Intisari (ISI SESUAI DENGAN DATA ANDA!)
%-----------------------------------------------------------------
% \begin{abstractind}
% 	Penyelesaian persamaan diferensial parsial (PDP) merupakan hal yang penting dalam
% 	Ilmu Fisika. Dalam simulasi plasma pada koordinat silinder, potensial listrik
% 	didefinisikan dalam bentuk salah satu jenis PDP, yaitu Persamaan Poisson
% 	dengan sisi kanan merupakan distribusi partikel. Penelitian ini menawarkan pemecahan
% 	solusi 162,5 kali lebih cepat dari metode iteratif Gauss-Seidel menggunakan pendekatan
% 	jaringan saraf (\textit{neural network}) yaitu \textit{fully convolutional
% 		layer} yang terimplementasi pada arsitektur U-Net yang dimodifikasi. Sebagai
% 	domain, digunakan bentuk fisis dari kanal pendorong Hall SPT-100 dengan syarat
% 	batas Dirichlet dan Neumann. Hasil ralat MSE yang dihasilkan dari pelatihan
% 	berada pada orde $10^{-5}$ dan fitur umum yang secara visual sangat
% 	mendekati dengan \textit{ground truth}.
%
% 	Kata kunci: jaringan saraf, convolutional neural network, Poisson
% \end{abstractind}
\begin{abstractind}
	Penyelesaian permasalahan komputasi umumnya memerlukan sistem yang cepat dalam melakukan proses komputasi. Penggunaan GPU merupakan salah satu cara agar proses komputasi bisa lebih cepat dijalankan. Julia merupakan bahasa pemrograman baru yang mempunyai integrasi dengan GPU Nvidia melalui pustaka CUDA.jl. Simulasi perbandingan antara kecepatan eksekusi sistem series oleh CPU dan kecepatan eksekusi sistem paralel oleh GPU diperlukan untuk melihat seberapa jauh perbedaan durasi eksekusi pada operasi - operasi matriks. Hasilnya adalah kecepatan eksekusi sistem series oleh CPU mampu lebih cepat pada operasi matriks sederhana dan pada matriks berukuran kecil. Untuk operasi matriks yang kompleks dan ukuran matriks yang besar, diperoleh eksekusi sistem paralel oleh GPU lebih cepat. Perbedaan durasi paling besar antara eksekusi series oleh CPU dan eksekusi paralel oleh GPU ada pada operasi paling kompleks dan variasi ukuran matriks paling besar.


	Kata kunci: Sistem Paralel, GPU, Julia
\end{abstractind}
%-----------------------------------------------------------------
%-----------------------------------------------------------------

%-----------------------------------------------------------------
%Disini awal masukan untuk Abstract (ISI SESUAI DENGAN DATA ANDA!)
%-----------------------------------------------------------------
\begin{abstracteng}
	% The solution of partial differential equations (PDEs) is of paramount importance
	% in the field of Physics. In plasma simulations in cylindrical coordinates, electric
	% potential is defined in the form of one type of PDE, namely the Poisson
	% Equation with the right-hand side representing the particle distribution. This
	% research offers a solution that is 162.5 times faster than the Gauss-Seidel iterative
	% method using a neural network approach, specifically the fully convolutional
	% layer implemented on a modified U-Net architecture. As a domain, the
	% physical shape of the Hall thruster SPT-100 channel with Dirichlet and
	% Neumann boundary conditions is used. The resulting mean squared error (MSE) from
	% the training is on the order of $10^{-5}$, and the common features are
	% visually very close to the ground truth.

	Computational problem-solving generally requires a system that can perform computational processes quickly. The use of GPUs is one way to accelerate these computational processes. Julia is a new programming language that integrates with Nvidia GPUs through the CUDA.jl library. A simulation comparing the execution speed of serial systems by the CPU and the parallel system execution speed by the GPU is needed to observe the differences in execution duration for matrix operations. The results show that the serial execution speed by the CPU is faster for simple matrix operations and smaller matrices. For complex matrix operations and larger matrices, the parallel execution by the GPU is faster. The most significant difference in duration between serial execution by the CPU and parallel execution by the GPU is found in the most complex operations and the largest matrix variations.

	Keyword: Parallel Systems, GPU, Julia
\end{abstracteng}
%-----------------------------------------------------------------
%-----------------------------------------------------------------

%-----------------------------------------------------------------
%Disini awal masukan untuk Bab (ISI BAB I DAPAT DIEDIT DI FILE Bab1.tex,
%ISI BAB II DAPAT DIEDIT DI FILE Bab2.tex, dst...)
%-----------------------------------------------------------------
\definecolor{codeblue}{HTML}{0B60B0}
\NewDocumentCommand{\cw}{O{\footnotesize}v}{ \textbf{\texttt{\textcolor{codeblue}{#1#2}}}}

\chapter{PENDAHULUAN}
\pagenumbering{arabic}
\setcounter{page}{1}
\section{Latar Belakang Masalah}
\label{latarbelakang}

% [Komputer]

Penggunaan komputer tidak bisa lepas dari kehidupan manusia saat ini. Hampir
semua manusia mempunyai komputer. Komputer yang digunakan ini berbentuk smartphone, laptop, hingga
komputer berbentuk \emph{microcontroller} seperti Arduino atau
\emph{microprocessor} seperti Raspberry Pi. Kebergantungan manusia akan
komputer ini disebabkan karena tingginya kecepatan komputer dalam melakukan
komputasi. Dari komputasi sederhana seperti pada aplikasi kalkulator,
hingga komputasi kompleks seperti \emph{event handler} pada aplikasi mobile.

% [CPU]

Kecepatan komputer dalam melakukan proses komputasi Tinggi disebabkan karena
arsitektur dari komputer sendiri yang mempunyai komponen utama bernama
\emph{Central Processing Unit} atau \emph{CPU}. CPU ini bertindak sebagai
\emph{"Control Center"} yang mana menginterpretasi, memproses, dan mengeksekusi
instruksi dari software dan hardware \citep{armWhatCentralProcessing2023}. Hal
ini menyebabkan pemilihan CPU menjadi hal yang krusial ketika ingin mendapatkan
proses komputasi yang cepat.

% [Komputasi Numerik]

Manfaat dari munculnya komputer tidak hanya dirasakan oleh masyarakat umum,
namun juga dirasakan oleh akademisi, peneliti, maupun pekerja di suatu
industri. Banyak aplikasi yang jauh lebih mudah dikerjakan jika dilakukan
menggunakan komputer. Bahkan saat ini muncul bidang bidang baru yang
menggunakan komputer sebagai dasar dari suatu komputasi, seperti \emph{Data Science},
\emph{Data Analyst}, \emph{Machine Learning Engineering}, dan masih
banyak lagi. Bidang - bidang tersebut sangat bergantung pada proses komputasi
yang cepat, sehingga memerlukan tinggi kecepatan CPU dalam melakukan kalkulasi.

Dalam lingkup penelitian, khusus nya penelitian di bidang metode numerik,
proses kalkulasi yang cepat dapat memudahkan para peneliti untuk mendapatkan
hasil yang diinginkan. Untuk itu, dengan adanya komputer, para peneliti dapat
menghasilkan metode numerik baru yang lebih berat dalam proses komputasi, namun
memiliki nilai yang lebih akurat. Proses komputasi ini juga pastinya sangat
bergantung pada CPU yang digunakan. Sehingga secara tidak langsung penggunaan
CPU juga mempengaruhi akurasi suatu kalkulasi berdasarkan suatu metode yang
digunakan.

% [CPU]

Pada komponen CPU, terdapat inti atau \emph{core} yang mana umumnya berjumlah
kurang dari 10. Salah satu hal yang mempengaruhi kecepatan komputasi CPU jumlah inti yang ada di dalam CPU. Namun, CPU sendiri juga memiliki aturan dalam banyak
nya inti yang digunakan. CPU didesain untuk menyelesaikan permasalahan
komputasi dari yang mudah hingga kompleks. Berdasarkan
\cite{kukunasChapterIntelPentium2015}, CPU sendiri mempunyai desain arsitektur
\emph{Complex Instruction Set Computer} atau \emph{CISC} untuk menyelesaikan komputasi
yang kompleks. Kemudahan dalam pemberian instruksi ini yang menjadi alasan
mengapa CPU sengaja menggunakan inti yang sedikit.

Proses kalkulasi yang memuat banyak data atau perhitungan, umumnya membutuhkan
inti CPU yang banyak agar hasil lebih cepat diperoleh. Namun, dikarenakan inti
CPU berjumlah sedikit, kecepatan kalkulasi ini tidak dapat dimaksimalkan lagi
pada suatu perangkat. Beberapa solusi lain agar kalkulasi dijalankan pada inti
yang lebih banyak, diantara nya adalah dengan menyewa penyedia server kalkulasi
seperti \emph{Google Colab}. Kekurangan menyewa penyedia server kalkulasi ini
yang pasti adalah perlu nya biaya tambahan ketika ingin menambah inti yang akan
digunakan. Alternatif lain yang dapat dilakukan supaya mendapat inti yang
banyak dengan biaya yang gratis adalah dengan menggunakan komponen lain dari
perangkat komputer yang bernama GPU.

% [GPU]

GPU atau \emph{Graphics Processing Unit} merupakan pemrosesan kalkulasi yang
ditujukan untuk me-\emph{render} suatu gambar atau \emph{3D computer graphics}
ke layar perangkat komputer \citep{armWhatGraphicsProcessing2023}. Pada
kebanyakan kasus, penggunaan GPU ini akan sangat berpengaruh ketika memainkan
game yang mempunyai tampilan yang baik, seperti model di gamenya mempunyai
jumlah \emph{mesh} yang banyak atau juga bahkan pencahayaan di game mempunyai
metode yang membutuhkan kalkulasi yang tinggi. Mampunya GPU untuk melakukan
kalkulasi yang berat ini disebabkan karena banyaknya inti yang ada di GPU. Beda
nya jumlah inti pada CPU dan GPU disebabkan karena perbedaan fungsi antara
kedua komponen tersebut. CPU lebih ditujukan pada kalkulasi yang kompleks
dengan jumlah kalkulasi yang lebih sedikit, sedangkan GPU lebih ditujukan untuk
kalkulasi yang tidak kompleks dengan jumlah kalkulasi yang banyak
\citep{intelCPUVsGPU2023}. Meskipun GPU mempunyai fungsi dasar untuk melakukan
\emph{rendering} ke layar perangkat komputer, namun GPU sendiri juga bisa
digunakan untuk melakukan kalkulasi tanpa perlu nya melakukan \emph{rendering}
ke layar komputer. Penggunaan GPU untuk melakukan komputasi umum ini biasaya
disebut dengan \emph{General Purpose GPU} atau GPGPU
\citep{gigabyteWhatGPGPUWhy2023}. Penggunaan GPGPU ini ditujukan agar tercapai
nya \emph{High Performance Computing} atau HPC yang mana merupakan komputasi
yang mampu melakukan perhitungan pada kecepatan tinggi
\citep{gigabyteHPCHighPerformance2023}.

% [Julia]

Penggunaan GPGPU pada dasarnya mempunyai kekurangan yang membuat banyak
pengguna memilih alternatif lain daripada GPGPU, yakni kompleks nya
implementasi ketika ingin menggunakan GPGPU. Implementasi yang kompleks ini
muncul karena GPU mempunyai jumlah inti yang banyak. Beberapa \emph{Application
	Programming Interface} atau API \citep{evansonExplainerWhatAPI2021} yang dapat
membuat penggunaan GPGPU lebih mudah, diantara nya adalah \emph{Vulkan API},
\emph{Direct3D}, dan \emph{OpenGL}
\citep{khairySurveyArchitecturalApproaches2019}. Namun, kedua API tersebut pada
dasarnya merupakan \emph{Graphics API} yang dituliskan dalam bahasa C/C++ yang
mana membutuhkan penyesuaian lagi jika ingin digunakan untuk proses kalkulasi
tanpa \emph{rendering}. Ditambah lagi, bahasa C/C++ mempunyai \emph{syntax} dan
paradigma yang relatif lebih susah jika dibandingkan dengan bahasa lain seperti
Python. Salah satu alternatif bahasa pemrograman yang dapat digunakan adalah
Julia. Julia mempunyai \emph{syntax} dan paradigma yang tidak terlalu kompleks seperti
Python, dan memiliki performa yang baik seperti C++
\cite{bezansonJuliaMicroBenchmarks2023}. Kombinasi \emph{syntax} dan performa inilah
yang membuat Julia menjadi bahasa pemrograman paling cocok untuk implementasi
GPGPU pada saat ini.

\section{Rumusan Masalah}
Berdasarkan latar belakang yang telah dipaparkan, maka rumusan masalah yang
diangkat pada penelitian ini adalah sebagai berikut:
\begin{enumerate}
	\item Bagaimana cara kerja komputasi sistem series pada CPU dan komputasi sistem paralel pada GPU dengan menggunakan
	      Bahasa Julia

	\item Bagaimana performa yang dihasilkan dari komputasi secara \emph{parallel}
	      dibandingkan dengan komputasi secara \emph{series}
\end{enumerate}

\section{Tujuan Penelitian}
\begin{enumerate}
	\item Mempelajari cara kerja komputasi sistem paralel pada GPU dengan menggunakan Bahasa Julia
	      untuk menyelesaikan permasalahan pada metode komputasi

	\item Membandingkan performa komputasi sistem paralel pada GPU dengan komputasi sistem series pada CPU
\end{enumerate}

\section{Batasan Masalah}
\begin{enumerate}
	\item Pada kajian ini, sistem series dan paralel menggunakan bahasa yang sama, yakni
	      Bahasa Julia, serta kajian ini tidak akan membandingkan performa sistem series
	      dan paralel pada bahasa lain

	\item Perbandingan performa komputasi series dan komputasi parelel dilakukan dalam
	      lingkup operasi berbasis matriks
\end{enumerate}

\section{Manfaat Penelitian}
\begin{enumerate}
	\item Diharapkan penelitian ini dapat membawa pandangan baru bahwa Bahasa Julia
	      mempunyai penulisan kode yang sederhana ketika dijalankan di sistem series
	      maupun sistem paralel

	\item Diharapkan hasil penelitian ini dapat menunjukkan komputasi paralel dengan
	      menggunakan Bahasa Julia mempunyai performa yang lebih baik seiring
	      meningkatnya ukuran data, daripada komputasi series
\end{enumerate}

\chapter{TINJAUAN PUSTAKA}\label{tipus}
Pemelajaran mendalam (\emph{deep learning}) sebagai sebuah pemahaman telah lahir sejak tahun 1940an. Namun pemelajaran mendalam sebagai istilah yang tren memang baru muncul belakangan. Setidaknya ada tiga gelombang pengembangan pemelajaran mendalam yang ditandai dengan nomenklaturnya \citep{goodfellow_bengio_courville_2016}. Gelombang pertama (1940-1960an), istilah pemelajaran mendalam dikenal dengan \emph{cybernetics} yang merupakan pengembangan dari teori pola pemelajaran biologis \citep{mcculloch_pitts_1943, morris_1999} dan implementasi pertama pada model \emph{perceptron} \citep{rosenblatt_1958}.

Pada gelombang kedua (1980-1995), dikenal dengan istilah \emph{connectionism} yang menggunakan pendekatan \emph{connectionist} dengan gebrakan baru yaitu perambatan mundur (\emph{back-propagation}) untuk melatih jaringan saraf (\emph{neural network}) dengan satu atau dua lapisan tersembunyi \citep{rumelhart_hinton_williams_1986}. Dan akhirnya gelombang ketiga atau masa kontemporer dimulai pada sekitar tahun 2006 hingga sekarang yang dikenal dengan \emph{deep learning} \citep{hinton_osindero_teh_2006, NIPS2006_5da713a6}.

Ilmu saraf memberikan wawasan yang banyak mengenai cara bekerja sebuah jaringan saraf biologis untuk dikembangkan ke dalam algoritma jaringan saraf buatan untuk belajar memecahkan beragam tugas yang berbeda \citep{goodfellow_bengio_courville_2016}. Salah satunya adalah penjelasan dari ilmuwan saraf bagaimana musang dapat belajar melihat dengan daerah pemrosesan pendengaran di otaknya jika otaknya diatur ulang untuk mengirimkan sinyal visual ke area tersebut \citep{von_melchner_pallas_sur_2000}. Hal ini kemudian menunjukkan bahwa mayoritas otak mamalia menggunakan algoritma tunggal untuk memecahkan kebanyakan tugas yang berbeda. Dari cara kerja tersebut, terinspirasilah ide utama untuk memiliki banyak unit komputasional yang menjadi pintar melalui interaksi dengan unit lain \citep{goodfellow_bengio_courville_2016}.

Neocognitron \citep{fukushima_1980} memperkenalkan model arsitektur yang sangat bagus untuk memroses gambar yang terinspirasi dari sistem visual mamalia dan yang kemudian menjadi cikal bakal ide untuk pengembangan jaringan saraf konvolusi modern (\emph{modern convolutional neural network} \citep{726791, lecun_kavukcuoglu_farabet_2010}.

Penyelesaian persamaan diferensial parsial menggunakan metode jaringan saraf buatan dimulai sejak tahun 1990an yang dimulai oleh \cite{lee_kang_1990} dengan artikelnya yang berjudul '\emph{Neural algorithm for solving differential equation}' yang terbit di \emph{Journal of Computational Physics}. Kemudian pada dekade yang sama muncul beberapa artikel yang membahas tentang topik ini, yaitu oleh Dissanayake dan Phan-Tien pada tahun 1994 dan Lagaris, dkk. pada tahun 1998. 

Dalam \citep{lagaris1998}, solusi persamaan diferensial dalam koordinat kartesian yang coba diselesaikan merupakan penjumlahan dari dua bagian. Bagian yang pertama merupakan permasalahan syarat batas atau syarat awal dan mengandung parameter yang tidak dapat diubah-ubah. Dan bagian yang kedua adalah bagian yang tidak memiliki hubungan dengan bagian syarat awal/syarat batas dan mengandung perambatan maju dari jaringan saraf buatan. Pengaplikasian metode ini dapat digunakan pada persamaan diferensial biasa tunggal, persamaan diferensial biasa berpasangan, dan persamaan diferensial parsial. Metode ini dikomparasikan dengan solusi dari metode numerik Galerkin \emph{finite element}.

Usaha-usaha awal penggunaan jaringan saraf buatan pada penyelesaian persamaan diferensial parsial ini sangat memanfaatkan algoritma perambatan mundur karena memberikan metode yang sangat akurat untuk menghitung ralat. Namun usaha-usaha ini memiliki kendala yang kurang lebih sama, yaitu hanya mampu menangani fungsi ruas kanan dan syarat batas tertentu saja.

Dengan perkembangan kekuatan komputasi pada tahun 2000an, dimungkinkan untuk membuat model yang lebih rumit, dan parameter yang lebih banyak, serta jumlah lapisan (\emph{layer}) yang lebih banyak. \cite{Smaoui2004} menggunakan jumlah daya komputasional yang lebih besar untuk menginvestigasi model yang lebih mendalam. menggunakan \emph{multilayer perceptron} untuk memprediksi dinamika dari dua persamaan diferensial parsial nonlinear menggunakan dekomposisi Kahunen-Loeve dan jaringan saraf buatan. \cite{baymani2010} mengembangkan jaringan saraf buatan untuk penyelesaian permasalahan Stokes. Permasalahan Stokes campuran ditransformasi ke dalam tiga permasalahan Poisson koordinat kartesian yang kemudian diselesaikan untuk mendapatkan hasil dari permasalahan Stokes tersebut. Hasil yang didapat dari metode jaringan saraf buatan ini kemudian dikomparasikan dengan metode numerik dari penelitian lainnya dan hasil eksaknya. Dari penelitian tersebut didapatkan bahwa pendekatan jaringan saraf tiruan yang baru memberikan hasil yang memiliki akurasi yang lebih tinggi dan jumlah parameter yang digunakan lebih sedikit dari model konvensional.

\cite{DBLP:journals/corr/abs-1711-10561} memperkenalkan jaringan saraf terinformasi fisika (\emph{physics informed neural networks} (PINN)), yaitu jaringan saraf yang dilatih untuk menyelesaikan tugas-tugas pembelajaran yang diawasi dengan tetap memperhatikan hukum fisika tertentu yang dijelaskan oleh persamaan diferensial parsial nonlinier umum. Pengembangan tersebut berada pada koridor untuk memecahkan dua masalah utama yaitu solusi berbasis data dan penemuan berbasis data dari persamaan diferensial parsial.  

Seiring dengan perkembangan model \emph{deep learning}, suatu terobosan dalam penyelesaian persamaan diferensial parsial menggunakan jaringan saraf buatan kemudian ditemukan. Para peneliti kemudian mengembangkan model jaringan saraf buatan konvolusi (\emph{convolutional neural network}/ CNN) dengan hipotesis bahwa dengan menggunakan CNN, parameter yang digunakan akan lebih sedikit sehingga akan banyak melakukan penghematan sumber daya komputer serta akan memberikan hasil yang lebih akurat. Selain itu, menurut \cite{Li_Li_Gao}, CNN memiliki kemampuan yang lebih baik untuk mengenali masukan gambar.

\cite{shan_2020_study} mengembangkan model penyelesaian persamaan Poisson menggunakan CNN untuk memprediksi potensial listrik  dengan variasi pada eksitasi dan distribusi permitivitas pada bidang kartesian 2 dimensi dan 3 dimensi. Penelitian ini menggunakan desain \emph{cost function} yang dikustomisasi dan data yang didapat dari metode numerik beda-hingga. Model yang dikembangkan dapat melakukan peforma yang cukup efektif dibandingkan dengan metode numerik yang digunakan untuk membentuk data latih, dengan rata-rata ralat prediksi kurang dari 3\%.

Perkembangan yang cukup signifikan dilakukan oleh \cite{Ozbay2021} dengan mengembangkan arsitektur CNN untuk menyelesaikan persamaan Poisson 2 dimensi pada koordinat kartesian dengan variasi resolusi yang diberikan oleh ruas kanan persamaan, sembarang syarat batas, dan variasi parameter grid. Permasalahan syarat batas diselesaikan dengan metode baru, mendekomposisikan persaman Poisson yang asli ke dalam satu persamaan Poisson homogen dan empat submasalah Laplace nonhomogen. Model yang dikembangkan terbukti dapat mengungguli model dengan jaringan saraf tiruan konvensional dan dapat memprediksi dengan rata-rata presentase ralat di bawah 10\%. 

\cite{cheng2021using} menyelesaikan persamaan Poisson 2 dimensi dengan syarat batas Dirichlet nol menggunakan lapisan konvolusi secara penuh dengan arsitektur U-Net \citep{DBLP:journals/corr/RonnebergerFB15} yang didefinisikan dengan variasi pada jumlah percabangan, kedalaman, dan medan reseptif. Pada penelitian tersebut didapati bahwa medan reseptif memiliki peran yang penting dalam menangkap struktur topologis dari masukan pada tiap lapisan. Untuk menghitung syarat batas dan interior, digunakan dua syarat batas, yaitu \emph{dichlet loss} dan \emph{inside loss} serta syarat batas alternatif yaitu \emph{laplacian loss}. Untuk pelatihan digunakan dataset dengan distribusi muatan acak dan dataset dengan dengan distribusi mengikuti aturan deret Fourier.

Sebagai domain fisis, dalam penelitian ini akan digunakan model pendorong Hall (\emph{Hall thruster}) sebagai perwakilan masalah fisis perhitungan potensial listrik di koordinat silinder. \cite{braga_miranda_2019} dari Laboratorium Fisika Plasma Universitas Brasil (LFP-UnB) sejak 2004 sedang mengembangkan pendorong Hall yang disebut PHall yang memiliki perbedaan dalam dimensi kanal, parameter operasi, dan mekanisme pembentukan medan magnet yang dibutuhkan dengan SPT-100. SPT-100 dijadikan sebagai patokan dan pembanding dalam pembangunannya. Model 2 dimensi, spesifikasi serta, gambaran simulasi tentang SPT-100 dijelaskan dalam karya mereka.

Penelitian ini akan memecahkan masalah persamaan diferensial parsial berupa persamaan Poisson yang diterapkan pada domain fisis pedorong Hall SPT-100 pada koordinat silinder dua dimensi menggunakan metode jaringan saraf konvolusi dengan \emph{ground truth} yang dihitung menggunakan metode numerik Gauss-Seidel.
\chapter{DASAR TEORI}

\section{Pengantar GPU dan \emph{General Purpose GPU} (GPGPU)}

\subsection{Sejarah dan perkembangan GPU}

% #### Awal Mula dan Evolusi Awal

Dalam sejarahnya, \emph{Graphics Processing Unit} (GPU) awalnya dirancang untuk
mempercepat pemrosesan gambar pada komputer. Pemrosesan gambar yang memerlukan
GPU pada awalnya hanya dibutuhkan untuk permainan video game. Pada tahun
1970-an hingga 1980-an, beberapa perangkat yang mengadopsi sistem paralel
seperti GPU diantara nya adalah Atari 2600 dan ARTC HD63484
\citep{wikipediaGraphicsProcessingUnit2023}.

Perkembangan GPU untuk komputer pribadi dimulai pada tahun 1980-an. Pada awal
tahun tersebut, terdapat produk NEC µPD7220 yang menjadi implementasi pertama
dari GPU yang dapat digunakan pada komputer pribadi. Produk NEC µPD7220 ini
mempunyai harga terjangkau dan kualitas grafis yang tinggi, sehingga membuat
GPU ini terkenal hingga pertengahan tahun 1980-an. Pada tahun 1985, muncul
produk baru bernama Amiga dengan menggunakan kustom chip grafis. Amiga ini
dapat digunakan untuk menggambar garis, memanipulasi bitmap, dan melakukan
isian warna pada suatu area \citep{wikipediaGraphicsProcessingUnit2023}.

Dalam perkembangannya, NVIDIA memperkenalkan GPU pertama mereka yang bernama
NVIDIA GeForce 256. GPU ini mampu melakukan pemrosesan grafis secara
\emph{real-time}, dan mampu melakukan banyak perhitungan \emph{floating-point} dari
tahap \emph{shading vertex} hingga tahap \emph{shading fragment}. NVIDIA GeForce 256 juga
memiliki \emph{bandwidth} memori yang tinggi
\citep{dallyEvolutionGraphicsProcessing2021}. Pada tahun berikutnya, NVIDIA
memperkenalkan NVIDIA GeForce 8 dengan unit pemrosesan stream generik yang
baru. NVIDIA GeForce 8 menunjukkan kemajuan performa komputasi yang jauh lebih
baik daripada CPU. Dari perbedaan performa inilah kemudian para peneliti
menemukan cara untuk mengubah \emph{shader vertex} dan \emph{shader fragment}, menjadi suatu
data yang dapat diolah oleh GPU secara paralel. Proses penggunaan GPU untuk
melakukan komputasi umum secara paralel disebut dengan \emph{General Purpose
	GPU} atau GPGPU. Beberapa bidang yang menggunakan GPGPU ini diantara lain
adalah bidang pembelajaran mesin, eksplorasi minyak, aljabar linear, statistik,
rekonstruksi 3D, dan penetapan harga opsi saham
\citep{wikipediaGraphicsProcessingUnit2023}.

Pada perkembangannya, GPU telah mengalami perubahan bentuk ukuran, daya,
kinerja, dan harga. Hal ini membuat GPU membukakan jalan bagi inovasi lebih
lanjut di berbagai bidang. Dari yang awalnya hanya digunakan untuk bermain
video game saja, menjadi komponen yang diperlukan di smartphone, \emph{virtual
	reality}, pembelajaran mesin, mobil tanpa pengemudi, pesawat ruang angkasa,
robot, perangkat rumah pintar, dan banyak bidang lainnya
\citep{businesswireHistoryGPUInception2023}.

\subsection{Transisi GPU dari pemrosesan grafis ke GPGPU}

% #### Kemajuan Teknologi dan Penemuan GPGPU
Perkembangan GPGPU menjadi lebih praktis dan populer setelah tahun 2001, dengan
munculnya \emph{shader} yang dapat diprogram dan mendukung \emph{floating point} pada
prosesor grafis. Masalah yang melibatkan vektor dan matriks yang mempunyai
n-dimensi dapat dengan mudah diterjemahkan dan diproses oleh GPU. Pada tahun
2003, ditemukan pendekatan berbasis GPU untuk solusi masalah aljabar linear
umum. Pendekatan pada GPU ini berjalan lebih cepat daripada pendekatan yang
menggunakan CPU. Perbedaan kecepatan ini merupakan tonggak awal GPGPU menjadi
banyak digunakan untuk masalah komputasi umum
\citep{wikipediaGeneralpurposeComputingGraphics2023}.
\cite{pharrGPUGemsProgramming2005} menyatakan bahwa pada bidang ilmu komputer
dan visualisasi, GPGPU juga menunjukkan hasil yang baik. Hasil yang baik ini
dapat dicapai dengan menerapkan GPU pada masalah komputasi data-paralel.

% #### Arsitektur GPU dan Pemrograman GPGPU
Memaksimalkan kinerja dari GPU memerlukan pemahaman tentang arsitekturnya. GPU
dirancang untuk grafis komputer yang memiliki gaya komputasi paralel tinggi dan
menghasilkan output piksel berwarna dari elemen data yang independen. Desain
ini adalah kunci saat memprogram GPU, baik untuk grafis maupun komputasi umum.
Pemrograman GPU untuk masalah komputasi umum awalnya tertanam dalam API
(\emph{Application Programming Interface}) dan bahasa pemrograman grafis
komputer. Dalam perkembangannya, API dan bahasa pemrograman grafis
disederhanakan, dimana menghasilkan API seperti CUDA dari NVIDIA, DirectCompute
dari Microsoft, dan OpenCL dari Apple/Khronos Group. API tersebut memungkinkan
penggunaan GPU tanpa memerlukan konversi data eksplisit ke bentuk grafis
\citep{pharrGPUGemsProgramming2005}.

% #### Kasus Penggunaan dan Contoh Aplikasi
Aplikasi yang mendapat manfaat paling besar dari pemrosesan GPU adalah aplikasi
yang memiliki intensitas aritmatika yang tinggi, seperti solusi sistem
persamaan linear dan simulasi berbasis fisika. Contoh lain nya adalah simulasi
aliran dengan batasan kompleks dan algoritma jarak terpendek untuk semua
pasangan, seperti yang digunakan dalam prediksi struktur protein. Jenis
komputasi ini berkinerja baik pada GPU karena sifatnya yang sangat
data-paralel: mereka terdiri dari aliran besar elemen data, di mana \emph{kernel}
komputasi yang identik diterapkan \citep{pharrGPUGemsProgramming2005}.

\subsection{Arsitektur dan karakteristik GPU}

% 2. Arsitektur GPU: Perbandingan dengan CPU

\begin{figure}[H]
	\centering
	\includegraphics[width=10cm]{images/cpu-vs-gpu-cores.png}
	\caption{Jumlah Inti CPU dan Inti GPU}
	\label{gambar gpu-cpu cores}
\end{figure}

Secara arsitektur, GPU memiliki arsitektur yang hampir sama dengan CPU. Namun
CPU berfokus pada kecepatan akses memori \emph{cache}. Sedangkan GPU dirancang untuk
melakukan komputasi data secara paralel, sehingga dibutuhkan sebanyak mungkin
inti yang tersedia, seperti yang terlihat pada Gambar \ref{gambar gpu-cpu cores}.
Arsitektur yang digunakan oleh GPU biasanya disebut dengan arsitektur
stream SIMD (\emph{Single Instruction Multiple Data}) yang mana memanfaatkan
banyak inti dengan pemrosesan ringan \citep{helenGPUArchitectureStructure2020}.

% 4. Fitur Arsitektural Utama dan Implikasinya

% Fitur arsitektural utama GPU mencakup penggunaan inti pemrosesan yang banyak dan
% ringan, yang berbeda dari CPU yang menggunakan inti yang lebih sedikit tetapi lebih
% kuat. Implikasi dari arsitektur ini terhadap pembuatan program untuk GPGPU
% melibatkan pertimbangan khusus dalam hal bagaimana data dan tugas-tugas
% diorganisir dan dijalankan, serta jenis perangkat lunak yang paling sesuai untuk
% perangkat ini.

% 3. Model Komputasi Streaming dan 'Gap' Memori

Salah satu faktor penting dalam evolusi GPU adalah '\emph{gap}' memori, yaitu
perbedaan antara kecepatan komputasi terhadap kecepatan akses memori. Model
komputasi '\emph{streaming}' merupakan pendekatan yang paling cocok untuk GPU
modern. Ini dikarenakan model ini mempunyai efisiensi yang tinggi dalam
pemrosesan paralel \citep{pharrGPUGemsProgramming2005}.

% 5. GPU dalam High Performance Computing (HPC)

Dalam konteks \emph{High Performance Computing} (HPC), GPU kini menjadi pilihan utama untuk
mempercepat beban kerja komputasi. Penggunaan GPGPU telah meluas ke berbagai
bidang, termasuk pemodelan 3D, infrastruktur VDI, dan lebih lanjut, menunjukkan
fleksibilitas dan kekuatan GPU dalam berbagai skenario komputasi
\citep{hagoortExploringGPUArchitecture2023}.

\begin{figure}[H]
	\centering
	\includegraphics[width=14cm]{images/cpu-gpu-architecture.png}
	\caption{Arsitektur CPU dan GPU}
	\label{gambar gpu-cpu architecture}
\end{figure}

Berdasarkan \cite{learningUnderstandingArchitectureGPU2023}, arsitektur CPU dan
GPU dapat digambarkan skema nya seperti pada Gambar \ref{gambar gpu-cpu
	architecture}. Warna hijau menggambarkan unit komputasi atau \emph{cores}
(inti), warna orange menggambarkan memori, dan warna kuning menggambarkan unit
kontrol.

\subsubsection{Unit Komputasi}
\label{unit komputasi}

Pada Gambar \ref{gambar gpu-cpu architecture}, terlihat bahwa inti dari CPU
mempunyai ukuran yang lebih besar dari pada inti GPU. Beda ukuran ini
menggambarkan bahwa inti CPU mempunyai kecerdasan yang lebih tinggi daripada
inti GPU. Ini berarti inti CPU mampu mengatasi berbagai macam tugas kompleks
yang diperlukan oleh komputer. Pada Gambar \ref{gambar gpu-cpu architecture}
terlihat juga bahwa inti dari CPU mempunyai jumlah yang jauh lebih sedikit dari
pada GPU. Perbedaan ini dikarenakan memang GPU memiliki inti dengan jumlah yang
jauh lebih banyak daripada CPU.

Seiring berjalan nya waktu, inti CPU mempunyai evolusi yang semakin baik dalam
segi performa. Sebaliknya, inti GPU mempunyai perubahan yang semakin menurun.
Penurunan inti GPU ini bertujuan agar penggunaan inti GPU dapat mengkonsumsi
daya yang lebih rendah. Sehingga GPU akan bisa digunakan di berbagai perangkat,
termasuk perangkat mobile.

Dalam eksekusi nya, CPU menggunakan konsep \emph{out of order execution} (OOE),
yang mana instruksi pada CPU dapat segera diproses setelah sumber daya yang
diperlukan tersedia. Instruksi yang dijalankan inipun tidak perlu menunggu
instruksi sebelumnya selesai. Model eksekusi ini bertujuan untuk meningkatkan
pemanfaatan sumber daya dan mengurangi waktu tunggu eksekusi.

Berbeda dengan CPU, GPU tidak menggunakan konsep OOE, dan hanya menjalankan
instruksi saja. Tidak menggunakan OOE ini bertujuan untuk tetap menyederhanakan
arsitektur unit komputasi pada GPU. Selain itu, GPU memang didesain untuk
mengatasi tugas yang tidak kompleks. Pada dasarnya, semua tugas yang
menggunakan GPU hanya menggunakan operasi penjumlahan dan perkalian.

Operasi penjumlahan dan perkalian pada GPU secara umum menggunakan dua metode,
yakni \emph{Multiply-Add} (MAD) dan \emph{Fused Multiply-Add} (FMA). Pada
Gambar \ref{gambar mad-and-fma}, terlihat bahwa perbedaan utama antara MAD dan
FMA adalah ada di proses nya. Proses perkalian dan penjumlahan MAD dilakukan
secara berurutan yang mana menyebabkan ketelitian dari hasilnya akan berkurang.
Sedangkan pada FMA, setelah proses perkalian dan penjumlahan dilakukan
sekaligus yang menyebabkan hasilnya memiliki ketelitian yang tinggi. Dalam
kinerja nya, operasi FMA memungkinkan GPU untuk dapat meningkatkan kinerja yang
signifikan. Namun pada implementasi nya, operasi FMA memerlukan hardware khusus
untuk optimalisasi, sedangkan MAD bisa lebih umum dan tidak spesifik pada
perangkat keras tertentu.

\begin{figure}[H]
	\centering
	\includegraphics[width=10cm]{images/mad-and-fma-2.png}
	\caption{MAD dan FMA pada operasi GPU}
	\label{gambar mad-and-fma}
\end{figure}

Meskipun GPU terlihat hanya bisa melakukan operasi sederhana, namun operasi
sederhana ini dapat berupa operasi penjumlahan dan perkalian yang kompleks yang
mana bisa membentuk operasi pada tensor dan operasi pada \emph{ray-tracing}
seperti yang terlihat pada Gambar \ref{gambar ray tracing technique}. Kemampuan
utama GPU dalam menghasilkan \emph{render} yang bagus seperti pada Gambar
\ref{gambar ray tracing technique} pada dasarnya terletak pada kemampuan GPU
untuk melakukan operasi secara paralel. Dalam melakukan operasi secara paralel
ini, GPU menggunakan model \emph{Single Instruction Multiple Data} (SIMD) di
mana semua inti nya semua core menjalankan operasi yang sama pada data yang
berbeda.

\begin{figure}[H]
	\centering
	\includegraphics[width=5cm]{images/rt-setup2.png}
	\includegraphics[width=5cm]{images/Reflections_02.png}
	\caption{Teknik \emph{Ray Tracing} dan Hasil \emph{Render} dengan menggunakan
		Teknik \emph{Ray Tracing}}
	\label{gambar ray tracing technique}
\end{figure}

\subsubsection{Memori}

Perbedaan model memori pada CPU dan GPU telah terlihat pada Gambar \ref{gambar
	gpu-cpu architecture} dengan blok yang berwarna orange. Sistem memori CPU
menggunakan \emph{Dynamic Random Access Memory} (DRAM) dengan ukuran yang
bervariasi. Komputer pribadi biasanya mempunyai ukuran DRAM berkisar 4GB hingga
128GB.

Komponen kunci lainnya dari memori CPU adalah \emph{cache}, yang mana memiliki
tujuan untuk mempercepat waktu akses antara CPU ke DRAM. Kecepatan akses ini
disebabkan karena letak CPU yang sangat dekat dengan \emph{cache}. \emph{Cache}
ini memiliki ukuran yang cukup kecil, biasanya berkisar puluhan KB per inti
CPU. \emph{Cache} umumnya terdiri dari tiga level, yakni Level 1 (L1), Level 2
(L2), dan Level 3 (L3). L1 memiliki letak paling dekat dengan CPU namun
memiliki ruang penyimpanan yang paling kecil (biasanya berkisar 64KB inti),
sedangkan L3 memiliki letak yang paling jauh dengan CPU namun memiliki ruang
penyimpanan yang paling besar daripada \emph{cache} lainnya (biasanya berkisar
4MB per inti). Proses CPU untuk mengambil data dari memori mempunyai urutan:
pencarian pada L1, pencarian pada L2, pencarian pada L3, pencarian pada DRAM.

Sementara itu, GPU menggunakan memori yang bertipe DRAM juga yang disebut
dengan \emph{Global Memory} atau GMEM. Ukuran GMEM ini biasanya lebih kecil
dibandingkan DRAM pada CPU. GMEM biasanya memiliki ukuran yang berkisar 2GB
pada kartu grafis (GPU) ekonomis, hingga ukuran 40GB atau 80GB pada kartu
grafis yang lebih mahal.

% \section{Pemrograman GPU}
% \subsection{Dasar-dasar pemrograman GPU}
% \subsection{Perbandingan antara pemrograman CPU dan GPU}
% \subsection{Alat dan teknologi yang digunakan dalam pemrograman GPU}

\section{Bahasa Pemrograman Julia}

Bahasa Julia dirilis pada tahun 2012 dan telah diterima secara luas di kalangan
data scientist dan para penggiat matematika. Bahasa ini diciptakan karena
adanya kebutuhan akan bahasa pemrograman yang ideal untuk pengkodean aritmatika
\citep{ismiPentingUntukData2021}. Tujuan utama Julia adalah mengkombinasikan
kemudahan dalam penggunaan, kecepatan, dan efisiensi dalam satu bahasa
pemrograman. Julia memiliki sifat pengetikan opsional, mengadopsi berbagai
bahasa pemrograman. Sehingga, dalam efisiensi dan kecepatan dan efisiensi
sistem tingkat rendah, Julia dapat bersaing dengan pemrograman seperti C atau
C++. Kemudian, dalam kemudahan penggunaan, Julia memiliki syntax yang mudah
dipahami seperti halnya bahasa pemrograman Python. Pada bagian ini akan
dijelaskan secara singkat mengenai sintaks - sintaks dasar Julia beserta
perbandingan nya dengan Bahasa Python. Penulis menggunakan
\cite{dokumentasijuliaJuliaDocumentationJulia2024} untuk menulis sintaks Julia,
dan menggunakan buku dari \cite{matthesPythonCrashCourse2016} untuk menulis
sintaks Python.

\subsection{Variable dan Tipe Data}

Variabel dalam Julia dibuat dengan cara menetapkan nilai pada suatu nama
menggunakan \emph{assignment operator} "\cw{=}". Sintaks ini cukup identik
dengan bahasa Python. Berikut merupakan contoh penulisan variable Julia dan
Python

\begin{lstlisting}
# Julia
my_name = "Alfarizi"
my_favourite_number = 42
my_favourite_pie = 3.1415

# Python
my_name = "Alfarizi"
my_favourite_number = 42
my_favourite_pie = 3.1415
\end{lstlisting}

\noindent
Sama seperti Python, Julia merupakan bahasa yang bertipe dinamis, yang berarti tidak memerlukan deklarasi
tipe variabel secara eksplisit. Tipe dari variabel secara otomatis diperoleh
berdasarkan tipe datau \emph{value} nya. Selain itu, variabel dalam Julia dapat mengubah
tipenya selama runtime. Misalnya saja seperti kode berikut:

\begin{lstlisting}
x = 20    # x adalah integer
x = "abc" # x berubah menjadi string
\end{lstlisting}

\label{basic tipe data julia} Dalam menyimpan suatu data, Julia mendukung
beberapa tipe data dasar seperti pada tipe data bahasa pemrograman pada umumnya.
Berikut merupakan tipe data dasar yang dimiliki oleh Julia:

\begin{itemize}
	\item \textbf{Angka}: Ini termasuk beberapa subjenis seperti:
	      \begin{itemize}
		      \item Bilangan bulat (\cw{Int8}, \cw{Int16}, \cw{Int32}, \cw{Int64}, \cw{Int128})

		      \item Angka floating-point (\cw{Float16}, \cw{Float32}, \cw{Float64})

		      \item Angka kompleks (\cw{ComplexF32}, \cw{ComplexF64})
	      \end{itemize}

	\item \textbf{Boolean}: \cw{Bool} mewakili nilai Boolean \cw{true} dan \cw{false}.

	\item \textbf{Karakter} dan \textbf{String}:
	      \begin{itemize}
		      \item Karakter: \cw{Char} digunakan untuk karakter Unicode tunggal.

		      \item String: \cw{String} mewakili rangkaian karakter.
	      \end{itemize}

	\item \textbf{Array}: \cw{Array{T,N}} menandakan array multidimensi dari
	      elemen tipe T.

	\item \textbf{Tuple}: \cw{Tuple{T1,T2,...}} menunjukkan koleksi terurut dari
	      elemen dengan tipe berbeda T1, T2, dan lain-lain.

	\item \textbf{Dictionaries}: \cw{Dict{K,V}} mewakili kumpulan pasangan kunci-nilai
	      di mana K dan V adalah tipe kunci dan nilai.
\end{itemize}

Dalam penamaan, nama variabel dalam Julia dapat mencakup subset simbol Unicode,
yang memungkinkan berbagai macam kemungkinan penamaan, termasuk penggunaan
simbol seperti huruf Yunani. Misalnya, untuk mewakili variabel dengan huruf
Yunani dalam sebagian besar lingkungan pengembangan Julia, Anda dapat
menggunakan sintaks seperti LaTeX (misalnya, \cw{\alpha} untuk $\alpha$).
Sintaks ini juga mencakup penambahan subskrip, superskrip, dan lain-lain.

% \markdownInput{markdowns/variable-dan-tipe-data.md}

% Melakukan variable

% \begin{lstlisting}[caption={Deklarasi variable}]{language=julia}
% my_name = "Aurelio"
% my_favourite_number = 42
% my_favourite_pie = 3.1415
% \end{lstlisting}

\subsection{Deklarasi Fungsi}

Fungsi merupakan deretan perintah yang dijalankan secara berurutan. Sama
seperti pada Matematika, fungsi digunakan untuk menyederhanakan deretan
perintah yang kemungkinan nanti nya akan dipanggi lagi. Pada Julia, fungsi
dapat dideklarasikan dengan 3 cara. Deklarasi fungsi yang umumnya dilakukan
pada Julia adalah seperti berikut

\begin{lstlisting}[label={contoh deklarasi fungsi 1}]{language=Julia}
# Julia
function plus_two(x)
  return x + 2
end

# Python
def plus_two(x):
  return x + 2
\end{lstlisting}

\noindent
Jika dilakukan \emph{breakdown}, fungsi diatas bernama \cw{plus_two} yang mana mempunyai
1 parameter \cw{x} kemudian mengembalikan nilai \cw{x + 2}. Sehingga jika
dilakukan pemanggilan fungsi \cw{plus_two(10)}, maka fungsi tersebut akan
mengembalikan nilai \cw{12}.

Pendeklarasian fungsi cara kedua disebut dengan \emph{inline-functions}, yang
mana berarti suatu fungsi yang perintah nya bisa dituliskan dalam satu baris
kode. Pada Python, \emph{inline-function} ini mirip seperti
\emph{lambda-function}. Contoh penulisan fungsi nya seperti berikut

\begin{lstlisting}[label={contoh deklarasi fungsi 2}]
# Julia
plus_two(x) = x + 2

# Python
plus_two = lambda x: x + 2
\end{lstlisting}

\noindent
Fungsi diatas identik dengan contoh fungsi pertama, yang mana mempunyai nama \cw{plus_two},
mempunyai 1 parameter \cw{x}, dan mengembalikan nilai \cw{x + 2}. Jika dilakukan
pemanggilan fungsi \cw{plus_two(20)} akan mengembalikan nilai \cw{22}.

Deklarasi fungsi ketiga disebut dengan \emph{anonymous-functions}. Pada Python,
deklarasi fungsi ini juga identik dengan \emph{lambda-function}. Contoh
penulisan fungsinya adalah seperti berikut mempunyai penulisan seperti berikut:

\begin{lstlisting}
# Julia
plus_two = x -> x + 2

# Python
plus_two = lambda x: x + 2
\end{lstlisting}

\noindent
Fungsi diatas identik dengan contoh fungsi pertama dan kedua, yana mana
mempunyai nama \cw{plus_two}. Parameter dan hasil pengembalian dari deklarasi
ini dipisahkan dengan \cw{->} yang mana seperti simbol panah ke kanan. Deklarasi
\emph{anonymous-functions} ini biasa digunakan ketika suatu fungsi mempunyai parameter
yang memerlukan fungsi lain. Dalam bahasa C/C++, istilah fungsi yang memerlukan fungsi
lain ini biasanya disebut dengan \emph{predicate}, sedangkan dalam bahasa Javascript
disebut dengan \emph{callback function}.

\subsection{Control Flow}

Pada suatu kode, diperlukan suatu mekanisme tertentu yang dapat mengarahkan
kode tergantung pada kondisi yang ada. Nama mekanisme ini adalah \emph{control
	flow}. Dalam bahasa pemrograman umum nya, terdapat 2 macam \emph{control flow},
yakni \emph{decision} (keputusan), dan perulangan. Penulisan keputusan pada
Julia bisa dilakukan seperti berikut:

\begin{lstlisting}
# Julia
x = 1000
if x<1
    print("$x < 1")
elseif x < 3
    print("$x < 3")
elseif x < 100
    print("$x < 100")
else
    print("$x is really big!")
end

# Python
x = 1000
if x < 1:
    print(f"{x} < 1")
elif x < 3:
    print(f"{x} < 3")
elif x < 100:
    print(f"{x} < 100")
else:
    print(f"{x} is really big!")

# Output
>> "1000 is really big!"
\end{lstlisting}

\noindent
Terlihat pada \emph{listing} kode diatas bahwa penulisan keputusan dapat
dilakukan dengan awalan \cw{if}, kemudian berisikan \cw{elseif} atau \cw{else}, dan
diakhiri \cw{end}. Penulisan seperti ini pada dasarnya merupakan penulisan yang umum
digunakan di bahasa pemrograman seperti C/C++, Javascript, ataupun Python.

\emph{Control flow} lainnya adalah perulangan. Dalam Julia, terdapat 2 tipe
perulangan. Perulangan pertama disebut dengan \emph{while loop} yang mana
penulisannya adalah seperti berikut

\begin{lstlisting}
# Julia
i = 0
while(i < 5)
    print("$i ")
    i += 1
end

# Python
i = 0
while i < 5:
    print(f'${i} ')
    i += 1

# Output
>> 0 1 2 3 4
\end{lstlisting}

\noindent
Kemudian perulangan tipe kedua, disebut dengan \emph{for loop} yang mana
penulisannya adalah seperti berikut

\begin{lstlisting}
# Julia
for i in 1:5
    print("$i ")
end

# Python
for i in range(1, 5):
    print(f'${i} ')

# Output
>> 1 2 3 4
\end{lstlisting}

\noindent
Kedua tipe perulangan dapat digunakan dalam dua kasus yang berbeda. \emph{for
	loop} dapat digunakan untuk kasus dimana pengguna tahu kapan akan berhenti, misalnya
akan berhenti ketika nilai nya nilai nya lebih dari 5. Sedangkan \emph{while
	loop} dapat digunakan ketika pengguna tidak tahu kapan akan berhenti, misalnya saja
dalam program game umumnya digunakan \emph{while loop} ini untuk menetukan
apakah pengguna masih berada di dalam game atau sudah keluar dari game.

\subsection{Data Structures}

Pada pengaplikasian suatu bahasa pemrograman, suatu masalah umumnya merupakan
kumpulan dari beberapa variable dengan tipe data tertentu. Untuk mengatur
beberapa variable ini diperlukan struktur khusus yang disebut dengan \emph{Data
	Structures} atau Struktur Data. Didalam Julia, terdapat 4 \emph{data
	structures} yang umumnya digunakan, yakni \emph{arrays}, \emph{tuples},
\emph{named tuples}, dan \emph{dictionaries}.

\emph{Dictionaries} merupakan struktur data yang mempunyai \emph{keys} dan \emph{values}.
\emph{keys} dan \emph{values} ini dipisahkan oleh simbol "\cw{=>}". Contoh penulisan
dictionaries adalah seperti berikut

\begin{lstlisting}
# Julia
person1 = Dict(
  "name" => "Alfarizi",
  "phone" => 123456789,
  "shoe-size" => 43
)
person2 = Dict(
  "name" => "Intan",
  "phone" => 123456789,
  "shoe-size" => 38
)

# Python
person1 = {
  "name": "Alfarizi",
  "phone": 123456789,
  "shoe-size": 43
}
person2 = {
  "name": "Intan",
  "phone": 123456789,
  "shoe-size": 38
}
\end{lstlisting}

\noindent
Pengaksesan isi dari \emph{dictionaries} bisa dilakukan dengan penulisan seperti
\cw{person1["name"]} yang mana akan mengembalikan nilai \cw{"Alfarizi"}.

Struktur data berikutnya adalah \emph{tuples} yang mana mempunyai ukuran yang
tetap. tuple bisa dituliska seperti berikut ini

\begin{lstlisting}
# Julia
a = (1,2,3)
b = 1, 2, 3

# Python
a = (1, 2, 3)
b = (1, 2, 3)

# Output
>> length(a)
3
>> length(b)
3
\end{lstlisting}

\noindent
Untuk mengakses isi dari tuple, bisa menggunakan penomoran yang dimulai dari 1. Istilah
penomoran ini disebut juga dengan \emph{indexing}. Misalnya, untuk mengakses
index pertama bisa dituliskan dengan \cw{a[1]} atau \cw{a[begin]}. Untuk mengakses
index terakhir bisa menggunakan \cw{a[end]}. Untuk mengakses index kedua bisa menggunakan
\cw{a[2]}.

Struktur data berikutnya adalah \emph{Named tuples} yang mana perpaduan antara
\emph{Tuples} dan \emph{Dictionaries}. Pada Python, tidak mendukung \emph{named
	tuple} secara \emph{built-in}, sehingga diperlukan \emph{import module} yang
bernama \emph{collection}. \emph{Named tuples} dapat dituliskan seperti berikut

\begin{lstlisting}
# Julia
namedTuple1 = (a = 1, b = "hello")

# Python
from collections import namedtuple
NamedTuple = namedtuple('NamedTuple', ['a', 'b'])
namedTuple1 = NamedTuple(a=1, b=2)
\end{lstlisting}

\noindent
Untuk mengakses isi dari \emph{named tuple}, dapat menggunakan penulisan \cw{namedTuple[:a]}
yang mana akan mengembalikan nilai \cw{1}. Nama dari tuple tidak boleh
menggunakan nama yang mengandung koma. Ukuran struktur data dari \emph{named
	tuple} akan selalu tetap dan tidak bisa bertambah banyak. Jika dilihat perbandingannya
dengan bahasa Python, implementasi \emph{named tuple} tampak cukup kompleks di Python.
Ini dikarenakan Python tidak mendukung \emph{built-in} \emph{named tuple}.

Struktur data terakhir adalah \emph{Arrays}. Array ini merupakan struktur data
yang paling banyak digunakan, dikarenakan array ini dapat membentuk vektor,
matrik, maupun tensor. Array yang mempunyai 1 dimensi disebut dengan
\emph{vektor}. Vektor dapat dituliskan seperti berikut

\begin{lstlisting}
# Julia
a = [1,2,3,4,5]
b = ["hello", "world", "julia"]

# Python
a = [1,2,3,4,5]
b = ["hello", "world", "python"]
\end{lstlisting}

\noindent
Untuk mengakses vektor, bisa menggunakan indexing seperti hal nya tuple.
Misalnya, dari contoh diatas untuk mengakses index kedua dari vektor \cw{b}
dapat dilakukan dengan \cw{b[2]} yang mana akan mengembalikan nilai \cw{"world"}.

Array yang berisikan 2 dimensi disebut dengan \emph{matriks}. Atau bisa juga
dikatakan bahwa Matrik merupakan kumpulan dari beberapa vektor. Matrik dapat
dituliskan seperti berikut

\begin{lstlisting}
# Julia
mat1 = [1 2 3; 4 5 6]

# Python
import numpy as np
mat1 = np.array([[1, 2, 3], [4, 5, 6]])
\end{lstlisting}

\noindent
Pada matrik terdapat kolom dan baris. Isi dari baris dipisahkan dengan titik
koma "\cw{;}", sedangkan isi dari kolom dipisahkan dengan spasi. Sehingga, jika
contoh matrik diatas diubah ke notasi matematika, maka hasilnya adalah sebagai berikut

\begin{equation}
	\text{mat1}\ = \left(
	\begin{matrix}
			1 & 2 & 3 \\
			4 & 5 & 6
		\end{matrix}
	\right)
\end{equation}

\noindent
Untuk melakukan akses ke elemen dari matrik, bisa menggunakan notasi \cw{mat1[row, col]}.
Misalnya jika ingin mengakses elemen pada baris pertama, kolom kedua, maka dapat
dituliskan dengan \cw{mat1[1, 2]} yang mana akan mengembalikan nilai \cw{2}.

\section{Fisika Komputasi dalam Operasi Matriks}

Fisika komputasi merupakan cabang interdisipliner yang memanfaatkan teknik
komputasi untuk memecahkan masalah fisika yang kompleks dan sulit diselesaikan
dengan metode analitis. Dengan perkembangan teknologi komputer yang semakin
canggih, fisika komputasi telah menjadi alat penting dalam penelitian dan
pengembangan di berbagai bidang, mulai dari astrofisika hingga ilmu material.

Dalam fisika komputasi, matriks memainkan peran krusial dalam pemodelan dan
analisis. Teori matriks, yang merupakan bagian dari matematika terapan,
menyediakan kerangka kerja untuk menggambarkan dan menyelesaikan sistem
persamaan linear yang sering muncul dalam perhitungan fisika. Misalnya, dalam
mekanika kuantum, matriks digunakan untuk merumuskan dan memecahkan persamaan
Schrödinger, sementara dalam dinamika fluida, matriks digunakan untuk
menghitung perilaku aliran fluida.

Pada GPU, semua proses yang diproses akan dipetakan operasi penjumlahan seperti
pada bagian \ref{unit komputasi}. Proses operasi penjumlahan merupakan proses
operasi dasar yang mana proses operasi lainnya bisa diubah ke operasi
penjumlahan ini. Misalnya saja, jika operasi pengurangan diubah ke bentuk
operasi penjumlahan, maka akan menghasilkan bentuk seperti berikut

\[
	A - B = A + (-B)
\]

\noindent
dimana $A$ dan $B$ merupakan suatu skalar, vektor, ataupun matriks. Dan untuk operasi
perkalian, jika diubah ke bentuk operasi penjumlahan, akan menghasilkan bentuk seperti
berikut

\[
	A \times n = \underbrace{A + A + A + A + \cdots}_{\text{berjumlah}\, n}
\]

\noindent
dimana $A$ adalah suatu skalar, vektor, ataupun matriks, dan $n$ adalah suatu
skalar. Berdasarkan kedua pembuktian diatas, maka dapat diambil kesimpulan bahwa
operasi pengurangan dan operasi perkalian dapat dilakukan di GPU. Sehingga,
operasi lain yang dapat dipetakan ke bentuk penjumlahan, pengurangan, ataupun
perkalian juga akan bisa diproses oleh GPU.

\subsection{Operasi Penjumlahan}

Operasi penjumlahan matriks dapat dituliskan sebagai berikut

%   \label{eq:matrix_sum}
%   \begin{bmatrix}
%     a_{11} & a_{12} & \cdots & a_{1n} \\ a_{21}&a_{22}&\cdots&a_{2n}\\ \vdots&\vdots&\ddots&\vdots \\ a_{m1}&a_{m2}&\cdots&a_{mn}
%   \end{bmatrix}
%   +
%   \begin{bmatrix}
%     b_{11} & b_{12} & \cdots & b_{1p} \\ b_{21}&b_{22}&\cdots&b_{2p}\\ \vdots&\vdots&\ddots&\vdots \\ b_{n1}&b_{n2}&\cdots&b_{np}
%   \end{bmatrix}
%   =
%   % Enter here
%   \begin{bmatrix}
%     a_{11} + b_{11} & a_{12} + b_{12} & \cdots & a_{1n} + b_{1n} \\ a_{21} + b_{21} & a_{22} + b_{22} & \cdots & a_{2n} + b_{2n}\\ \vdots&\vdots&\ddots&\vdots \\ a_{m1} + b_{m1} & a_{m2} + b_{m2}&\cdots& a_{mn} + a_{mn}
%   \end{bmatrix}
\begin{align} \label{eq:matrix_sum}
	\begin{bmatrix}
		a_{11} & a_{12} & \cdots & a_{1n} \\
		a_{21} & a_{22} & \cdots & a_{2n} \\
		\vdots & \vdots & \ddots & \vdots \\
		a_{m1} & a_{m2} & \cdots & a_{mn}
	\end{bmatrix}
	 & +
	\begin{bmatrix}
		b_{11} & b_{12} & \cdots & b_{1p} \\
		b_{21} & b_{22} & \cdots & b_{2p} \\
		\vdots & \vdots & \ddots & \vdots \\
		b_{n1} & b_{n2} & \cdots & b_{np}
	\end{bmatrix}
	\nonumber \\
	 & =
	\begin{bmatrix}
		a_{11} + b_{11} & a_{12} + b_{12} & \cdots & a_{1n} + b_{1n} \\
		a_{21} + b_{21} & a_{22} + b_{22} & \cdots & a_{2n} + b_{2n} \\
		\vdots          & \vdots          & \ddots & \vdots          \\
		a_{m1} + b_{m1} & a_{m2} + b_{m2} & \cdots & a_{mn} + b_{mn}
	\end{bmatrix}
\end{align}

\noindent
dimana $a_{11}$ hingga $a_{mn}$ dan $b_{11}$ hingga $b_{mn}$ merupakan elemen - elemen dari matriks. Suatu penjumlahan
dapat dikatakan operasi penjumlahan matriks jika memenuhi sifat - sifat operasi penjumlahan
matriks:

\begin{itemize}
	\label{property_additional_matrix}

	\item \emph{Komuntatif}: $A + B = B + A$

	\item \emph{Asosiatif}: $(A + B) + C = A + (B + C) = A + B + C$

	\item \emph{Pertambahan dengan matriks nol}: $A + 0 = 0 + A = A$

	\item \emph{Penjumlahan transpose}: $(A + B)^{T}= A^{T}+ B^{T}$
\end{itemize}

\noindent
dimana $A$, $B$, $C$ adalah matriks, dan $0$ adalah matriks nol.

\subsection{Operasi Pengurangan}
\label{Operasi Pengurangan}

Operasi pengurangan matriks dapat diperoleh dari operasi penjumlahan matriks
dengan penulisan sebagai berikut

\begin{equation}
	\label{eq:derivatif_additional_operand}A + (-B) = C\\ \Leftrightarrow A - B = C
\end{equation}

\noindent
dimana $A$, $-B$, dan $C$ adalah suatu matriks. Dari pernyataan persamaan \ref{eq:derivatif_additional_operand}
tersebut, maka bentuk operasi pengurangan matriks dapat dituliskan

\begin{align} \label{eq:matrix_substraction}
	\begin{bmatrix}
		a_{11} & a_{12} & \cdots & a_{1n} \\
		a_{21} & a_{22} & \cdots & a_{2n} \\
		\vdots & \vdots & \ddots & \vdots \\
		a_{m1} & a_{m2} & \cdots & a_{mn}
	\end{bmatrix}
	 & -
	\begin{bmatrix}
		b_{11} & b_{12} & \cdots & b_{1p} \\
		b_{21} & b_{22} & \cdots & b_{2p} \\
		\vdots & \vdots & \ddots & \vdots \\
		b_{n1} & b_{n2} & \cdots & b_{np}
	\end{bmatrix}
	\nonumber \\
	 & =
	\begin{bmatrix}
		a_{11} - b_{11} & a_{12} - b_{12} & \cdots & a_{1n} - b_{1n} \\
		a_{21} - b_{21} & a_{22} - b_{22} & \cdots & a_{2n} - b_{2n} \\
		\vdots          & \vdots          & \ddots & \vdots          \\
		a_{m1} - b_{m1} & a_{m2} - b_{m2} & \cdots & a_{mn} - b_{mn}
	\end{bmatrix}
\end{align}

\noindent
dimana $a_{11}$ hingga $a_{mn}$ dan $b_{11}$ hingga $b_{mn}$ merupakan elemen - elemen dari matriks. Operasi
pengurangan matriks mempunyai sifat - sifat yang hampir sama seperti operasi
penjumlahan matriks. Berikut merupakan sifat - sifat operasi pengurangan matriks:

\begin{itemize}
	\label{property_substraction_matrix}

	\item \emph{Komuntatif}: $A - B = B - A$

	\item \emph{Asosiatif}: $(A - B) - C = A - (B - C) = A - B - C$

	\item \emph{Pertambahan dengan matriks nol}: $A - 0 = 0 - A = A$

	\item \emph{Penjumlahan transpose}: $(A - B)^{T}= A^{T}- B^{T}$
\end{itemize}

\noindent
dimana $A$, $B$, $C$ adalah matriks, dan $0$ adalah matriks nol.

\subsection{Operasi Perkalian dengan Skalar}
\label{Operasi Perkalian dengan Skalar}

Operasi perkalian matriks dengan skalar dapat dituliskan

\begin{equation}
	\label{eq:matrix_mult_scalar}\gamma \times
	\begin{bmatrix}
		a_{11} & a_{12} & \cdots & a_{1n} \\ a_{21}&a_{22}&\cdots&a_{2n}\\ \vdots&\vdots&\ddots&\vdots \\ a_{m1}&a_{m2}&\cdots&a_{mn}
	\end{bmatrix}
	=
	\begin{bmatrix}
		\gamma a_{11} & \gamma a_{12} & \cdots & \gamma a_{1n} \\ \gamma a_{21} & \gamma a_{22} & \cdots & \gamma a_{2n}\\ \vdots & \vdots & \ddots & \vdots \\ \gamma a_{m1} & \gamma a_{m2} & \cdots & \gamma a_{mn}
	\end{bmatrix}
\end{equation}

\noindent
dimana $a$, $b$, $c$, $d$, dan $\gamma$ adalah suatu skalar. Operasi perkalian matriks
dengan skalar mempunyai sifat - sifat:

\begin{itemize}
	\item \emph{Asosiatif}: $(cd) A = c (dA)$

	\item \emph{Distributif}: $c(A+B)=cA+cB$ dan $(c+d)A=cA+dA$

	\item \emph{Identitas multiplikasi}: $1 A = A$

	\item \emph{Multiplikasi nol}: $0\cdot A=O$ dan $c\cdot O= O$

	\item \emph{Penutup perkalian}: $cA$ adalah matriks dengan dimensi yang sama dengan
	      $A$
\end{itemize}

\noindent
dimana $c$, $d$ adalah skalar, $A$, $B$ adalah matriks, dan $O$ adalah matriks
nol.

\subsection{Operasi Perkalian Matriks dengan Matriks}
\label{Operasi Perkalian Matriks dengan Matriks}

Bentuk umum operasi perkalian matriks dengan matriks dapat dituliskan

\begin{equation}
	\label{eq:matrix_mult_matrix}C = AB \quad \text{dimana}\quad c_{ij}= \sum_{k=1}
	^{n}a_{ik}b_{kj}
\end{equation}

\text{dengan:}
\begin{align*}
	A & = \begin{bmatrix}a_{11}&a_{12}&\cdots&a_{1n}\\ a_{21}&a_{22}&\cdots&a_{2n}\\ \vdots&\vdots&\ddots&\vdots \\ a_{m1}&a_{m2}&\cdots&a_{mn}\end{bmatrix}, \\
	B & = \begin{bmatrix}b_{11}&b_{12}&\cdots&b_{1p}\\ b_{21}&b_{22}&\cdots&b_{2p}\\ \vdots&\vdots&\ddots&\vdots \\ b_{n1}&b_{n2}&\cdots&b_{np}\end{bmatrix}, \\
	C & = \begin{bmatrix}c_{11}&c_{12}&\cdots&c_{1p}\\ c_{21}&c_{22}&\cdots&c_{2p}\\ \vdots&\vdots&\ddots&\vdots \\ c_{m1}&c_{m2}&\cdots&c_{mp}\end{bmatrix}.
\end{align*}

\noindent
dimana $A$ merupakan matriks berukuran $m \times n$, $B$ merupakan matriks
berukuran $n \times p$, $C$ merupakan matriks berukuran $m \times p$, $c_{ij}$
merupakan elemen matriks $C$ pada baris $i$ kolom $j$, $a_{ik}$ merupakan elemen
matriks $A$ pada baris $i$ kolom $k$, dan $b_{kj}$ merupakan elemen matriks $B$
pada baris $k$ kolom $j$.

Berikut ini merupakan sifat - sifat yang dapat memastikan bahwa suatu operasi
perkalian merupakan operasi perkalian matriks dengan matriks. Jika diasumsikan
$A$, $B$, dan $C$ adalah suatu matriks, sedangkan $\gamma$ adalah sautu skalar,
maka sifat-sifat dari operasi perkalian matriks dengan matriks adalah
\begin{itemize}
	\item \emph{Asosiatif}: $(AB)C = A(BC)$. Oleh karena itu kita dapat menulis
	      produk tersebut secara sederhana sebagai $ABC$.

	\item \emph{Asosiatif dengan perkalian skalar}: $\gamma(AB) = (\gamma A)B$, di
	      mana $\gamma$ adalah skalar, dan $A$ dan $B$ adalah matriks (yang dapat
	      dikalikan). Produk $(\gamma A)B$ juga bisa dituliskan $A(\gamma B)$. (Perhatikan
	      bahwa produk $\gamma A$ dan $\gamma B$ didefinisikan sebagai produk matriks-skalar,
	      tetapi secara umum, kecuali $A$ dan $B$ memiliki satu baris, tidak sebagai produk
	      matriks-matriks.)

	\item \emph{Distributif dengan penjumlahan}: $A(B + C) = AB + AC$ dan $(A + B)C
		      = AC + BC$. Pada sisi kanan dari persamaan ini kita menggunakan preseden
	      yang lebih tinggi dari perkalian matriks dibandingkan penjumlahan, jadi,
	      misalnya, $AC + BC$ diinterpretasikan sebagai $(AC) + (BC)$.

	\item Transpose dari produk. Transpose dari suatu produk adalah produk dari
	      transpose-transpose, tetapi dalam urutan yang berlawanan: $(AB)^{T}=
		      B^{T}A^{T}$.
\end{itemize}

\subsection{Operasi Inverse}
\label{Operasi Inverse}

Berdasarkan \cite{strangIntroductionLinearAlgebra2023}, suatu matriks $A$
berukuran $n \times n$ jika dikatakan mempunyai matriks inverse $A^{-1}$, maka
akan memenuhi definisi berikut

\begin{equation}
	\label{eq:inverse_matrix}A A^{-1}= A^{-1}A = I
\end{equation}

\noindent
dimana $I$ adalah matriks identitas yang berukuran $n \times n$. Suatu matriks
dapat mempunyai inverse matriks jika memenuhi:

\begin{enumerate}
	\item Merupakan matriks persegi (mempunyai banyak baris yang sama dengan banyak
	      kolom).

	\item Mempunyai nilai determinant yang bukan nol.
\end{enumerate}

Untuk matriks $2 \times 2$, cara mencari nilai dari inverse nya dapat
menggunakan persamaan

\begin{equation}
	\begin{bmatrix}
		a & b \\
		c & d
	\end{bmatrix}^{-1}= \frac{1}{ad - bc}
	\begin{bmatrix}
		d  & -b \\
		-c & a
	\end{bmatrix}
\end{equation}

\noindent
Nilai $ad - bc$ ini merupakan nilai determinant dari matriks 2x2, sehingga berdasarkan
syarat matriks untuk mempunyai inverse matriks, maka nilai $ad - bc \neq 0$. Pada
kasus matriks yang mempunyai ukuran lebih banyak, umum nya terdapat beberapa
metode pencarian inverse matriks yang bisa digunakan seperti metode matriks
adjoint dan metode eliminasi gauss-jordan. Namun, perhitungan matriks secara
manual umumnya memerlukan ketelitian yang cukup tinggi, sehingga sangat umum mencari
inverse matriks dengan menggunakan bantuan komputer.

\subsection{Nilai Eigen}
\label{Nilai Eigen}

Berdasarkan \cite{thomasscofieldEigenvaluesEigenvectors2018}, jika terdapat
matriks $\mathbf{A}$ berukuran $n \times n$, dan terdapat vektor $\vec{x}$ yang
berukuran $n$, maka dapat diperoleh persamaan

\begin{equation}
	\label{eq:eigen}\mathbf{A}\vec{x}= \lambda \vec{x}
\end{equation}

\noindent
dimana $\lambda$ adalah nilai eigen. Jika diasumsikan $\lambda$ berupa skalar, maka
persamaan \ref{eq:eigen} dapat dilanjutkan menjadi

\begin{gather*}
	\begin{array}{ c c }
		                & \mathbf{A}\vec{x} = \lambda \vec{x}                       \\
		\Leftrightarrow & \mathbf{A} \vec{x} -\lambda \vec{x} = 0                   \\
		\Leftrightarrow & \mathbf{A} \vec{x} - \mathbf{I} \lambda \vec{x} = 0       \\
		\Leftrightarrow & ( \mathbf{A} \vec{x} - \mathbf{I} \lambda ) \ \vec{x} = 0
	\end{array}\\
\end{gather*}

\noindent
dimana $\mathbf{I}$ adalah matriks identitas yang berukuran $n \times n$. Jika
$\vec{x}$ tidak boleh vektor nol, dan $( \mathbf{A}\vec{x}- \mathbf{I}\lambda )$
merupakan matriks $n \times n$, maka solusi dari nilai $\lambda$ yang skalar
harus memenuhi

\begin{equation}
	\label{eq:eigenvalue}\det{(\mathbf{A} - \mathbf{I} \lambda)}= 0
\end{equation}

\noindent
Jika disimpulkan, persamaan \ref{eq:eigen} mempunyai solusi yang tidak nol terhadap
vektor $\vec{x}$ jika dan hanya jika memenuhi persamaan \ref{eq:eigenvalue}.

Dalam bidang fisika, persamaan \ref{eq:eigen} digunakan dalam persamaan
schrödinger. Persamaan schrödinger satu dimensi tak gayut waktu dapat
dituliskan

\begin{equation} \label{eq:tise} % time independent schrodinger equation
	-\frac{\hbar^2}{2m} \frac{d^2\psi}{dx^2} = E\psi
\end{equation}

\noindent
Jika dihubungkan dengan persamaan \ref{eq:eigen}, maka terlihat bahwa $\psi$
merupakan vektor $\vec{x}$, kemudian $-\frac{\hbar^2}{2m} \frac{d^2}{dx^2}$ merupakan
matriks (atau operator) $\mathbf{A}$, dan $E$ merupakan nilai eigen $\lambda$.

\begin{figure}[h]
	\centering
	\includegraphics[width=7cm]{images/infinite_square_well.png}
	\caption{Sumur tak hingga}
	\label{img:infinite_square_well}
\end{figure}

Pada aplikasi nya, berdasarkan
\cite{griffithsIntroductionQuantumMechanics2019}, persamaan \ref{eq:tise} dapat
digunakan untuk mencari penyelesaian permasalahan potensial sumur tak hingga,
seperti yang terlihat pada Gambar \ref{img:infinite_square_well}. Solusi dari
potensial sumur tak hingga, diperoleh nilai eigen ($E$) yang dapat dituliskan
sebagai berikut

\begin{equation} \label{eq:eigenvalue_infinite_square_well}
	E_n = \frac{\hbar^2 k_n^2}{2m} = \frac{n^2 \pi^2 \hbar^2}{2ma^2}
\end{equation}

dengan $n = 1, 2, 3, 4, \dots$

\begin{figure}[H]
	\centering
	\includegraphics[width=7cm]{images/solution_infinte_square_well.png}
	\caption{Di sebelah kiri adalah fungsi gelombang, di sebelah kanan adalah representasi probabilitas untuk menemukan partikel pada posisi tertentu untuk berbagai keadaan kuantum}
	\label{img:solution_infinite_square_well}
\end{figure}

\noindent
Solusi dari persamaan \ref{eq:eigenvalue_infinite_square_well} merupakan bentuk dari
persamaan sinusoidal. Berdasarkan \cite{dalessandrisSpiralModernPhysics2024},
bentuk grafik nya dapat digambarkan seperti Gambar \ref{img:solution_infinite_square_well}.

Aplikasi penggunaan pencarian nilai eigen juga dapat ditemukan ketika mencari nilai eigen dari persamaan schrodinger melalui pendekatan metode beda hingga. Bentuk turunan kedua dari pendekatan metode beda hingga pada titik $x_i$ dapat dituliskan

\begin{equation} \label{eq:finite_difference_method}
	\frac{d^2 f}{dx^2} \bigg|_{x=x_i} \approx \frac{f_{i-1} - 2f_i + f_{i+1}}{(\Delta x)^2}
\end{equation}

\noindent
dengan \( f_i = f(x_i) \) dan \( \Delta x = x_{i+1} - x_i \). Kemudian bentuk umum persamaan Schrodinger pada suatu sistem dengan potensial $V(x)$ dapat dituliskan

\begin{equation} \label{eq:schrodinger_eq}
	-\frac{\hbar^2}{2m} \frac{d^2 \psi(x)}{dx^2} + V(x) \psi(x) = E \psi(x)
\end{equation}

Jika Persamaan \ref{eq:finite_difference_method} disubstitusikan ke dalam Persamaan \ref{eq:schrodinger_eq}, maka akan diperoleh persamaan

\begin{equation*}
	-\frac{\hbar^2}{2m} \left( \frac{\psi_{i-1} - 2\psi_i + \psi_{i+1}}{(\Delta x)^2} \right) + V_i \psi_i = E \psi_i
\end{equation*}

\noindent
dengan \( V_i = V(x_i) \) dan \( \psi_i = \psi(x_i) \). Persamaan tersebut juga dapat disajikan menjadi

\begin{equation} \label{eq:schrodinger_finite_method_1}
	-\psi_{i-1} + \left(2 + \frac{2m(\Delta x)^2}{\hbar^2} V_i - \frac{2m(\Delta x)^2}{\hbar^2} E\right) \psi_i - \psi_{i+1} = 0
\end{equation}

\noindent
Jika diperkenalkan besaran tak bersatuan berikut

\begin{equation} \label{eq:unitless_schrodinger_finite}
	b_i = \gamma^2 (\Delta x)^2 = \frac{2m(\Delta x)^2 V_i}{\hbar^2}, \quad \lambda = \epsilon^2 (\Delta x)^2 = \frac{2m(\Delta x)^2 E}{\hbar^2}
\end{equation}

\noindent
dimana $\epsilon$ mewakili energi sistem yang dinormalisasi, sedangkan $\gamma_i$ mewakili potensial sistem pada titik tertentu. Kemudian jika besaran tersebut dimasukkan ke Persamaan \ref{eq:schrodinger_finite_method_1}, maka akan diperoleh persamaan

\begin{equation}\label{eq:schrodinger_finite_method_2}
	-\psi_{i-1} + (2 + b_i - \lambda) \psi_i - \psi_{i}
\end{equation}

Jika misalnya terdapat partikel bermassa \( m \) yang terjebak dalam potensial berbentuk sumur dengan kedalaman tak berhingga dan dibatasi oleh dinding sumur di \( x = -1 \) dan \( x = +1 \). Anggap bahwa \( x_0 = -1 \), \( x_N = +1 \) dan daerah di dalam sumur dibagi menjadi \( N \) bagian sedemikian hingga

\[
	\Delta x = \frac{2}{N}; \quad x_j = x_0 + j\Delta x; \quad j = 1, 2, 3, \ldots, N
\]

\noindent
Berdasarkan hasil pada penyelesaian potensial sumur tak berhingga maka diketahui bahwa \( \psi_0 = 0 \), \( \psi_N = 0 \) dan \( b_i = 0 \). Dengan demikian maka persamaan beda hingga yang mewakili persamaan Schrodinger \ref{eq:schrodinger_finite_method_2} dapat dituliskan dalam bentuk matriks

\begin{equation} \label{eq:schrodinger_finite_matrix}
	\begin{bmatrix}
		2 - \lambda & -1          & \cdots & 0           \\
		-1          & 2 - \lambda & \cdots & 0           \\
		\vdots      & \vdots      & \ddots & \vdots      \\
		0           & 0           & \cdots & 2 - \lambda \\
	\end{bmatrix}
	\begin{bmatrix}
		\psi_1     \\
		\psi_2     \\
		\vdots     \\
		\psi_{N-1} \\
	\end{bmatrix}
	= \begin{bmatrix}
		0      \\
		0      \\
		\vdots \\
		0      \\
	\end{bmatrix}
\end{equation}

\noindent
Persamaan diatas sudah dalam bentuk matriks, sehingga akan lebih mudah untuk dimasukkan ke model komputasi agar diperoleh nilai eigen nya. Berdasarkan besaran tak bersatuan \ref{eq:unitless_schrodinger_finite}, dapat diperoleh kaitan seperti berikut

\begin{equation}
	\epsilon = \frac{1}{\Delta x} \sqrt{\lambda}, \quad \gamma_i = \frac{1}{\Delta x} \sqrt{b_i}
\end{equation}

\noindent
Pada persamaan \ref{eq:schrodinger_finite_matrix}, nilai $b_i$ dianggap nol. Namun, jika nilai $b_i$ ini tidak dianggap nol, maka berdasarkan kaitan diatas, dapat diperoleh nilai $b_i$ dari nilai $gamma_i$ nya.

Untuk melakukan validasi hasil antara numerik dan analitik dari persamaan schrodinger pada sumur tak hingga, nilai eigen yang diperoleh dari Persamaan \ref{eq:schrodinger_finite_matrix}, dapat dimasukkan ke persamaan berikut yang mana merupakan penyusunan ulang dari Persamaan \ref{eq:unitless_schrodinger_finite}

\begin{equation} \label{eq:numerical_energy}
	E = \frac{\hbar^2 \lambda}{2m (\Delta x)^2}
\end{equation}

\noindent
Sedangkan besar energi secara analitis, berdasarkan \cite{griffithsIntroductionQuantumMechanics2019}, dapat diperoleh dengan persamaan

\begin{equation} \label{eq:analytical_energy}
	E = \frac{n^2 \pi^2 \hbar^2}{2m L^2}.
\end{equation}

% \section{Fisika Komputasi}

% Fisika Komputasi adalah cabang ilmu pengetahuan yang memadukan prinsip-prinsip
% fisika dengan teknik komputasi untuk menyelesaikan masalah yang kompleks dan
% sulit dipecahkan melalui eksperimen atau teori konvensional. Dalam prakteknya,
% Fisika Komputasi memanfaatkan algoritma komputer untuk mensimulasikan sistem
% fisika, dari skala atomik hingga skala astronomis, dan untuk mengolah data
% eksperimental yang besar[1].

% Bidang ini mengemuka seiring dengan kemajuan teknologi komputer, yang
% memungkinkan ilmuwan untuk mengatasi keterbatasan analitis dalam fisika
% tradisional. Dengan menggunakan model dan simulasi komputer, Fisika Komputasi
% mampu memberikan wawasan baru tentang fenomena fisika yang sebelumnya sulit atau
% bahkan tidak mungkin diamati[2].

% Ruang lingkup Fisika Komputasi sangat luas dan mencakup berbagai sub-bidang dalam
% fisika. Beberapa di antaranya adalah:

% \begin{enumerate}
%   \item Mekanika Statistik dan Termodinamika: Penggunaan metode komputasi untuk
%     memahami perilaku sistem dengan banyak partikel, seperti gas, cairan, dan material
%     padat[3].

%   \item Mekanika Kuantum: Simulasi sistem pada skala atomik dan subatomik untuk
%     memahami perilaku partikel seperti elektron dan foton[1].

%   \item Astrofisika: Penggunaan model komputasi untuk mempelajari proses-proses
%     yang terjadi dalam bintang, galaksi, dan alam semesta[2].

%   \item Fisika Material: Pemodelan sifat-sifat material untuk memahami dan
%     merancang material baru dengan karakteristik khusus[2].

%   \item Fisika Partikel: Simulasi interaksi partikel subatomik untuk
%     mengeksplorasi aspek-aspek fundamental dari materi dan energi[1].

%   \item Dinamika Fluida Komputasi (CFD): Penggunaan teknik komputasi untuk
%     mengkaji aliran fluida dan fenomena terkait[3].

%   \item Biofisika dan Sistem Kompleks: Menerapkan metode fisika komputasi untuk
%     memahami sistem biologis dan perilaku kolektif dalam sistem yang kompleks[2].
% \end{enumerate}

% Bidang ini terus berkembang, mengikuti perkembangan teknologi komputasi dan
% algoritma. Peran penting Fisika Komputasi dalam kemampuannya untuk menguji teori,
% mencari parameter yang tidak dapat diakses oleh eksperimen, dan memberikan prediksi
% yang dapat diuji melalui eksperimen atau observasi.

% \subsection{Pengantar Metode Numerik}

% [Isi nya berdasarkan metode numerik yang digunakan]

% Metode numerik dalam Fisika Komputasi adalah teknik-teknik yang digunakan untuk mendekati
% solusi persamaan fisika yang sulit atau tidak mungkin diselesaikan secara analitis.
% Dalam prakteknya, metode numerik memungkinkan ilmuwan untuk mengubah persamaan fisika
% menjadi bentuk yang bisa dihitung dan diselesaikan dengan komputer. Ini menjadi
% sangat penting dalam menghadapi masalah fisika modern yang seringkali terlalu
% rumit untuk diselesaikan dengan metode matematika tradisional[1].

% % Metode Differensial Numerik

% Salah satu aplikasi metode numerik yang paling umum dalam fisika adalah dalam
% penyelesaian persamaan differensial, baik yang bersifat biasa maupun parsial.
% Persamaan ini sering muncul dalam banyak masalah fisika, dari dinamika fluida hingga
% fisika kuantum. Teknik seperti metode Euler, Runge-Kutta, dan metode finite
% difference digunakan untuk mendekati solusi dari persamaan-persamaan ini[2].

% \subsubsection{Metode Monte Carlo}

% Metode Monte Carlo dinamai dari kasino di Monaco, mencerminkan unsur "acak" dalam
% metode ini. Metode ini bergantung pada sampel acak untuk menghasilkan nilai
% numerik. Metode ini sangat berguna dalam kasus di mana solusi analitis sulit diperoleh
% atau bahkan tidak mempunyai solusi analitis, seperti pada perhitungan integral
% berdimensi tinggi atau dalam simulasi sistem yang memiliki banyak variabel
% stokastik \citep{kalosMonteCarloMethods2008}.

% Konsep dasar Metode Monte Carlo dapat diilustrasikan melalui perhitungan integral.
% Misalnya, kita ingin menghitung integral dari sebuah fungsi $f(x)$ dalam interval
% $[a, b]$. Nilai dari integral tersebut dapat didekati sebagai berikut:

% \begin{equation}
%   \int_{a}^{b}f(x) dx \approx \frac{b-a}{N}\sum_{i=1}^{N}f(x_{i})
% \end{equation}

% \noindent
% di mana $x_{i}$ adalah sampel acak dari interval $[a, b]$ dan $N$ adalah jumlah sampel
% yang diambil. Jika $N$ bernilai besar, maka nilai pendekatan nya akan mendekati nilai
% integral nya.

% \begin{figure}[H]
%   \centering
%   \includegraphics[width=10cm]{
%     images/visualization-of-the-sampled-points-by-the-Monte-Carlo-simulation-to-compute-pi.png
%   }
%   \caption{Contoh simulasi metode monte carlo pada pencarian luas lingkaran}
%   \label{gambar contoh pencarian luasan monte carlo}
% \end{figure}

% Sebagai contoh, metode ini dapat digunakan untuk menghitung area di bawah kurva dari
% seperempat lingkaran. Jika berdasarkan Gambar
% \ref{gambar contoh pencarian luasan monte carlo}, seperempat lingkaran nya kita tarik
% dari $x = 0$ hingga $x = 1$. Secara analitis, seperempat lingkaran tersebut mempunyai
% luasan $L = \frac{1}{4}\pi r^{2}$. Dalam metode monte carlo, nilai nya akan mendekati
% nilai analitis nya dengan cara menjumlahkan titik-titik yang masuk ke dalam seperempat
% lingkaran kemudian membaginya dengan total titik yang ada.

% % Aplikasi dalam Fisika

% Dalam fisika, Metode Monte Carlo digunakan dalam berbagai konteks, seperti dalam
% simulasi dinamika molekuler, fisika partikel, dan fisika statistik. Sebagai contoh,
% dalam dinamika molekuler, metode ini digunakan untuk memprediksi perilaku sistem
% partikel dengan mensimulasikan interaksi mereka selama periode waktu tertentu.

% % Variansi dan Pengurangan Kesalahan

% Salah satu aspek penting dalam Metode Monte Carlo adalah manajemen variansi. Karena
% metode ini bergantung pada sampel acak, hasilnya memiliki variansi yang terkait
% dengan jumlah sampel. Pengurangan kesalahan dapat dicapai dengan meningkatkan jumlah
% sampel atau menggunakan teknik seperti \emph{importance sampling} untuk memilih
% sampel yang lebih representatif \citep{casellaMonteCarloStatistical2004}.

% \subsubsection{Metode Elemen Hingga}

% \emph{Finite Element Method} (FEM) atau dalam bahasa indonesia disebut Metode Elemen
% Hingga (MEH) merupakan metode numerik yang digunakan untuk menyelesaikan permasalahan
% persamaan diferensial parsial. FEM melibatkan proses pemecahan suatu objek atau
% struktur menjadi serangkaian elemen yang lebih kecil. Setiap elemen ini dihubungkan
% melalui titik-titik yang dikenal dengan nama \emph{nodes}. Metode FEM memungkinkan
% penyelesaian masalah yang lebih kompleks dengan mengganti model kontinu dengan model
% diskrit \citep{etsworldsPengertianDanKonsep2018}.

% Berdasarkan \cite{ichrossofilmubarotPerancanganKonstruksiMesin2017}, matematika dasar
% dari FEM melibatkan beberapa langkah berikut:

% \begin{enumerate}
%   \item \textbf{Diskritisasi Struktur}: Struktur dibagi menjadi elemen-elemen hingga,
%     yang dapat berbentuk satu, dua, atau tiga dimensi tergantung kebutuhan.

%   \item \textbf{Pembuatan Persamaan Displacement}: Setiap elemen memiliki persamaan
%     \emph{displacement} yang didasarkan pada bentuknya.

%   \item \textbf{Hubungan \emph{Strain-Displacement} dan \emph{Stress-Strain}}: Hubungan
%     ini penting untuk menghasilkan persamaan pada setiap elemen. Hukum Hooke
%     sering digunakan untuk menghubungkan stress dengan strain.

%   \item \textbf{Matriks Kekakuan}: Persamaan matriks kekakuan dibuat menggunakan
%     metode seperti \emph{direct equilibrium}, \emph{energy method}, atau \emph{weight
%     residual method}.

%   \item \textbf{Persamaan Global dan Boundary Condition}: Persamaan matriks kekakuan
%     dari setiap elemen dikombinasikan untuk membentuk persamaan global. \emph{Boundary
%     condition} diterapkan untuk menyelesaikan permasalahan singular.

%   \item \textbf{Menyelesaikan Degree of Freedom yang Tidak Diketahui}: Ini melibatkan
%     penggunaan metode eliminasi atau iteratif seperti Gauss atau Gauss-Seidel untuk
%     menemukan nilai-nilai yang tidak diketahui.

%   \item \textbf{Analisis Strain dan Stress}: Terakhir, \emph{strain} dan \emph{stress}
%     pada setiap elemen dihitung untuk mendapatkan pemahaman tentang respons
%     struktur terhadap beban yang diterapkan.
% \end{enumerate}

% \section{Julia dalam Fisika Komputasi}

% [Implementasi Julia dalam Fisika Komputasi]

% % \section{Integrasi GPU dalam Fisika Komputasi}

% \section{Penggunaan Julia untuk Komputasi Fisika pada GPU}

% [Mungkin penjelasannya meliputi dasar dasar CUDA.jl dan library serupa]

% =======================================================================

% \section{Persamaan Diferensial Parsial}
% \subsection{Persamaan Diferensial}
% Suatu sistem yang cenderung berubah-ubah terhadap waktu akan lebih mudah untuk digambarkan perubahan yang terjadi ketimbang keadaan mutlak pada saat tertentu. Dalam berbagai bidang keilmuan, persamaan diferensial digunakan untuk menggambarkan perubahan yang terjadi pada suatu sistem. Persamaan diferensial mengandung fungsi yang tidak diketahui (yang merupakan solusi dari persamaan diferensial tersebut) dan beberapa turunannya. Namun, pencarian solusi yang eksplisit seringkali sulit didapat, sehingga dilakukan penyelesaian menggunakan grafik dan pendekatan numerik untuk mendapatkan informasi yang dibutuhkan \citep{stewart_2020_calculus}.  Apabila persamaan tersebut mengandung dua atau lebih variabel bebas, maka persamaan diferensial tersebut merupakan persamaan diferensial parsial (PDP) yang dihitung menggunakan derivatif parsial.
% \subsection{Derivatif Parsial}
% Derivatif parsial dalam sebuah fungsi dengan beberapa variabel adalah derivatif terhadap salah satu variabel, dengan menganggap variabel lainnya konstan \citep{robertalexanderadams_2014_calculus}. Dalam PDP dua variabel $f(x,y)$, misalnya, turunan (derivatif) terhadap $x$ didapat dengan menganggap persamaan tersebut adalah satu variabel dengan menganggap $y$ adalah tetap dan diperlakukan sebagai konstanta \citep{riley_2006_mathematical}. Secara formal, derifatif parsial dari persamaan $f(x,y)$ terhadap $x$ dapat dinyatakan sebagai berikut (dengan menganggap bahwa limitnya ada):
% \begin{equation} \label{partial_limit}
% \frac{\partial f}{\partial x} = \lim_{\Delta x\to\infty} \frac{f(x+\Delta x,y)-f(x,y)}{\Delta x}.
% \end{equation}
% \subsection{Persamaan Diferensial Parsial}
% Berikut akan disajikan sebuah contoh persamaan diferensial parsial:
% \begin{equation} \label{partial_orde_dua}
%     A \frac{\partial^2 u}{\partial x ^2} + B\frac{\partial^2 u}{\partial x \partial y} + C\frac{\partial^2u}{\partial y ^2} = f(x,y),
% \end{equation}
% dengan $x$ dan $y$ merupakan variabel bebas, dan $u$ merupakan solusi yang akan dicari, dan $A, B, C$ merupakan fungsi yang diketahui dari $x$ dan $y$.

% Secara umum, persamaan \eqref{partial_limit} dapat ditulis menjadi persamaan diferensial parsial yang lebih umum menjadi:
% \begin{equation}\label{partial_umum}
%     Lu = f,
% \end{equation}
% dengan $L$ merupakan operator diferensial, $u$ merupakan fungsi yang tidak diketahui (\emph{unknown}) atau solusi yang akan dicari, dan $f$ adalah fungsi yang diketahui. Apabila nilai $f$ pada persamaan \eqref{partial_umum} bernilai  $= 0$, maka persamaan tersebut disebut dengan persamaan diferensial parsial homogen. Sedangkan apabila nilai $f \neq 0$, maka persamaan tersebut disebut dengan persamaan diferensial parsial nonhomogen.

% \subsection{PDP Orde Dua}\label{PDP_orde_dua}
% Pada persamaan diferensial parsial, dikenal istilah orde yang merupakan turunan tertinggi dari fungsi yang ada pada PDP tersebut \citep{riley_2006_mathematical}. Dalam permasalahan fisika, jamak ditemukan pelibatan solusi dari PDP orde dua (yang bentuk umumnya ditampilkan pada Persamaan \eqref{partial_orde_dua})\citep{arfken_2013_mathematical}. Persamaan-persamaan diferensial parsial tersebut menggunakan operator diferensial berupa operator Laplacian: $\nabla ^ 2$. Salah satu contoh dari PDP orde dua yang cukup terkenal dan digunakan adalah persamaan Laplace:
% \begin{equation}\label{laplace_umum}
%     \nabla ^2 u = 0.
% \end{equation}

% Pada persamaan \eqref{laplace_umum} fungsi $u$ dapat berupa potensial gravitasi pada daerah yang tidak mengandung massa, potensial elektrostatis pada daerah tanpa muatan, temperatur keadaan tunak, atau potensial kecepatan untuk fluida mampat tanpa vortisitas dan tanpa sumber atau tenggelam \citep{riley_2006_mathematical}. Intinya, sisi kanan pada persamaan ini menggambarkan keadaan tanpa sumber atau gangguan. Solusi dari persamaan Laplace disebut fungsi harmonik \citep{stewart_2020_calculus}.

% Sejalan dengan persamaan Laplace, persamaan Poisson juga menggambarkan kuantitas fisis yang sama pada persamaan Laplace pada ruas kanannya, hanya saja pada daerah yang mengandung massa, arus listrik, atau sumber panas atau fluida \citep{boas_2006_mathematical}. Persamaan Poisson yang penyelesaiannya akan menjadi bahasan utama di penulisan ini akan dijelaskan pada subbab yang lain.

% Dalam PDP orde dua, persamaan-persamaan yang ada dapat diklasifikasikan menjadi beberapa jenis. Pengklasifikasian ini memberikan panduan mengenai syarat batas dan syarat awal yang tepat dan kehalusan dari solusi analitik. Sama seperti halnya pengklasifikasian irisan kerucut dan bentuk kuadrat ke dalam parabolik, hiperbolik, dan eliptik berdasarkan diskriminan, PDP orde dua juga dapat diklasifikasikan ke kelompok tersebut dengan nilai diskriminan $B^2 - AC$ pada bentuk persamaan \eqref{partial_orde_dua} \citep{yehudapinchover_2013_an}:
% \begin{enumerate}
%     \item $B^2 - 4AC < 0$: Persamaan diferensial parsial eliptik
%     \item $B^2 - 4AC = 0$: Persamaan diferensial parsial parabolik
%     \item $B^2 - 4AC > 0$: Persamaan diferensial parsial hiperbolik
% \end{enumerate}

% Persamaan Laplace dan persamaan Poisson merupakan jenis persamaan diferensial parsial eliptik dengan bentuk:
% \begin{equation}\label{pdp_eliptik}
%     \frac{\partial^2u}{\partial x^2} + \frac{\partial^2u}{\partial y^2}
% \end{equation}
% dengan $A = 1, B = 0, C = 1$, sehingga nilai diskriminannya adalah $0-4(1)(1) = -4 <0$

% \section{Persamaan Poisson}
% Persamaan Poisson adalah sebuah PDP yang menghubungkan antara persamaan Laplace persamaan \eqref{laplace_umum} dengan suku sumber tertentu, seperti yang sudah dibahas pada bagian \ref{PDP_orde_dua}.
% \begin{equation}
%     \nabla^2 u = \rho(\textbf{r})
% \end{equation}
% Fungsi $\rho(\textbf{r})$ inilah yang disebut sebagai suku sumber atau kepadatan sumber. Dalam aplikasi fisiknya, fungsi ini terdiri dari banyak konstanta fisika. Contohnya, jika $u$ adalah potensial elektrostatis pada sebuah ruang, maka $\rho$ adalah kepadatan muatan listrik, sehingga $\nabla^2 u = -\rho(\textbf{r})/\epsilon_{0}$, dengan $\epsilon_0$ adalah permitivitas ruang hampa. Dalam konteks lain, $u$ merupakan potensial gravitasi pada suatu ruang dengan kepadatan materinya diwakili oleh $\rho$; sehingga $\nabla^2 u = 4\pi G \rho (\textbf{r})$ \citep{riley_2006_mathematical}.

% \subsection{Penurunan Persamaan Poisson}
% Bentuk dari Hukum Gauss dapat ditampilkan dalam dua cara: hubungan antara medan listrik $\textbf{E}$ dan muatan listrik total, maupun dalam hubungan antara medan listrik perpindahan $\textbf{D}$ dengan muatan listrik bebas\citep{grant_1990_electromagnetism}. Hubungan antara medan listrik perpindahan $\textbf{D}$ dan muatan bebas digambarkan dengan hubungan sebagai berikut:
% \begin{equation}\label{medan_perpindaha}
%     \nabla \cdot \textbf{D} = \rho_{V}
% \end{equation}
% Seperti diketahui, bahwa $\textbf{D} = \epsilon \textbf{E}$. Sehingga persamaan \eqref{medan_perpindaha} dapat kita tuliskan kembali dengan:
% \begin{equation}\label{epsilon_dot_e}
%     \nabla \cdot (\epsilon \textbf{E}) = {\rho_V}
% \end{equation}
% Dengan asumsi medannya homogen pada ruang hampa, maka $\epsilon$ adalah konsntanta permitivitas ruang hampa:
% \begin{equation}\label{epsilonpindahruas}
%     \nabla \cdot \textbf{E} = \frac{\rho_V}{\epsilon}
% \end{equation}
% Hubungan antara medan listrik $E$ dan potensial listrik $V$ dapat digambarkan dengan persamaan \eqref{medan_listrik}:
% \begin{equation}\label{medan_listrik}
%     \textbf{E} = -\nabla V.
% \end{equation}
% Dalam persamaan tersebut dikatakan bahwa medan listrik merupakan gradien dari potensial skalar \citep{davidjeffreygriffiths_2018_introduction}. Maka persamaan \eqref{epsilonpindahruas} dapat kita tulis ulang menjadi:
% \begin{equation}
%     \nabla \cdot (\nabla V) = - \frac{\rho_V}{\epsilon}
% \end{equation}
% \begin{equation}\label{poisson umum}
%     \nabla ^2 V = - \frac{\rho_V}{\epsilon}
% \end{equation}
% Persamaan \eqref{poisson umum} merupakan persamaan Poisson.

% \section{Metode Numerik}
% \subsection{Motivasi}
% Dalam penyelesaian PDP, secara umum terdapat dua metode: metode analitikal dan metode numerik. Metode analitikal merupakan metode perhitungan langsung untuk mendapatkan perhitungan yang eksak. Namun, seringkali metode analitik tidak dapat menjangkau solusi dari PDP. Sehingga dibutuhkan pendekatan numerikal untuk mencari pendekatan numerik pada solusi PDP.

% Berbagai penyelesaian dengan metode numerik membutuhkan metode iteratif, bahkan kadang rekursif, sehingga dibutuhkan teknologi yang dapat melakukan metode numerik secara cepat. Kini perhitungan metode numerik dapat dilakukan secara komputasional lewat berbagai aplikasi maupun pemrograman. Fokus pada metode ini adalah memodelkan PDP secara numerik sehingga dapar didiskritisasi melalui simulasi komputer \citep{wick_2022_numerical}.

% Mengembangkan dan menganalisa algoritma untuk memecahkan masalah PDP dengan komputer adalah bagian dari metode numerik, yang merupakan bagian dari komputasi saintifik. Komputasi saintifik terbagi menjadi tiga bidang \citep{wick_2022_numerical}:
% \begin{enumerate}
%     \item Pemodelan matematika dan analisis objek
%     \item Pengembangan metode numerik dan algoritma yang efisien dan dapat diandalkan, serta analisanya
%     \item Pengembangan menggunakan perangkat lunak riset (implementasi dari algoritma yang ada)
% \end{enumerate}
% Yang menjadi tugas kunci (\emph{key task}) dari bidang tersebut adalah perancangan dan analisa algoritma. Tujuan utama dari algoritma adalah meformulasi sebuah skema sehingga dapat diimplementasikan ke komputer untuk menjalankan simulasi numerikal. Ada skema yang langsung yang memecahkan masalah hingga ke pembulatan kesalahannya (\emph{error}) seperti halnya eliminasi Gaussian, ada juga skema yang iteratif yang memberikan perkiraan pada solusi sampai pada akurasi tertentu, seperti halnya iterasi Richardson untuk memecahkan sistem persamaan linear. Algoritma dapat dianalisa pada akurasi, efisiensi, dan ketahanannya.

% Dalam perhitungan numerik PDP secara umum, akan ditemukan ralat (\emph{error}) yang merupakan selisih dari hasil pendekatan numerik dengan hasil yang eksak. Misal sebuah PDP diajukan dalam ruang fungsi dan dianalisa:
% \begin{itemize}
%     \item Ruang fungsi $V$ dari solusi eksak $u$ (mungkin tidak diketahui). Maka $u \in V$.
%     \item Ruang fungsi $\Tilde{V}$ dari solusi pendekatan $\Tilde{u}$. Maka $\Tilde{u} \in \Tilde{V}$
% \end{itemize}
% Diskretisasi numerik pada ruang dan waktu menghasilkan parameter $h$ dan $k$ untuk menghitung solusi pendekatan, yaitu $\Tilde{u}:=u_{hk}$. Kemudian dapat digambarkan diskretisasi ralat adalah
% \begin{equation}
%     \left \lvert u - u_{kh} \right \rvert \rightarrow  0 \quad untuk \quad  h \rightarrow 0, k \rightarrow 0
% \end{equation}
% Diskretisasi ralat ini adalah salah satu sumbanngan ralat pada hasil perhitungan PDP dengan metode numerik. Sumbangan ralat lainnya masih sangat mungkin akan terjadi sehingga harus dapat diterima bahwa kita tidak bisa menghindari ralat. Namun yang terpenting adalah mengontrol ralat dan menemukan pembahasan yang tepat apabila ralat yang dihasilkan terlalu besat sehingga memengaruhi interpretasi terhadap simulasi numerik, atau bahkan pembahasan yang tepat saat ralat dapat diabaikan.

% Dalam \cite{richter_2017_einfhrung} (via \cite{wick_2022_numerical}), disimpulkan bahwa ada tujuh aspek yang merupakan karakteristik pemodelan numerik:
% \begin{enumerate}
%     \item Aproksimasi
%     \item Konvergen
%     \item Orde kekonvergenan
%     \item Ralat
%     \item Estimasi ralat
%     \item Efisiensi
%     \item Stabilitas
% \end{enumerate}

% Selanjutnya, akan disebutkan beberapa metode numerik yang lazim digunakan pada penyelesaian PDP:
% \begin{enumerate}
%     \item Metode beda hingga
%     \item Metode garis
%     \item Metode elemen terbatas
%     \item Metode diskretisasi gradien
%     \item Metode volume hingga
%     \item Metode spektral
%     \item Metode tanpa \emph{mesh}
%     \item Metode dekomposisi domain
%     \item Metode \emph{multigrid}
%     \item Metode Gauss-Seidel
% \end{enumerate}
% Metode beda hingga merupakan metode yang paling mudah dan sering digunakan. Metode elemen hingga dan volume hingga sering digunakan pada bidang keteknikan dan komputasi fluida dinamik, dan cocok untuk permasalahan geometri yang rumit. Metode spektral secara umum paling akurat, dan hasilnya cukup halus.

% Pada tulisan ini, akan digunakan metode Gauss-Seidel sebagai penghasil data latih dan data uji, serta sebagai pembanding dari metode \emph{machine learning} yang akan digunakan.

% \section{Metode Gauss-Seidel}
% \subsection{Skema Numerik}
% Sebagai metode iteratif, metode Gauss-Seidel memiliki karakteristik umum untuk mereformulasi bentuk
% \begin{equation}\label{linear}
%    A\textbf{X} = \textbf{B}
% \end{equation}
%  menjadi
% \begin{equation}
%     \textbf{x}^{(n+1)} = M\textbf{x}^{(n)} + \textbf{c}
% \end{equation}
% dengan $n$ merupakan jumlah iterasi dan $M$ merupakan matriks iterasi dan $c$ merupakan vektor baru yang terbentuk dari proses reformulasi \citep{Blackledge2006}.

% Pada persamaan \eqref{linear}, $A$ merupakan matriks persegi nonlinear dengan orde $N$, $X = (x_1, x_2, ..., x_N)^T$ adalah vektor yang tidak diketahui, dan $B$ merupakan matriks yang diketahui. Persamaan \eqref{linear} terdiri dari $N$ persamaan aljabar linear:
% \begin{equation}\label{linear_matriks}
%     \begin{split}
%         a_{11}x_1 + a_{12}x_2 +        ...       + a_{1N}x_N = b_1,\\
%         a_{21}x_1 + a_{22}x_2 +        ...       + a_{2N}x_N = b_2,\\
%         a_{i1}x_1 + a_{i2}x_2 +    a_{ii}x_i     + a_{iN}x_N = b_i,\\
%         a_{N1}x_1 + a_{N2}x_2 +        ...       + a_{NN}x_N = b_N.\\
%     \end{split}
% \end{equation}
% Diasumsikan akan di selesaikan permasalahan \eqref{linear_matriks} secara iteratif. Dimulai dengan menginisialisasi dengan vektor aproksimasi $X^0 = (x_1^0, x_2^0, ..., x_N^0)^T$ yang berikutnya akan dijadikan solusi aproksimasi pada iterasi ke-$r$, $X^r = (x_1^r, x_2^r, ..., x_N^r)^T$. Dari nilai $X^r$ akan dicari aproksimasi yang lebih baik untuk nilai $X^{r+1}$.

% Dalam metode Gauss-Seidel, terdapat pengembangan dari nilai konvergensi dibandingkan metode iteratif serupa, metode Jacobi. Pada metode Gauss-Seidel (atau metode perpindahan berturut), nilai yang ditemukan sebelumnya $x_1^{r+1}, x_2^{r+1}, ..., x_{i-1}^{r+1}$ digunakan untuk menyelesaikan persamaan ke-$i$ untuk $x_i^{r+1}$. Sehingga bentuk umum solusinya menjadi:

% \begin{equation}\label{konfigurasi gauss seidel}
%     \begin{split}
%         x_i^{r+1} = \frac{1}{a_{ii}} \left(b_i-\sum_{j=1}^{i-1}a_{ij}x_j^{r+1} - \sum_{j=i+1}^N a_{ij}x_j^r \right)\\
%         x_N^{r+1} = \frac{1}{a_{NN}} \left(b_N-\sum_{j=1}^{N-1}a_{Nj}x_j^{r+1}\right)
%     \end{split}
% \end{equation}

% \subsection{Formalisme Umum}

% Pada persamaan \eqref{linear}, matriks $A$ dapat dibagi menjadi diagonal bawah $L$, diagonal $D$ dan diagonal atas $U$:

% \begin{equation}
%     A = L + D + U.
% \end{equation}

% Kemudian,

% \begin{equation}
%     \begin{split}
%         (L+D+U)x=b,\\
%         Dx=-Lx-Ux+b,
%     \end{split}
% \end{equation}

% dan

% \begin{equation}
%     x = -D^{-1}Lx - D^{-1} Ux+D^{-1}b
% \end{equation}
% dengan
% \begin{equation}
%     D^{-1} = diagonal(a_{11}^{-1}, a_{22}^{-1},...,a_{nn}^{-1}).
% \end{equation}
% Kemudian metode Gauss-Seidel dapat ditulis sebagai berikut:

% \begin{equation}
%     x^{n+1} = -D^{-1}Lx^{n+1}-D^{-1}Ux^n+D^{-1}b,
% \end{equation}

% dan setelah disusun ulang menjadi

% \begin{equation}
%     x^{n+1} = -(D+L)^{-1}Ux^{n}+(D+L)^{-1}b = Mx^n+c
% \end{equation}

% \subsection{Penambahan Parameter Relaksasi}
% Secara umum, metode Gauss-Seidel dapat dikembangkan dengan menambahkan parameter \emph{sucessiver over-relaxation} (SOR) \citep{Bottoni2022}. Vektor $X^{r+1}$ yang diperoleh dari persamaan \eqref{konfigurasi gauss seidel} dianggap sebagai nilai sementara, misalkan $X^*$, dan nilai yang lebih baik dicari menggunakan persamaan:

% \begin{equation}\label{gauss seidel with provisional value}
%     X^{r+1} = X^r + \omega(X^* - X^r).
% \end{equation}

% yang merupakan bentuk simplikasi dari persamaan:
% \begin{equation}\label{metode relaksasi}
%     x_i^{r+1} = x_i^k+\frac{\omega}{a_{ii}} \left(b_i-\sum_{j=1}^{i-1}a_{ij}x_j^{r+1} - \sum_{j=i+1}^N a_{ij}x_j^r \right)
% \end{equation}

% Persamaan \eqref{gauss seidel with provisional value} menunjukkan bahwa nilai yang lebih baik $X^{r+1}$ dicari dengan menambahkan solusi sebelumnya $X^r$ dengan kenaikan ($X^* - X^r)$ kemudian dikali dengan parameter $\omega$, yang disebut dengan parameter relaksasi. Metode relaksasi menggunakan parameter relaksasi $\omega$ untuk 'menyetel' sistem agar jumlah iterasi yang dibutuhkan berkurang. Untuk alasan stabilitas, $\omega$ harus berada pada rentang ($0 \le \omega \le 2).$ Skema \eqref{gauss seidel with provisional value} disebut \emph{successive under-relaxation} (SUR) apabila $\omega \le 1$ dan \emph{successive over-relaxation} apabila $1 < \omega \le 2$. Saat $\omega$ = 1, $X^{r+1} = X^*$, yang artinya persamaan Gauss-Seidel.

% Konvergensi persamaan Gauss-Seidel dapat diperiksa dengan kriteria toleransi larat yang diinginkan $\varepsilon_s$ \citep{chapra2015}:
% \begin{equation}
%     |\varepsilon_{ij}| = \left|\frac{x_{ij}^n-x_{ij}^{n-1}}{x_{ij}^n}\right| 100\% < \varepsilon_s,
% \end{equation}
% dengan $\varepsilon_{ij}$ merupakan ralat dari sebuah titik pada koordinat yang dihitung dengan mengoperasikan dengan titik yang sama pada iterasi ke-$n$ dan ke-$n-1$.

% \subsection{Metode Gauss-Seidel untuk Penyelesaian Persamaan Poisson di Koordinat Kartesian Dua Dimensi}
% Persamaan Poisson dua dimensi pada koordinat kartesian ($x,y$) diberikan oleh persamaan berikut:
% \begin{equation}\label{poisson_kartesian}
%     \left(\frac{\partial^2}{\partial x^2} + \frac{\partial^2}{\partial y^2}\right) \phi_{ij} = \frac{-\rho_{ij}}{\epsilon_{0}}
% \end{equation}

% Potensial listrik yang akan dicari yang berada pada titik $i$ dan $j$ pada bidang kartesian diwakili oleh simbol $\phi$. Distribusi partikel direpresentasikan oleh $\rho$ dan permitivitas bahan oleh $\epsilon_0$. Pembaruan langkah dari $x$ dan $y$ diwakili oleh simbol $\Delta x$ dan $\Delta y$. Kemudian $\phi(x,y)$ didiskretisasi pada titik ($x_i, y_j$). Titik ($x_i, y_j$) akan ditulis sebagai indeks ($i,j$) dan komponen $\phi(x,y)$ ditulis sebagai $\phi_{i,j}$. Dengan diasumsikan bahwa ada sebanyak $M$ titik sepanjang arah $x$ dan $N$ titik sepanjang arah $y$ yang membentuk \emph{mesh}, maka $i = 1, 2, ..., M$ dan $j = 1,2, ...,N$. Ukuran langkah sepanjang arah $x$ diwakili oleh $\Delta x$ dan arah $y$ oleh $\Delta y$. Untuk $x,y \neq 0$, digunakan skema \emph{central difference} untuk tiap suku pada persamaan \eqref{poisson_kartesian}:

% \begin{equation}\label{diskrit kartesian}
%     \begin{split}
%         \frac{\partial^2}{\partial x^2} \phi=\frac{\phi_{i+1,j}-2 \phi_{ij}+\phi_{i-1, j}}{\Delta x^2}\\
%         \frac{\partial^2}{\partial y^2} \phi = \frac{\phi_{i,j+1}-2 \phi_{ij}+\phi_{i, j-1}}{\Delta y^2}.
%     \end{split}
% \end{equation}.

% Subtitusi persamaan \eqref{diskrit kartesian} ke \eqref{poisson_kartesian} menghasilkan:

% \begin{equation}
%     \frac{\phi_{i+1,j}-2 \phi_{ij}+\phi_{i-1, j}}{\Delta x^2}+\frac{\phi_{i,j+1}-2 \phi_{ij}+\phi_{i, j-1}}{\Delta y^2} = \frac{-\rho_{ij}}{\epsilon_{0}}.
% \end{equation}

% Dengan mempertimbangkan persamaan \eqref{konfigurasi gauss seidel} untuk iterasi, maka dihasilkan konfigurasi sebagai berikut:

% \begin{equation}
%     -2\left(\frac{1}{\Delta x^2} + \frac{1}{\Delta y^2}\right) \phi_{ij}^{n+1} = -\left(\frac{\phi^{n}_{i+1,j}+\phi^{n}_{i-1,j}}{\Delta x^2}-\frac{\phi^{n}_{i,j+1}+\phi^{n}_{i,j-1}}{\Delta y^2}\right),\\
% \end{equation}

% Disusun ulang menjadi:

% \begin{equation}
%     \phi^{n+1}_{ij} \left(\frac{\Delta y^2 + \Delta x^2}{\Delta x^2 \Delta y^2} \right) = \frac{\left(\frac{\phi^{n}_{i+1,\ j}+\phi^{n}_{i-1,\ j}}{\Delta x^2}\right)+\left(\frac{\phi^{n}_{i,\ j+1}+\phi^{n}_{i,\ j-1}}{\Delta y^2}+\frac{\rho_{ij}}{\epsilon_0}\right)}{2}
% \end{equation}

% \begin{equation}\label{GS_cartesian}
%     \phi_{ij}^{n+1}=\frac{\left(\frac{\phi^n_{i+1,\ j}+\phi^n_{i-1,\ j}}{\Delta x^2}\right)+\left(\frac{\phi^n_{i,\ j+1}+\phi^n_{i,\ j-1}}{\Delta y^2}\right)+\frac{\rho_{i,j}}{\epsilon_0}}{2} \cdot \frac{\Delta x^2 \Delta y^2}{\Delta y^2 + \Delta x^2}
% \end{equation}
% Persamaan \eqref{GS_cartesian} merupakan persamaan penyelesaian Persamaan Poisson dua dimensi pada koordinat kartesian.

% \subsection{Metode Gauss-Seidel untuk Penyelesaian Persamaan Poisson di Koordinat Silinder Dua Dimensi}\label{gauss_seidel_silinder}
% Pada penelitian ini akan digunakan koordinat silinder pada sumbu $r$ dan $z$ (simetri aksial). Persamaan Poisson dua dimensi pada koordinat silinder ($r, z$) diberikan oleh persamaan \eqref{Poisson silinder awal} \citep{Shiferaw2013}:
% \begin{equation}\label{Poisson silinder awal}
%     \nabla ^2 \phi = \frac{\partial^2\phi}{\partial r^2} + \frac{1}{r}\frac{\partial \phi}{\partial r}+\frac{\partial^2\phi}{\partial z^2}=\frac{\rho_{ij}}{\epsilon_0}
% \end{equation}

% Potensial listrik yang akan dicari yang berada pada titik $i$ dan $j$ pada koordinat silinder diwakili oleh simbol $\phi$. Distribusi partikel direpresentasikan oleh $\rho$ dan permitivitas bahan oleh $\epsilon_0$. Pembaruan langkah dari $r$ dan $z$ diwakili oleh simbol $\Delta x$ dan $\Delta y$, Kemudian $\phi(r,z)$ didiskretisasi pada titik ($r_i, z_j$). Titik ($r_i, z_j$) akan ditulis dengan indeks ($i,j$) dan $\phi(r,z)$ ditulis sebagai $\phi_{i,j}$. Dengan diasumsikan bahwa ada sebanyak $M$ titik sepanjang arah $r$, dan $N$ titik sepanjang arah $z$ yang membentuk \emph{mesh}, maka $i = 1,2,...,M, j=1,2,...,N$. Ukuran langkah sepanjang arah $r$ diwakili oleh $\Delta r$ dan arah $z$ oleh $\Delta z$. Untuk $r \neq 0$, digunakan skema \emph{central difference} untuk tiap suku pada persamaan \eqref{Poisson silinder awal}:

% \begin{equation}\label{diskretisasi_poisson_silinder}
% \begin{split}
%     \frac{\partial \phi}{\partial r} = \frac{\phi_{i+1,j}-\phi_{i-1,j}}{2\Delta r}\\
%     \frac{\partial^2 \phi}{\partial r^2} = \frac{\phi_{i+1,j}-2\phi_{i,j}+\phi_{i-1,j}}{(\Delta r)^2}\\
%     \frac{\partial^2 \phi}{\partial z^2} = \frac{\phi_{i,j+1}-2\phi_{i,j}+\phi_{i,j-1}}{(\Delta z)^2}
% \end{split}
% \end{equation}

% Dengan subtitusi persamaan \eqref{diskretisasi_poisson_silinder} ke persamaan \eqref{Poisson silinder awal}, didapat persamaan:

% \begin{equation}
%     \frac{\phi_{i+1,j}-2\phi_{i,j}+\phi_{i-1,j}}{(\Delta r)^2} + \frac{\phi_{i+1,j}-\phi_{i-1,j}}{r_i 2 \Delta r} + \frac{\phi_{i,j+1}-2\phi_{ij}+\phi_{i-1,j}}{(\Delta z)^2}=\frac{-\rho_{ij}}{\epsilon_0}.
% \end{equation}

% Kemudian dilakukan perhitungan menghasilkan:
% \begin{equation}
%     \begin{split}
%         \phi_{ij} &= \frac{\frac{\rho_{ij}}{\epsilon_0}+\left(\frac{\phi_{i+1,j}-\phi{1-1,j}}{r_i2\Delta r}\right)(\Delta r)^2 (\Delta z)^2 + \phi_{i-1,j}\left((\Delta z)^2+(\Delta r)^2\right)}{-2\left((\Delta r)^2 + (\Delta z)^2\right)}\\
%         &+\frac{(\Delta z)^2 \phi_{i+1,j} + \phi_{i,j+1}(\Delta r)^2}{-2\left((\Delta r)^2 + (\Delta z)^2\right)}
%     \end{split}
% \end{equation}
% Dengan mempertimbangkan persamaan \eqref{konfigurasi gauss seidel} untuk iterasi, maka dihasilkan konfigurasi persamaan iterasi Gauss-Seidel untuk persamaan Poisson 2 dimensi di koordinat silinder ($r,z$):

% \begin{equation}\label{GS_silinder}
%     \begin{split}
%         \phi^{n+1}_{ij} = \frac{\frac{\rho_{ij}}{\epsilon_0}+\left(\frac{\phi^n_{i+1,j}-\phi^n_{i-1,j}}{r_i 2 \Delta r}\right)(\Delta r)^2(\Delta z)^2 + \phi^n_{i-1,j}((\Delta z)^2 + (\Delta r)^2) + (\Delta z)^2 \phi^n_{i+1,j} + \phi^n_{i,j+1}(\Delta r)^2}{-2((\Delta r)^2 + (\Delta z)^2)}
%     \end{split}
% \end{equation}

% \section{Syarat Batas}

% Dalam bidang persamaan diferensial, permasalahan nilai batas merupakan persamaan diferensial dengan sekumpulan pembatas tambahan yang disebut syarat batas \citep{zwilinger2022}. Solusi dari persamaan diferensial parsial umumnya tidak unik. Agar unik, persamaan diferensial parsial harus menggunakan syarat batas dan syarat awal \citep{waletPartial}. Syarat batas numerik muncul dalam proses implementasi numerik pada syarat batas fisis tertentu pada keadaan fisis sesungguhnya ataupun cuplikan domain (karena keterbatasan sumber daya komputasi)\citep{Brio2010}. Syarat batas menunjukkan perilaku dari sebuah fungsi pada batas area yang didefinisikan.

% Untuk dapat digunakan, permasalahan nilai batas harus didefinisikan secara jelas. Artinya, bahwa dengan memberikan masukan untuk masalah tertentu, terdapat solusi yang unik yang bergantung terus menerus pada masukan tersebut. Masalah nilai batas memiliki kondisi yang ditentukan pada kondisi ekstrim (batas) dari variabel independen pada persamaan.

% \subsection{Syarat Batas Dirichlet}
% Syarat batas Dirichlet merupakan syarat batas yang menemukan nilai dari fungsi tersebut di syarat batas (syarat batas tipe pertama). Dalam persamaan diferensial parsial, syarat batas Dirichlet digambarkan secara matematis dengan contoh berikut:
% \begin{equation*}
%     \nabla^2 u = f(x),
% \end{equation*}
% dengan $\nabla$ merupakan operator laplace dan $f$ merupakan fungsi yang diketahui, maka syarat batas Dirichlet pada domain $\Omega \subset \textbf{R}^n$ memiliki bentuk:
% \begin{equation}
%     u(x) = f(x)\quad \quad \forall x \in \partial \Omega
% \end{equation}

% Dalam konteks persamaan Poisson, syarat batas Dirichlet dan interiornya diilustrasikan dalam Gambar \ref{gambar dirichlet}.

% \begin{figure}[h]
%     \centering
%     \includegraphics[width=10cm]{gambar/dirichlet.png}
%     \caption{Contoh geometri syarat batas Dirichlet pada persamaan Poisson di koordinat kartesian 2 dimensi ($x,y$) pada domain $\Omega$.\ \emph{Sumber: Penulis}}
%     \label{gambar dirichlet}
% \end{figure}

% \subsection{Syarat Batas Neumann}
% Syarat batas Neumann merupakan syarat batas yang menemukan nilai dari derivatif normal dari fungsi tersebut (syarat batas kedua). Dalam persamaan diferensial parsial, syarat batas Neumann digambarkan secara matematis dengan contoh berikut:
% \begin{equation*}
%     \nabla^2 y + y = f(x),
% \end{equation*}

% dengan $\nabla^2$ melambangkan operator Laplace, maka syarat bentuk syarat batas Neumannya pada domain $\Omega \subset \mathbf{R}^n$ menjadi:
% \begin{equation}\label{NeumannBC}
%     \frac{\partial y}{\partial \textbf{n}}(x) = f(x)\quad \forall x \in \partial \Omega,
% \end{equation}
% dengan $\mathbf{n}$ menunjukkan vektor normal eksterior ataupun interior pada batas $\partial\Omega$, dan $f$ adalah fungsi skalar yang diberikan.

% Derivatif normal yang ditunjukkan pada sisi kiri persamaan \eqref{NeumannBC} didefinisikan sebagai berikut:
% \begin{equation}
%     \frac{\partial y}{\partial\mathbf{n}}(x) = \nabla y (x) \cdot \hat{\mathbf{n}}(x),
% \end{equation}
% dengan $\nabla y (x)$ mewakili gradien dari vektor $y(x)$, $\hat{\mathbf{n}}$ adalah unit normal, dan $\cdot$ sebagai operator \emph{inner product}.

% \section{Pemelajaranan Mesin (\emph{Machine Learning})}
% \subsection{Konsep Umum}
% pemelajaranan mesin (US: \emph{machine learning}) adalah sekumpulan algoritma yang mencoba mengaktifkan kemampuan pemelajaranan komputer, sehingga mereka dapat belajar dari data atau pengalaman masa lalu \citep{bayen_2020_python}.
% Tom Mitchell dapat menggambarkan konsep umum pemelajaranan mesin dengan singkat dan mudah dimengerti pada kata pengantar bukunya: bidang pemelajaranan mesin berkaitan dengan pertanyaan tentang bagaimana membuat program komputer yang secara otomatis meningkat seiring dengan meningkatnya pengalaman \citep{Mitchell1997}.

% Mitchell memberikan sedikit formalisme mengenai pendapatnya tentang konsep umum pemelajaranan mesin tersebut: sebuah program komputer dikatakan belajar dari pengalaman \textbf{E} (\emph{experience}) sehubungan dengan beberapa kelas tugas \textbf{T} (\emph{task}) dan ukuran kinerja \textbf{P} (\emph{performance}), jika kinerjanya pada tugas \textbf{T}, yang diukur dengan \textbf{P}, meningkat dengan pengalaman \textbf{E}.

% Istilah pemelajaran mesin digunakan dalam hal yang sangat umum dan merujuk pada teknik yang umum untuk mengekstrapolasi pola dari kumpulan yang besar atau kemampuan untuk membuat prediksi pada basis data yang baru tentang hal yang dipelajari melalui analisa yang tersedia dari data yang diketahui sebelumnya \citep{zocca_spacagna_slater_roelants_2017}

% \subsection{Tipe-tipe pemelajaranan Mesin}
% Biasanya, pemelajaranan mesin dibagi ke dalam dua kategori: pemelajaranan dengan pengawasan (\emph{supervised learning}) dan pemelajaranan tanpa pengawasan (\emph{unsupervised learning})\citep{zocca_spacagna_slater_roelants_2017}. Pengawasan dalam hal ini adalah pemberian label yang benar atau jawaban dari permasalahan yang akan dipecahkan. Informasi tersebut akan digunakan selama pelatihan (\emph{training}).

% Dalam pemelajaranan dengan pengawasan, berdasarkan pada sifat luarannya, dapat dibagi ke dalam 2 algoritma: klasifikasi (\emph{classification}) dan regresi (\emph{regression}). Klasifikasi adalah algoritma yang mengeluarkan hasil berupa data kategori hasil pemelajaranan mengenai data latih yang diberikan. Sedangkan regresi merupakan algoritma yang menghasilkan data kuantitas dari sebuah data keadaan.

% Dalam pemelajaranan tanpa pengawasan, yang merupakan algoritma tanpa label, dapat dibagi menjadi dua algoritma, yaitu pengklasteran (\emph{clustering}) dan pengurangan dimensi (\emph{dimensionality reduction}). Dalam algoritma pengklasteran, dibutuhkan ciri-ciri tersembunyi dari objek yang digunakan untuk dilakukan pengelompokkan. Algoritma pengurangan dimensi adalah sekelompok algoritma pemelajaranan tanpa pengawasan untuk mengurangi masalah dimensi yang lebih tinggi menjadi dimensi yang lebih rendah \citep{bayen_2020_python}.

% Masih ada banyak tipe-tipe algoritma pemelajaranan mesin yang tidak dibahas karena tidak relevan dengan penelitian ini. Pada penelitian ini sendiri akan menggunakan tipe algoritma \emph{deep learning} untuk regresi, yaitu \emph{convolutional neural network} (CNN).

% \subsection{Analisis Regresi}
% Algoritma regresi merupakan jenis algoritma \emph{supervised learning} yang menggunakan fitur dari data masukan untuk memprediksi sebuah nilai (biasanya nilai kontinyu), seperti harga rumah yang diberi nilai masukan ukuran, usia, jumlah kamar mandi, jumlah lantai, lokasi, dan sebagainya \citep{zocca_spacagna_slater_roelants_2017}. Analisis regresi berusaha untuk mencari nilai dari parameter fungsi yang paling cocok pada dataset masukan. Tujuan utama dari analisis regresi adalah meminimalkan fungsi kerugian dengan mencari parameter yang pantas untuk fungsi pada data masukan yang mendekati data target dengan sangat baik \citep{zocca_spacagna_slater_roelants_2017}.

% Regresi linear pada dasarnya adalah model linear. Misalnya, sebuah model yang mengasumsikan hubungan linear antara variabel masukan ($x$) dan variabel luaran tunggal ($y$). Lebih spesifik, $y$ dapat dihitung dari kombinasi linear dari variabel masukan $x$ \citep{brownlee_2016}.

% Analisis regresi memiliki variabel respon berupa numerik, misalnya seperti memprediksi usia seseorang dari karakteristik orang tersebut. Ada dua langkah dalam membangun model regresi, yaitu:
% \begin{enumerate}
%     \item Langkah pemelajaranan
%     \item Langkah regresi
% \end{enumerate}

% \section{Jaringan Saraf Buatan/ \emph{Neural Network} (NN)}
% \emph{Neural network} merupakan algoritma \emph{deep learning} (pemelajaran dalam) yang merupakan metode yang bersifat hierarkis dan berlapis. Istilah berlapis ini yang kemudian memberi makna pada kata '\emph{deep}'. Disebut demikian karena model ini menggunakan banyak lapisan proses pemrosesan data (\emph{layers}). Setiap lapisan berisi \emph{node} yang melakukan serangkaian kalkulasi matematis dan masing-masing lapisan belajar untuk mengenali fitur yang berbeda dari masukan yang diberikan.

% \emph{Deep learning} khususnya jaringan saraf buatan dapat dipahami sebagai proses pengektrakan otomatis fitur dari data dan menggunakannya untuk melakukan prediksi atau keputusan. Hal ini sangat berbeda dari metode pemelajaran mesin tradisional, yang pendefinisian fitur harus dilakukan secara manual dan kemudian diaplikasikan pada model pemelajaran mesin (Gambar \ref{deepflow})\citep{Li_Li_Gao}

% \begin{figure}[h]
%     \centering
%     \includegraphics[width=10cm]{gambar/alur_deep_learning.png}
%     \caption{Pemelajaran mesin tradisional membutuhkan ekstraksi fitur secara manual. Pemelajaran dalam (\emph{deep learning}) mengekstrak fitur secara otomatis dengan memberikan input pada lapisannya}
%     \label{deepflow}
% \end{figure}

% Kemudian muncul pertanyaan mengenai awal inspirasi terbentuknya ide dasar mengenai pembentukan jaringan saraf buatan. Seperti halnya banyak penemuan yang diilhami oleh alam, begitupun dalam pembentukan model dari pemelajaran mesin yang cerdas, para ilmuwan berusaha untuk mencari inspirasi ke alam, dan didapatkanlah sebuah pemikiran bahwa algoritma dari sebuah mesin yang cerdas dapat diilhami pula dari sebuah mekanisme yang kompleks dan cerdas, yaitu arsitektur otak manusia \citep{aurélien_géron_2022}. Logika inilah yang menghasilkan jaringan saraf buatan, sebuah model pemelajaran mesin yang diilhami dari jaringan saraf manusia.

% Satuan komputasional paling mendasar pada otak manusia adalah neuron yang saling terkoneksi dengan sinapsis. Setiap neuron menerima sinyal masukan dari dendrit dan menghasilkan sinyal luaran yang ditransmisikan sepanjang akson. Akson tersebut kemudian bercabang dan tersambung melalui sinapsis ke dendrit neuron lain.

% Dalam model komputasional sebuah neuron, sinyal (misalnya $x_0$) yang dihantarkan melalui akson, berinteraksi berkali-kali (misalnya $w_0 x_0$) dengan dendrit lain berdasarkan kekuatan atau bobot sinaptik pada sinapsis yang bersangkutan (misalnya $w_0$). Bobot $w$ ini merupakan parameter yang dapat belajar dan mengontrol kekuatan pengaruh dari suatu neuron ke neuron lainnya.

% Pada model yang sangat standar, dendrit membawa sinyal ke sel tubuh dan semuanya dijumlahkan. Apabila hasil penjumlahan akhir di atas ambang batas, neuron aktif dan meneruskan sinyal melalui akson. Dalam model komputasionalnya, pewaktuan (\emph{timing}) bukanlah hal yang penting, melainkan frekuensi penyampaian informasi. Dari interpretasi ini, kita memodelkan laju aktif dari neuron sebagai fungsi aktifasi $f$ yang merepresentasikan frekuensi lonjakan di sepanjang akson.

% \begin{figure}[ht]
%   \centering
%   \begin{minipage}{0.45\textwidth}
%     \centering
%     \includegraphics[width=1\linewidth]{gambar/neuron.png} % Adjust the width as necessary
%   \end{minipage}\hfill
%   \begin{minipage}{0.45\textwidth}
%     \centering
%     \includegraphics[width=1\linewidth]{gambar/neuron_model.jpeg} % Adjust the width as necessary
%   \end{minipage}
%   \label{neuron_model}
%   \caption{Kiri: Gambaran skema saraf biologis. Kanan: skema model matematis jaringan saraf buatan. \emph{Sumber: CS231n}}
% \end{figure}

% \subsection{Struktur Neural Network}
% Disadur dari \cite{aurélien_géron_2022}, terdiri dari 4 bagian utama:
% \begin{enumerate}
%     \item \emph{Input layer}: Lapisan yang menerima data masukan. Setiap neuron dalam lapisan ini mewakili satu fitur data.
%     \item \emph{Hidden layaer(s)}: Lapisan ini berada di antara lapisan masukan dan luaran. Jumlah lapisan dan neuron dalam bagian ini dapat bervariasi sesuai kompleksitas masalah.
%     \item \emph{Weight connection}: \emph{Weights} atau bobot diterapkan pada tiap sambungan antara node untuk merepresentasikan seberapa penting pengaruh node tersebut pada hasil luaran.
%     \item \emph{Output layer}: Lapisan ini memberikan hasil akhir dari jaringan saraf (selanjutnya, padanan kata \emph{neural network} akan ditulis sebagai jaringan saraf, tanpa kata buatan). Jumlah neuron dalam lapisan ini biasanya sama dengan jumlah kelas dalam masalah klasifikasi atau berjumlah satu neuron dalam masalah regresi.
% \end{enumerate}

% \subsection{\emph{Forward Propagation} (Perambatan Maju)}
% Tahap ini merupakan langkah awal dalam proses belajar sebuah jaringan saraf. Dalam tahap ini, masukan $x$ dikirimkan melalui jaringan dari lapisan masukan ke lapisan luaran. Setiap neuron mengambil $x$ kemudian mengalikan dengan bobot $w$, menambahkan bias $b$ dan kemudian meneruskan hasil melalui fungsi aktivasi \citep{goodfellow_bengio_courville_2016}:
% \begin{equation}
%     s_i = w_i^T x_i +b_i
% \end{equation}

% yang kemudian diikuti oleh fungsi aktivasi nonlinear seperti pada Gambar \ref{feedforward}:
% \begin{equation}
%     y_i = h(s_i)
% \end{equation}

% \begin{figure}[h]
%     \centering
%     \includegraphics[width=12cm]{gambar/feedforward.png}
%     \caption{Perkalian input dengan bobot pada unit neuron yang kemudian diteruskan dengan operasi dengan fungsi aktivasi. \emph{Sumber: \cite{szeliski_2011}}}
%     \label{feedforward}
% \end{figure}

% $x_i$ merupakan masukan dari unit ke-$i$, $w_i$ dan $b_i$ berturut-turut adalah bobot dan bias yang dapat belajar, $s_i$ adalah luaran dari penjumlahan linear berbobot, dan $y_i$ adalah luaran final setelah $s_i$ masuk pada fungsi aktivasi $h$. Luaran dari tiap tahap lapisan akan menjadi input untuk lapisan selanjutnya. Lapisan pada jaringan saraf terhubung secara berurutan dengan lapisan saraf berikutnya seperti pada Gambar \ref{fully_connected}.

% \begin{figure}[h]
%     \centering
%     \includegraphics[width=10cm]{gambar/fully_connected.png}
%     \caption{\emph{Fully connected layer} \emph{Sumber: \cite{szeliski_2011}}}
%     \label{fully_connected}
% \end{figure}

% Kita dapat menganggap semua unit dalam lapisan adalah sebuah vektor dengan kombinasi perhitungan linear sebagai berikut:

% \begin{equation}
%     s_l = W_lx_l
% \end{equation}

% dengan $x_l$ adalah masukan pada layer $l$, $W_l$ adalah matriks bobot, dan $s_l$ adalah penjumlahan berbobot. Kemudian kenonlinearan diterapkan berdasarkan elemen untuk dijadikan masukan untuk layer selanjutnya ($l+1$) dengan rumusan:

% \begin{equation}
%     x_{l+1} = y_l = h(s_l)
% \end{equation}

% Lapisan yang menggunakan matriks bobot penuh (padat) untuk kombinasi linier disebut lapisan terhubung penuh (\emph{fully connected layer} (FC)). Jaringan yang hanya berisi FC disebut dengan \emph{multi-layer perceptron} (MLP).

% \subsection{Fungsi Aktivasi}
% Secara biologis, jaringan saraf memiliki neuron yang memiliki fungsi tertentu untuk mengolah impuls tertentu. Begitupun dengan jaringan saraf buatan yang menggunakan fungsi aktivasi, yaitu fungsi yang mendefinisikan keadaan internal neuron untuk menghitung masukkan dari neuron lain \citep{zocca_spacagna_slater_roelants_2017}. Fungsi aktivasi bertugas mengubah masukan neuron menjadi luaran yang akan diteruskan ke neuron selanjutnya. Fungsi aktivasi disebut juga dengan fungsi transfer atau kenonlinearan karena mengubah kombinasi linear dari penjumlahan bobot menjadi model nonlinear dan terletak pada ujung tiap unit atau perceptron untuk memutuskan apakah akan mengaktifkan neuron tersebut \citep{elgendy_2020}. Tujuan dari fungsi aktivasi adalah untuk memperkenalkan nonlinearitas pada jaringan. Tanpa fungsi aktivasi, MLP akan berkelakuan mirip dengan perceptron tunggal, entah berapapun jumlah lapisannya. Fungsi aktivasi dibutuhkan untuk membatasi nilai luaran pada batas tertentu. Ada beberapa fungsi aktivasi, seperti ReLU (Rectified Linear Unit) dan tanh (Gambar \ref{fungsi_aktivasi}).

% Pilihan fungsi aktivasi tergantung pada masalah yang sedang dihadapi dan arsitektur dari jaringan saraf yang digunakan. Pada penelitian ini, akan digunakan dua macam fungsi aktivasi, yaitu ReLU pada \emph{hidden layer} dan tanh pada \emph{output layer}.

% \begin{figure}[ht]
%   \centering
%   \begin{minipage}{0.45\textwidth}
%     \centering
%     \includegraphics[width=1\linewidth]{gambar/relu1.png} % Adjust the width as necessary
%   \end{minipage}\hfill
%   \begin{minipage}{0.45\textwidth}
%     \centering
%     \includegraphics[width=1\linewidth]{gambar/tanh.png} % Adjust the width as necessary
%   \end{minipage}
%   \caption{Kiri: Fungsi aktivasi ReLU mengeliminasi semua nilai negatif dari masukan dengan metransformasi menjadi nilai nol. Kanan: Fungsi aktivasi tanh: apabila nilai masukan sangat besar atau sangat kecil, maka nilai gradien akan sangat kecil; mendekati nol. \emph{Sumber: \citep{elgendy_2020}}}
%   \label{fungsi_aktivasi}
% \end{figure}

% \subsubsection{Fungsi Aktivasi \emph{Rectified Linear Unit} (ReLU)}
% Fungsi aktivasi ReLU adalah fungsi bagian-positif (\emph{positive-part function}) yaitu menghasilkan fungsi identitas untuk argumen (masukkan) positif dan menghasilkan nol untuk masukkan lainnya \citep{Lederer2021}. Fungsi bagian-positif membiarkan masukan positif lewat tanpa diubah, tapi memotong masukan negatif (mengubah masukan negatif menjadi luaran nonnegatif). Secara matematis dapat digambarkan dengan persamaan \eqref{relu} dengan $z$ adalah elemen masukan:
% \begin{equation}\label{relu}
%     ReLU(z) =
%     \begin{dcases}
%         z & \text{jika } z \leq 0 \\
%         0 & \text{jika } z < 0
%     \end{dcases}
% \end{equation}
% Fungsi aktivasi ReLU sering digunakan pada berbagai model jaringan saraf tiruan karena sifat linearitasnya sehingga lebih mudah untuk dilatih dan menghasilkan performa yang lebih baik \citep{elgendy_2020}.

% \subsubsection{Fungsi Aktifasi Tanh (\emph{Hyperbolic Tangent)}}
% Fungsi aktivasi tanh adalah versi pergeseran dari fungsi sigmoid. Singkatnya, pada fungsi sigmoid membatasi nilai luaran pada rentang nilai 0 hingga 1, sedangkan pada tanh, membatasi nilai pada rentang nilai -1 hingga 1. Tanh bekerja dengan lebih baik daripada sigmoid pada karena memliki efek pemusatan data, sehingga memiliki rata-rata data mendekati 0, bukannya 0,5 seperti sigmoid, sehingga membuat pemelajaran untuk lapisan selanjutnya lebih mudah \citep*{elgendy_2020}. Fungsi tanh dirumuskan sebagai berikut untuk nilai masukan $z$:
% \begin{equation}
%     tanh(z) = \frac{sinh(z)}{cosh(z)} = \frac{e^x-e^{-x}}{e^x + e^{-x}}
% \end{equation}

% \subsection{\emph{Backward Propagation} (Perambatan Mundur)}
% Konsep umum yang perlu kita pahami adalah bahwa setiap jaringan saraf adalah pendekatan pada sebuah fungsi. Sehingga hasil yang dihasilkan akan memiliki perbedaan nilai dari fungsi yang didekati. Nilai inilah yang kita sebut dengan kesalahan/ralat (\emph{error}) yang berusaha untuk kita minimalkan nilainya. Karena ralat yang muncul merupakan fungsi bobot (\emph{weights}), maka kita akan mengecilkan ralat terhadap nilai bobot. Secara matematis, himpunan titik-titik yang fungsinya nol mewakili suatu permukaan hiper (\emph{hypersurface}) dan untuk mencari titik minimum pada permukaan ini kita ingin memilih sebuah titik dan kemudian mengikuti kurva ke arah minimum \citep{zocca_spacagna_slater_roelants_2017}. Kesalahan ini kemudian dipropagasi mundur melalui jaringan, yang kemudian bobot antar neuron akan disesuaikan berdasarkan kesalahan tersebut. Proses ini dilakukan berulang-ulang hingga kesalahan mencapai nilai minimum atau setelah jumlah iterasi tertentu \citep{szeliski_2011}. Isitilah yang lebih matematis untuk mendefinisikan tentang perambatan mundur adalah menghitung gradien dari sebuah ekspresi melalui penerapan rekursif dari aturan rantai \citep{Li_Li_Gao}; menghitung gradien dari fungsi $f(\textbf{x})$ di $\textbf{x}$ ($\nabla f(\textbf{x})$) dengan $\textbf{x}$ adalah vektor masukan.

% \begin{figure}[h]
%     \centering
%     \includegraphics[width=10cm]{gambar/backprop2.png}
%     \caption{Perambatan maju (\emph{forward propagation}. \emph{Sumber: penulis}}
%     \label{backprop}
% \end{figure}

% Sebagai contoh, akan digunakan 2 lapisan, satu sebagai lapisan masukan (\emph{input layer}) dan lapisan lainnya sebagai lapisan luaran (\emph{output layer}) pada Gambar \ref{backprop}. Misal kita notasikan $J$ sebagai galat (error) antara nilai $Layer 2$ dan $Layer 1$, dengan $y$ adalah fungsi aktivasi dengan input dari niai bobot $w_{i,j}$ dan nilai masukan dari \emph{Layer 1} $y_i$ dan luaran adalah nilai $y_j$.

% Tujuan dari perambatan mundur adalah mencari nilai bobot dari nilai galat yang dihasilkan $\frac{\partial J}{\partial w_{i,j}}$. Dengan menggunakan aturan rantai, didapat rumusan:
% \begin{equation}
%     \frac{\partial J}{\partial w_{i,j}} = \frac{\partial J}{\partial y_j} \frac{\partial y_j}{\partial y_i} \frac{\partial y_i}{\partial w_{i,j}}
% \end{equation}

% \subsection{Fungsi Kerugian (\emph{Loss Function})}
% Jaringan saraf dilatih menggunakan \emph{stochastic gradient descent} dan bobot diperbarui menggunakan algoritma perambatan mundur ralat \citep{brownlee_2019}. Dalam rangka mengoptimasi bobot, harus didefinisikan fungsi kerugian (\emph{loss function}) yang diminimalkan selama proses pelatihan \citep{szeliski_2011}.

% Fungsi kerugian mengkuantifikasi jarak antara nilai sesungguhnya dan nilai hasil prediksi dari data target. Kerugian (\emph{loss}) biasanya merupakan angka nonnegatif yang semakin kecil, berarti nilainya semakin bagus \citep{goodfellow_bengio_courville_2016}.

% Dalam optimisasi matematis dan teori keputusan, fungsi kerugian adalah fungsi yang memetakan kejadian atau nilai dari satu atau lebih variabel terhadap nilai kebenaran yang secara intuitif merepresentasikan harga yang terasosiasi dengan kejadian tersebut. Secara sederhana, fungsi kerugian adalah metode untuk mengevaluasi seberapa baik algoritma memodelkan set data \citep{shankar_2022}.

% \subsubsection{Rerata Ralat Kuadrat (\emph{Mean Squared Error}/ MSE)/ \emph{L2 Loss}}\label{mse}
% Untuk permasalahan regresi, fungsi kerugian yang paling umum digunakan adalah MSE atau fungsi kerugian L2 \citep{szeliski_2011}:

% \begin{equation}
%    MSE = \frac{1}{n} \sum\limits_n E_n(\textbf{w}) = \frac{1}{n}\sum\limits_n ||\textbf{y}_n - \textbf{t}_n||^2
% \end{equation}

% dengan $\textbf{y}_n$ merupakan prediksi dari jaringan untuk sampel sejumlah $n$ dan $\textbf{t}_n$ merupakan nilai target dari data latih.

% \subsubsection{Rerata Presentasi Ralat Absolut (\emph{Mean Average Percentage Error}/ MAPE) }
% Contoh lain untuk mengukur kualitas regresi adalah dengan MAPE. MAPE seringkali digunakan karena sangat intuitif diinterpretasikan sebagai ralat relatif \citep{DEMYTTENAERE201638}. Untuk $x$ sebagai masukan dari model regresi, $g$ merupakan model regresi, dan $y$ adalah nilai target dari data latih, maka MAPE dirumuskan dengan:
% \begin{equation}\label{mape}
%     MAPE = \frac{1}{n} \sum \limits_n \left(\frac{|g(\textbf{x})-\textbf{y}|}{|\textbf{y}|}\right)
% \end{equation}

% \section{Jaringan Saraf Konvolusional/ \emph{Convolutional Neural Network} (CNN)}
% \subsection{CNN pada \emph{Deep Learning}}
% \emph{Convolutional Neural Network} (selanjutnya akan disebut dengan CNN) merupakan pengembangan dari jaringan saraf buatan (\emph{artificial neural network (ANN)/ neural network (NN)}). Pada dasarnya, CNN masih sama dengan jaringan saraf reguler. Perbedaan utamanya adalah bahwa arsitektur CNN membuat asumsi secara gamblang bahwa masukannya adalah gambar. Hal ini membuat perambatan maju lebih efisien serta mengurangi jumlah parameter pada jaringan secara signifikan \citep{Li_Li_Gao}.

% Lebih lanjut lagi, meskipun jaringan saraf reguler dapat memecahkan masalah dengan gambar, namun tidak dapat menskala-ulang (\emph{rescale}) keseluruhan gambar. Hal ini menimbulkan masalah karena gambar yang dijadikan masukan tetap pada ukuran asli, sehingga membutuhkan banyak parameter, dan terlalu banyak parameter dapat menyebabkan \emph{overfitting} \citep{Li_Li_Gao}.

% Ide utama dari CNN adalah menggunakan lapisan (\emph{layer}) (biasa disebut dengan filter) yang mempunyai tugas spesifik masing-masing untuk memanipulasi gambar untuk mendapatkan fitur tertentu pada gambar \citep{zocca_spacagna_slater_roelants_2017}. Filter yang telah berisi berbagai macam fitur tersebut akan diteruskan kepada node-node pada \emph{layer} yang lainnya untuk digabungkan dan dikenali berdasarkan hasil latihan yang dilakukan.

% Dalam jaringan saraf reguler, gambar diperlakukan sebagai vektor kolom dengan cara me-\emph{flatten} tanpa memperhatikan hubungan spasial antar piksel \citep{zhang2023dive}. Idealnya, jaringan dapat memanfaatkan pengetahuan bahwa piksel yang berdekatan biasanya terkait satu sama lain, untuk membangun model yang efisien untuk belajar dari data gambar \citep*{zhang2023dive}

% Dari keuntungan mengenai wawasan spasial dari jaringan, dibanding NN, CNN dapat memebentuk model yang lebih akurat dengan cara yang efisien. Hal tersebut dikarenakan CNN membutuhkan parameter yang lebih sedikit dan pada CNN juga lebih mudah untuk diterapkan perhitungan paralel menggunakan \emph{graphic processing unit} (GPU) \citep{chetlur2014cudnn}.

% \subsection{Lapisan Konvolusi (\emph{convolution layer})}
% Setelah kita hijrah dari jaringan saraf reguler ke jaringan saraf konvolusi, kita dikenalkan dengan berbagai keuntungan yang sekaligus menjadi prinsip dalam jaringan saraf konvolusi. Salah satu prinsip tersebut adalah translasi invarian \citep{1992SPIE.1709..257Z}. Hal ini berarti pergeseran pada masukan $\textbf{X}$ membuat pergeseran pada lapisan tersembunyi (\emph{hidden layer}) $\textbf{H}$ (keduanya merupakan matriks 2 dimensi dengan ukuran yang sama). Agar setiap unit/neuron pada lapisan tersembunyi menerima masukan dari tiap unit masukan, penggunaan bobot seperti pada jaringan saraf reguler akan diubha menjadi penggunaan parameter $\textbf{V}$.Dengan $u$ adalah bias (konstan), maka lapisan tersembunyi pada CNN didefinisikan sebagai:

% \begin{equation}\label{konvolusi_real}
%     [\textbf{H}]_{i,j} = u + \sum\limits_a \sum\limits_b [\textbf{V}]_{a,b} [\textbf{X}]_{i+a,j+b}
% \end{equation}

% Persamaan \eqref{konvolusi_real} merupakan persamaan konvolusi. Pada persmaan tersebut, masukan $\textbf{X}$ pada indeks $(i+a, j+b)$ diberikan bobot secara efektif dengan koefisien $[\textbf{V}]_{a,b}$ untuk mendapatkan nilai $[\textbf{H}]_{i,j}$ \citep{zhang2023dive}. Dalam komunitas penelitian pemelajaran mendalam (\emph{deep learning}), \textbf{$V$} dikenal sebagai lapisan konvolusi (\emph{convolution layer}), filter, atau lapisan bobot yang merupakan parameter yang dapat belajar \citep{zhang2023dive}.

% Berdasarkan deskripsi tersebut, pada tiap lapisan, matriks masukan dan matriks konvolusi (\emph{kernel}) memproduksi matriks luaran melalui operasi korelasi silang (\emph{cross-correlation}).

% Operasi korelasi silang ditemukan pada filter atau \emph{kernel} konvolusional yang beroperasi pada  gambar dan melakukan perhitungan \emph{dot product}. Setiap filter akan mengekstrak fitur-fitur yang berbeda dari gambar \citep{patel_2020}.

% Sebagai contoh, akan digunakan \emph{kernel} dengan ukuran $3\times 3$ dan ukuran gambar masukan $5\times 5$. Akan dilakukan perkalian berdasarkan elemen (\emph{element-wise}) antara nilai warna piksel dari gambar yang cocok dengan ukuran filter dengan filter itu sendiri, dan kemudian dijumlahkan untuk mendapatkan nilai tunggal.

% \begin{figure}[h]
%     \centering
%     \includegraphics[width=10cm]{gambar/convolution-operation1.png}
%     \caption{Operasi konvolusi pertama. \emph{Sumber: \citep{patel_2020}}}
%     \label{convolution_operation}
% \end{figure}

% Pada Gambar \ref{convolution_operation}, operasi yang dilakukan adalah sebagai berikut:
% $(2 \times 1) + (4 \times 2) + (9 \times 3) + (2 \times (-4)) + (1 \times 7) + (4 \times 4) + (1 \times 2) + (1 \times (-5)) + (2 \times 1) = 51  $

% Kemudian filter konvolusional melanjutkan operasinya dengan bergeser satu sel ke kanan:
% \begin{figure}[h]
%     \centering
%     \includegraphics[width=10cm]{gambar/convolution-operation.png}
%     \caption{Operasi konvolusi kedua. \emph{Sumber: \citep{patel_2020}}}
%     \label{operasi_konvolusi_kedua}
% \end{figure}

% Operasi konvolusi dari filter tersebut adalah: $(4 \times 1) + (9 \times 2) + (1 \times 3) + (1 \times (-4)) + (4 \times 7) + (4 \times 4) + (1 \times 2) + (2 \times (-5)) + (9 \times 1) = 66 $

% Filter tersebut akan terus berjalan dari kiri ke kanan, menuju bawah, hingga tiba pada bagian pojok kanan bawah dengan bergeser 1 piksel. Jumlah pergeseran filter ini dinamakan \emph{stride}. Ukuran \emph{stride} berpengaruh terhadap ukuran fitur (\emph{feature}) yang dihasilkan. Sehingga ukuran fitur secara lengkap yang dihasilkan adalah:

% \begin{equation}\label{ukuran_fitur}
%     ukuran\ fitur = \left(\frac{ukuran\ gambar - ukuran\ filter}{\emph{stride}}\right) + 1
% \end{equation}

% Neuron pada lapisan konvolusi pertama tidak terkoneksi pada tiap piksel pada gambar masukan, melainkan hanya piksel pada bidang reseptif (\emph{receptive fields}) (Gambar \ref{receptive_fields}). Selanjutnya, tiap neuron pada lapisan konvolusi kedua hanya terhubung pada neuron yang terletak di dalam persegi kecil pada lapisan pertama. Struktur arsitektur demikian membuat jaringan tersebut berkonsentrasi pada fitur kecil tingkat rendah pada lapisan konvolusi pertama, dan kemudian menyusun fitur-fitur tersebut pada tingkat yang lebih besar di lapisan selanjutnya, dan seterusnya \citep{aurélien_géron_2022}.

% \begin{figure}[h]
%     \centering
%     \includegraphics[width=10cm]{gambar/snapedit_1694621260840.png}
%     \caption{Lapisan konvolusi dengan bidang reseptif persegi . \emph{Sumber: \citep{aurélien_géron_2022}}}
%     \label{receptive_fields}
% \end{figure}

% \subsection{\emph{Padding}}
% Dalam persamaan \eqref{ukuran_fitur}, apabila kita aplikasikan pada $240 \times 240$ piksel gambar dengan $10$ layer matriks konvolusi $5 \times 5$, maka akan dihasilkan luaran matriks $200 \times 200$, yang artinya, layer konvolusi mereduksi $30\%$ informasi tentang batas-batas gambar aslinya. Isu demikian dapat diatasi dengan \emph{Padding} \citep{zhang2020dive}.

% \emph{Padding} adalah proses penambahan piksel pada pinggiran gambar asli agar hasil luaran yang diinginkan menjadi lebih besar. Ide awal dari \emph{padding} adalah bahwa piksel di sudut gambar sangat jarang terkena operasi konvolusi sehingga piksel tersebut kurang berkontribusi pada luaran dari operasi konvolusi. Maka ditambahkanlah \emph{padding} pada tepi batas dari matriks masukan agar peran dari piksel tepi digantikan oleh piksel \emph{padding} sehingga luaran yang dihasilkan dapat sama dari masukan. Biasanya digunakan nilai $0$ pada piksel \emph{padding} (disebut dengan istilah (\emph{zero-padding}) \citep{Li_Li_Gao}.

% \begin{figure}[h]
%     \centering
%     \includegraphics[width=10cm]{gambar/padding.png}
%     \caption{Contoh penggunaan \emph{padding} pada gambar asli ukuran $3 \times 3$.\\ \emph{Sumber: dokumen penulis}}
%     \label{padding}
% \end{figure}

% Secara umum, suatu matriks masukan sebesar $n_h \times n_w$ yang dioperasikan pada kernel konvolusi $k_h \times k_w$ dan kemudian ditambah jumlah baris \emph{padding} $p_h$ (di atas dan di bawah) dan jumlah kolom \emph{padding} $p_w$ (di kiri dan di kanan), maka akan menghasilkan luaran dengan bentuk:
% \begin{equation}
%     \left(n_h - k_h + p_h +1\right) \times \left(n_w - k_w + p_w + 1\right)
% \end{equation}

% \subsection{Lapisan Penyatuan (\emph{Pooling Layer})}
% \emph{Pooling layer} atau lapisan penyatuan berfungsi untuk mengurangi ukuran spasial dari fitur hasil operasi konvolusi dan membantu mengurangi \emph{overfitting} \citep{zocca_spacagna_slater_roelants_2017}. Operasi \emph{pooling} yang paling jamak digunakan adalah \emph{max-pooling}, yang membuat kisi (\emph{grid}) pada tiap potongan fitur dan kemudian mengambil nilai piksel yang paling besar pada tiap kisi dan mengabaikan yang lain. \emph{Pooling layer} tidak menambahkan parameter baru, karena hanya mengekstraksi nilai (yang tertinggi, misalnya) tanpa membutuhkan tambahan bobot atau bias.
% \begin{figure}[h]
%     \centering
%     \includesvg[width=5cm]{max_pooling}
%     \caption{Ilustrasi \emph{max pooling layer} dengan ukuran $2 \times 2$ dan \emph{stride} = 2 dikenakan pada fitur dengan ukuran $4 \times 4$. \emph{Sumber: penulis.} }
%     \label{max_pooling}
% \end{figure}

% \emph{Pooling layer} memiliki dua parameter yaitu ukuran sel dan \emph{stride}. Dapat dirumuskan bentuk luaran dari hasil operasi \emph{pooling layer}. $I$ merupakan masukkan (\emph{input}) atau layer fitur hasil konvolusional, $F$ merupakan ukuran \emph{pooling layer} dan $S$ merupakan \emph{stride} dari \emph{pooling layer}. Indeks $w$ dan $h$ merupakan representasi dari lebar (\emph{width}) dan tinggi (\emph{height}) secara berturut-turut.

% \begin{equation}\label{max_pooling}
% \begin{split}
%     O_w = \frac{(I_w - F_w)}{S_w} + 1\\
%     O_h = \frac{(I_h - F_h)}{S_h} + 1
% \end{split}
% \end{equation}

% Operasi \emph{max pooling} bukanlah satu-satunya operasi penyatuan (\emph{pooling}). Ada beberapa operasi penyatuan lain, misalnya \emph{average pool} yang mengambil nilai rata-rata dari piksel yang dikenai oleh \emph{pooling layer}, atau $L^2$ yang merupakan akar pangkat dari jumlah nilai piksel. Dibandingkan dengan operasi penyatuan yang lain, \emph{max pooling} memiliki peforma yang paling baik karena mereduksi \emph{noise} dengan mengabaikan piksel \emph{noisy} yang nilainya kecil \citep{patel_2020}.

% \subsection{Arsitektur CNN}
% Telah ditinjau bahwa CNN umumnya terdiri dari tiga tipe lapisan: konvolusional, \emph{pooling}, \emph{fully conected} (FC). Bentuk paling umum dari arsitektur CNN adalah tumpukkan dari beberapa pasangan konvolusional-ReLU yang diikuti oleh \emph{pooling layer}, dan kemudian pola ini terus berulang hingga gambar input dipadatkan secara spasial menjadi ukuran yang lebih kecil. Lapisan akhir \emph{fully-connected} menghasilkan luaran yang dikehendaki, misalnya kategori untuk klasifikasi ataupun hasil pada regresi. Dengan kata lain, arsitektur paling umum dari CNN adalah sebagai berikut:

% \begin{equation*}
%     \begin{split}
%         MASUKAN \rightarrow &[[KONVOLUSIONAL \rightarrow RELU]*N \rightarrow POOL]\\
%         &*M \rightarrow [FC \rightarrow RELU]*K \rightarrow FC
%     \end{split}
% \end{equation*}

% dengan $*$ mengindikasikan perulangan (repetisi), dan $POOL$ merupakan langkah \emph{pooling} yang opsional. Jumlah perulangan $N, M, K$ masing-masing lebih dari sama dengan $0$. Biasanya, $N \leq 3$ dan $K < 3$ \citep{zhang2020dive}.

% \subsection{U-Net}\label{sub_unet}
% U-Net pertama kali diperkenalkan oleh Olaf Ronneberger, Philipp Fischer, dan Thomas Brox untuk segmentasi biomedis \citep{DBLP:journals/corr/RonnebergerFB15}, dan kemudian mulai banyak digunakan pada komunitas ML-CFD (\emph{Machine Learning - Computation Fluid Dynamics}) \citep{DBLP:journals/corr/abs-2109-13076}.

% Arsitektur ini menggunakan lapisan konvolusi pada semua lapisannya (\emph{fully convolutional layer}). \cite{DBLP:journals/corr/LongSD14} menunjukkan bahwa \emph{fully convolutional layer} pada segmentasi gambar dapat bekerja dengan lebih efisien dan lebih canggih dari arsitektur lainnya tanpa permesinan (\emph{machinery}) lebih lanjut.

% Model U-Net merupakan modifikasi dari model \emph{fully connected layer} yang awalnya dicetuskan oleh \cite{DBLP:journals/corr/LongSD14}. \cite{DBLP:journals/corr/RonnebergerFB15} memodifikasi dan memperpanjang arsitektur yang ada sehingga dapat bekerja dengan beberapa gambar latih dan menghasilkan segmentasi yang lebih presisi (Gambar \ref{unet}).

% \begin{figure}[h]
%     \centering
%     \includegraphics[width=10cm]{gambar/snapedit_1694701272217.png}
%     \caption{Arsitektur U-Net. Tiap kotak biru menunjukkan peta fitur multi kanal. Jumlah kanal ditunjukkan di atas kotak. Ukuran $x-y$ ditunjukkan di tepi kiri bawah kotak. Kotak putih menunjukkan peta fitur yang disalin. Panah menunjukkan operasi-operasi yang terjadi. \emph{Sumber: \citep{DBLP:journals/corr/RonnebergerFB15}}}
%     \label{unet}
% \end{figure}

% Modifikasi yang cukup signifikan terletak pada bagian \emph{upsampling}. Pada bagian ini, masukan yang ukuran spasialnya sudah lebih kecil dari matriks masukan dinaikkan lagi ukurannya (\emph{upsampling}). Pada bagian ini, sejumlah besar kanal fitur memungkinkan jaringan untuk menyebarkan informasi konteks ke lapisan resolusi yang lebih tinggi. Sebagai konsekuensinya, jalur ekspansif (sebelah kanan) kurang lebih simetris dengan jalur kontraksi (sebelah kiri) dan menghasilkan arsitektur berbentuk huruf U. Jaringan ini tidak memiliki \emph{fully connected layer} dan hanya menggunakan bagian yang valid dari setiap luaran konvolusi, yaitu peta segmentasi yang hanya berisi piksel, yang konteks lengkapnya tersedia di dalam gambar masukan. Untuk memprediksi piksel di wilayah perbatasan gambar, konteks yang hilang diekstrapolasi dengan mencerminkan gambar masukan (\emph{crop and copy}).

% Pada arsitektur Gambar \ref{unet}, jalur kontraksi mengikuti arsitektur jaringan konvolusi seperti biasanya. Isinya aplikasi dari lapisan konvolusi $3 \times 3$ (\emph{unpadded convolution}), yang masing-masingnya diikuti oleh fungsi aktivasi ReLU dan operasi \emph{maxpooling} $2 \times 2$ dengan jumlah \emph{stride} 2 untuk \emph{downsampling}. Pada tiap langkah \emph{downsampling}, jumlah kanal fitur diperbanyak 2 kali lipat. Setiap langkah pada jalur ekspansi terdiri dari \emph{upsampling} peta fitur yang diikuti oleh konvolusi $2 \times 2$ (\emph{up-convolution}) yang membagi 2 jumlah kanal fiturnya . Kemudian penggabungan dengan peta fitur terkait yang dipotong dari jalur kontraksi, dan konvolusi $3 \times 3$ yang diikuti oleh fungsi aktivasi ReLU. Penggabungan peta fitur yang terpotong merupakan hal yang perlu dilakukan karena hilangnya batas piksel pada tiap konvolusi \citep{DBLP:journals/corr/RonnebergerFB15}. 

\chapter{Metode Penelitian}

% Pada bagian ini, penulis akan memaparkan metodologi yang digunakan untuk
% memperdalam pemahaman mengenai aplikasi General Purpose GPU (GPGPU) dengan
% menggunakan bahasa pemrograman Julia dalam lingkup komputasi berkinerja tinggi.
% Pemilihan metode ini didasarkan pada kebutuhan untuk mendapatkan pemahaman
% komprehensif mengenai adaptasi GPU, yang secara tradisional ditujukan untuk
% pemrosesan grafis, menjadi sebuah alat yang mampu menjalankan komputasi paralel
% dan intensif. Bahasa pemrograman Julia, dikarenakan desain yang efisien dan
% modern, diidentifikasi sebagai platform yang ideal untuk memfasilitasi
% eksplorasi ini. Selanjutnya, penulis akan menjelaskan secara detail tentang
% rancangan penelitian, termasuk pemilihan dan konfigurasi perangkat keras, serta
% pendekatan dalam pemrograman yang diterapkan untuk mengoptimalkan performa
% komputasi.

Pada bagian ini, penulis akan memaparkan metodologi yang digunakan untuk
memperdalam pemahaman mengenai aplikasi General Purpose GPU (GPGPU) dengan
menggunakan bahasa pemrograman Julia dalam lingkup komputasi berkinerja tinggi.
Pemilihan metode ini didasarkan pada kebutuhan untuk mendapatkan pemahaman
komprehensif mengenai adaptasi GPU, yang secara tradisional ditujukan untuk
pemrosesan grafis, menjadi sebuah alat yang mampu menjalankan komputasi paralel
dan intensif. Bahasa pemrograman Julia, dikarenakan desain yang efisien dan
modern, diidentifikasi sebagai platform yang ideal untuk memfasilitasi
eksplorasi ini. Selanjutnya, penulis akan menjelaskan secara detail tentang
prosedur kerja, serta pemilihan dan konfigurasi perangkat keras.

\section{Bahan}

Data yang digunakan untuk melakukan operasi matriks pada penelitian ini adalah
data acak yang di-\emph{generate} oleh Julia menggunakan fungsi \cw{rand()}. Khusus pada bagian simulasi pencarian nilai eigen, digunakan bentuk matriks berdasarkan Persamaan \ref{eq:schrodinger_finite_matrix}.

\section{Alat}

Perangkat keras yang digunakan dalam penelitian ini adalah berupa komputer
dengan bentuk komputer jinjing (\emph{laptop}) dengan spesifikasi sebagai
berikut:

\begin{table}[H]
	\centering
	\caption{Spesifikasi komputer alat}
	\begin{tabular}{|M{5cm}|M{8cm}|}
		\hline
		\textbf{Komponen} & \textbf{Spesifikasi}                                   \\
		\hline
		Sistem operasi    & Manjaro Linux x86\_64 (Kernel 6.1.55-1-MANJARO)        \\
		\hline
		CPU               & Intel i7-10510U, 4.9 GHz, 4 Cores, 8 Logical Processor \\
		\hline
		GPU               & NVIDIA GeForce MX2500, 2GB                             \\
		\hline
		RAM               & 20 GB                                                  \\
		\hline
		Penyimpanan       & Solid State Drive 480 GB                               \\
		\hline
	\end{tabular}
\end{table}

Kemudian, dalam menjalankan penelitian, penulis menggunakan berbagai perangkat
lunak untuk melakukan \emph{benchmarking}. Adapun perangkat-perangkat lunak
tersebut adalah:
\begin{itemize}
	\item Visual Studio Code: sebagai \emph{code editor} untuk Julia

	\item Jupyter Notebook: sebagai \emph{execution cell} untuk Julia

	\item CUDA.jl: sebagai pustaka utama untuk menjalankan paralelisasi pada
	      perangkat GPU NVIDIA

	\item Bahasa pemrograman Julia: untuk menjalankan metode fisika komputasi

	      % \item Bahasa pemgoraman Python: untuk menjalankan metode fisika komputasi
\end{itemize}

% \section{Bahan}
% Data latih dalam penelitian ini akan menggunakan data sumber distribusi $\rho$
% dari dataset acak yang memiliki rentang [-1,1] yang dibuat menggunakan program PlasmaNet
% \citep{cheng_illarramendi_bauerheim_cuenot_2021}. Dan kemudian dihitung potensial
% listrik $\phi$ menggunakan \emph{solver} Gauss-Seidel untuk plasma berbasis C++
% \citep{lubos_brieda_2019}.

% \section{Alat}
% Perangkat keras yang digunakan peneliti dalam penelitian ini adalah berupa komputer
% dengan bentuk komputer jinjing (\emph{laptop}) dengan spesifikasi sebagai
% berikut:

% \begin{table}[ht]
%   \centering
%   \caption{Spesifikasi komputer alat}
%   \begin{tabular}{|M{3cm}|M{4cm}|}
%     \hline
%     \textbf{Komponen} & \textbf{Spesifikasi}                                                            \\
%     \hline
%     Sistem operasi    & Windows 11 Home                                                                 \\
%     \hline
%     CPU               & AMD Ryzen 5 5600U with Radeon Graphics, 2.30 GHz, 6 Cores, 12 Logical Processor \\
%     \hline
%     GPU               & NVIDIA GeForce RTX 3050, 4GB                                                    \\
%     \hline
%     RAM               & 16 GB                                                                           \\
%     \hline
%     Penyimpanan       & Solid State Drive 512 GB                                                        \\
%     \hline
%   \end{tabular}
% \end{table}

% Kemudian, dalam menjalankan penelitian, penulis menggunakan berbagai perangkat lunak
% untuk membuat data, mengolah data, dan melatih data. Adapun perangkat-perangkat
% lunak tersebut adalah:
% \begin{itemize}
%   \item Visual Studio Code: sebagai pengolah kode

%   \item Google Colab Pro dengan GPU T4: untuk melatih model dengan \emph{graphic
%     processing unit} (GPU)

%   \item Bahasa pemrograman C++: untuk membangun \emph{solver} Gauss-Seidel

%   \item Bahasa pemrograman Python: untuk membangun model dan melatih model \emph{machine
%     learning}
% \end{itemize}

\section{Prosedur Kerja dan Pengambilan Data}

\begin{figure}[h]
	\centering
	\includegraphics[width=3cm, scale=1]{schema/metode.drawio.png}
	\caption{Diagram alir penelitian}
	\label{img:methods}
\end{figure}

Secara garis besar, penelitian ini mencari waktu kecepatan eksekusi pada sistem
series (CPU) dan sistem paralel (GPU). Pada masing - masing sistem, alur
penelitian berawal dari pembuatan kode pada masing masing operasi matriks
setelah itu dilanjutkan dengan pengeksekusian kode di skenario - skenario
tertentu pada masing - masing operasi matriks, kemudian setiap hasil dari
eksekusi tersebut akan dianalisis perbedaan kecepatan eksekusi nya. Pada Gambar
\ref{img:methods} disajikan alur penelitian dengan model diagram alir atau
\emph{flowchart}.

Sistem series dan sistem paralel dituliskan dalam Bahasa Julia. Penggunaan
bahasa yang sama ini bertujuan untuk mengurangi hambatan - hambatan external
seperti hambatan karena perbedaan compiler dari suatu bahasa pemrograman.

Proses komputasi paralel pada GPU menggunakan bantuan pustaka bernama
CUDA.jl. CUDA.jl ini merupakan pustaka untuk melakukan komputasi pada
sistem paralel GPU, khusus nya untuk Vendor GPU NVidia. Secara sederhana, cara
kerja CUDA.jl ini adalah dengan mengirimkan data array dari CPU ke vertex
shader di GPU, kemudian dilakukan kalkulasi yang hasilnya akan dikirimkan ke
bagian fragment shader di GPU. Proses kalkulasi yang dijalankan di GPU disebut
dengan \textbf{kernel}. Kernel ini pada dasarnya hanya berupa \emph{function}
Julia yang dijalankan di GPU.

\subsection{Proses Eksekusi Operasi Matriks}

Proses eksekusi diawali dengan penghasilan data sample yang mana cara
mendapatkan data sample ini sudah dijelaskan pada bagian bahan, yakni dengan
menggunakan \cw{rand()} yang merupakan fungsi \emph{random generator} dari
Julia. Setelah proses penghasilan data sudah selesai, langkah selanjutnya
adalah melakukan proses operasi matriks menggunakan CPU kemudian dilanjutkan
dengan melakukan proses operasi matriks menggunakan GPU. Proses penghasilan
data hingga proses kalkulasi menggunakan CPU dan GPU dilakukan sebanyak sepuluh
kali untuk memastikan bahwa dalam kalkulasi nya tidak terdapat hambatan
eksternal pada GPU dan CPU. Hambatan yang dimaksud antara lain seperti hambatan
kapasitas DRAM maupun VRAM yang sedang sedikit dan hambatan keterbatasan
software Jupyter dalam mengeksekusi operasi yang ada. Pada Gambar
\ref{img:methods_execution}, telah disajikan diagram alir dalam proses eksekusi
operasi matriks.

\begin{figure}[H]
	\centering
	\includegraphics[width=6cm, scale=1]{schema/langkah-2.drawio.png}
	\caption{Diagram alir dalam Proses Eksekusi Setiap Metode}
	\label{img:methods_execution}
\end{figure}

Terdapat 6 operasi berbasis matriks yang dijalankan pada sistem series dan
sistem paralel sebagai berikut

\begin{enumerate}
	\item Operasi Penjumlahan Matriks
	\item Operasi Pengurangan Matriks
	\item Operasi Perkalian antara Skalar dengan Matriks
	\item Operasi Perkalian antar Matriks
	\item Inverse Matriks
	\item Pencarian Nilai Eigen
	      % \item Pencarian Nilai Eigen dari Matriks berdasarkan Persamaan  Persamaan \ref{eq:schrodinger_finite_matrix}
\end{enumerate}

\noindent
Operasi penjumlahan hingga operasi inverse matriks menggunakan data elemen bernilai acak. Namun pada operasi pencarian nilai eigen, digunakan data matriks berdasarkan Persamaan \ref{eq:schrodinger_finite_matrix}.

Pada langkah terakhir dari Gambar \ref{img:methods_execution} terdapat
pengecekan variasi percobaan. Setiap operasi matriks terdapat 7 variasi
pengukuran berdasarkan ukuran matriks nya. Berikut merupakan 7 variasi
pengukuran nya:

\begin{itemize}
	\item Variasi 1: Matriks berukuran $10 \times 10$
	\item Variasi 2: Matriks berukuran $50 \times 50$
	\item Variasi 3: Matriks berukuran $10^2 \times 10^2$
	\item Variasi 4: Matriks berukuran $500 \times 500$
	\item Variasi 5: Matriks berukuran $10^3 \times 10^3$
	\item Variasi 6: Matriks berukuran $5000 \times 5000$
	\item Variasi 7: Matriks berukuran $10^4 \times 10^4$ (Jika memungkinkan)
\end{itemize}

\noindent
Khusus untuk variasi ke-7, terkadang terdapat keterbatasan memory (RAM/VRAM) sehingga tidak memungkinkan untuk dilakukan operasi dengan besar matriks $10^4 \times 10^4$. Untuk itu, pengkaji akan menurunkan ukuran matriks hingga cukup dialokasikan ke memory, sehingga operasi bisa dijalankan.

\subsection{Analisis Data dan Penampilan Grafik}

Proses analisis data meliputi pencatatan hasil eksekusi, kemudian dilanjutkan
dengan pengumpulan semua data yang telah dicatat, kemudian yang terakhir adalah
pencarian nilai rata-rata dari sepuluh kali pengulangan. Pada Gambar
\ref{img:methods_analysist} telah disajikan diagram alir dari proses analisis
ini.

\begin{figure}[h]
	\centering
	\includegraphics[width=4cm, scale=1]{schema/langkah-3.drawio.png}
	\caption{Diagram alir dalam Proses Analisis Data}
	\label{img:methods_analysist}
\end{figure}

Setelah proses analisis data selesai, langkah selanjutnya adalah penampilan
grafik berdasarkan data dari hasil analisis data. Jenis grafik yang dipilih
adalah grafik batang grup atau \emph{grouped bar chart}. Alasan dipilihnya
jenis grafik ini dikarenakan penelitian dari
\cite{hunoldBenchmarkingJuliaCommunication2020} juga menggunakan grafik batang
grup untuk menampilkan perbandingan kecepatan eksekusi Bahasa Julia dengan
Bahasa C yang mana telah dijelaskan pada Bagian Tinjauan Pustaka. Hasil grafik
akan ditampilkan dengan sumbu-x untuk perubahan variasi ukuran matriks,
sedangkan sumbu-y untuk waktu kecepatan eksekusi pada setiap operasi.

% ================================================

% \label{data_latih} Dalam konteks pemelajaran mendalam (\emph{deep learning}),
% data latih berarti pasangan data antara data masukan dan data target. Dan data hasil
% prediksi adalah data uji dan data hasil prediksi. Tahap pertama penelitian ini
% adalah pembuatan data latih. Untuk data masukan, digunakan data distribusi acak
% (sisi kanan pada Persamaan \eqref{poisson umum}) yang dibuat menggunakan program
% PlasmaNet yang dapat dikontrol skala spasialnya. Data distribusi acak memiliki
% nilai yang membentang pada rentang [-1, 1] secara kontinyu. Data ini dibuat pada
% kisi dengan resolusi kasar dengan ukuran $n_{kasar}= [n_{p}/c]$. Dengan $n_{p}$
% adalah jumlah piksel pada tiap sumbu dan $c$ adalah ukuran filter. Kemudian interpolasi
% bikubik membuat bidang random dengan ukuran struktur terkontrol pada kisi target
% \citep{cheng_illarramendi_bauerheim_cuenot_2021}, dengan $c$ adalah ukuran struktur
% minimum. Data distribusi yang diambil pada penelitian ini adalah sebanyak 20.200
% data dengan domain $100 \times 100$.

% \begin{figure}[h!]
%   \centering
%   \includegraphics[width=8cm]{gambar/random.png}
%   \caption{Contoh nilai distribusi acak dalam $6 \times 6$ pada kisi kasar (kiri)
%   yang diinterpolasikan pada kisi halus $101 \times 101$ (kanan) \emph{Sumber: \citep{cheng_illarramendi_bauerheim_cuenot_2021}}}
% \end{figure}

% Tahapan pertama yang dilakukan dalam pengambilan data sekaligus dalam penelitian
% ini adalah pembuatan \emph{solver} Gauss-Seidel untuk menghasilkan pasangan
% dataset latih (\emph{training dataset}). Hal pertama yang harus diperhatikan
% dalam pembuatan program pemecah (\emph{solver}) masalah Poisson adalah domain fisis
% yang digunakan. Seperti yang telah disebutkan pada Bab \ref{tipus}, konteks pemecahan
% masalah ini adalah terletak pada koordinat silinder Gambar \ref{domain_silinder}.
% Untuk menjaga karakter kinetik model sepenuhnya, perlu dilakukan pengurangan dimensi
% sistem, membatasi domain menjadi dua dimensi ($r,z$) mengabaikan variasi azimut
% dari besaran yang terlibat (simulasi simetri aksial) \citep{f_taccogna_longo_capitelli_schneider_2005}
% dengan $r \neq 0$. Pemilihan domain ini dikarenakan ada banyak pemanfaatan yang
% dapat digunakan dalam implementasi fisis, salah satunya adalah pendorong Hall (\emph{Hall
% thruster}). Untuk itu, disesuaikanlah domain silinder ini dengan model pendorong
% Hall, adapun model domain pendorong Hall yang digunakan dalam penelitian ini adalah
% SPT-100 karena selain memiliki sejarah keberhasilan yang banyak \citep{braga_miranda_2019},
% perangkat ini juga banyak digunakan sebagai patokan atau percontohan dalam beberapa
% penelitian \citep{Shiferaw2013, f_taccogna_longo_capitelli_schneider_2005, braga_miranda_2019, boeuf_2017}.

% \begin{figure}[h!]
%   \centering
%   \includegraphics[width=5cm]{gambar/silinder.png}
%   \caption{Tampang lintang koordinat silinder. Penelitian ini menghitung potensial
%   listrik pada koordinat silinder dengan pendekatan simetri aksial, yaitu pada
%   sumbu $r-z$.\emph{ Sumber: \citep{Shiferaw2013} dengan penyesuaian oleh Penulis}}
%   \label{domain_silinder}
% \end{figure}

% Disadur dari \cite{braga_miranda_2019} dan \cite{f_taccogna_longo_capitelli_schneider_2005},
% spesifikasi teknis dan ukuran domain dari model SPT-100 ditampilkan dalam Tabel
% \ref{spt_100}. Domain komputasi numerik pada penelitian ini diilustrasikan dalam
% model 2 dimensi dari SPT-100 yang ditampilkan pada Gambar \ref{spt_100_2d}.

% \begin{table}[h!]
%   \centering
%   \caption{Informasi domain fisis dan teknis pendorong Hall SPT-100 }
%   \label{spt_100}
%   \begin{tabular}{ll}
%     \hline
%     \textbf{Dimensi}             & \textbf{Ukuran} \\
%     \hline
%     Panjang kanal (m)            & 0,025           \\
%     \hline
%     Lebar kanal (m)              & 0,015           \\
%     \hline
%     Panjang kanal keluar (m)     & 0,01            \\
%     \hline
%     Jari-jari silinder dalam (m) & 0,035           \\
%     \hline
%     Jari-jari silinder luar (m)  & 0,05            \\
%     \hline
%     Dorongan (N)                 & 0,08            \\
%     \hline
%   \end{tabular}
% \end{table}

% \begin{figure}[h!]
%   \centering
%   \includegraphics[width=5cm]{gambar/spt_100_2d.png}
%   \caption{Model 2 dimensi dari kanal SPT-100 sebagai domain komputasi numerik. Lebar
%   kanal dimulai dari 0,035 - 0,050 m dan panjang kanal dimulai dari 0,00 - 0,025
%   m. \emph{ Sumber: \citep{braga_miranda_2019}}}
%   \label{spt_100_2d}
% \end{figure}

% \subsubsection{Pengambilan Data Distribusi Pada Program PlasmaNet}

% Data distribusi muatan $\rho$ yang akan digunakan untuk menghitung potensial
% listrik pada pemecah Gauss-Seidel diambil menggunakan program PlasmaNet \citep{cheng_illarramendi_bauerheim_cuenot_2021}.
% Program ini menyediakan pasangan data distribusi partikel secara acak maupun
% distribusi partikel dengan pola sinusoidal dan potensial listriknya. Namun,
% pasangan data tersebut tidak dapat digunakan karena potensial listrik yang
% digunakan mengikuti syarat batas tertentu seperti pada domain komputasi yang
% disebutkan sebelumnya, sehingga hanya akan digunakan data distribusinya ($\rho$)
% saja.

% Dalam pembentukan data distribusi acak, perumusan matematis telah dijelaskan
% sebelumnya pada Subbagian \ref{data_latih}. Hal yang harus diperhatikan untuk pembuatan
% data distribusi acak ini adalah mengenai syarat awal domain serta ukuran spasial
% domain kasar dan halus. Pertama, akan ditinjau mengenai syarat awal yang digunakan
% untuk membentuk distribusi acak.

% \begin{mypythoncode}
%   [Syarat awal pembuatan data distribusi acak di PlasmaNet] xmin: 0.035 xmax: 0.050
%   nnx: 100 ymin: 0.0 ymax: 0.050 nny: 100
% \end{mypythoncode}

% Kode cuplikan Kode 1 dapat dibandingkan dengan Gambar \ref{spt_100_2d}, \texttt{xmin}
% dan \texttt{xmax} merupakan lebar kanal dan \texttt{ymin} dan \texttt{ymax} merupakan
% panjang kanal. Pada bagian ini dilakukan peniruan domain SPT-100 untuk
% distribusi muatan (Gambar \ref{ukuran spt 100}). Kemudian, \texttt{nnx} dan \texttt{nny}
% adalah jumlah piksel untuk tiap sumbu.

% \begin{figure}[h!]
%   \centering
%   \includegraphics[width=10cm]{gambar/domain spt100 ukuran.png}
%   \caption{Ukuran kanal SPT-100. \emph{ Sumber: Penulis}}
%   \label{ukuran spt 100}
% \end{figure}

% Tahapan selanjutnya dalam pembuatan set data acak adalah memilih parameter $c$ yang
% akan menentukan ukuran struktur minimum pada ukuran domain kasar. Pada
% penelitian ini, akan digunakan domain $100 \times 100$ dan dipilih $c = 16$. Hal
% ini akan membuat domain medan acak dengan ukuran $100/8 \times 100/8 \approx 12$
% dan kemudian akan diinterpolasikan kembali ke ukuran asli $100 \times 100$.

% Setelah didapatkan kumpulan data acak, data tersebut tidak dapat langsung digunakan
% karena setelah dilakukan investigasi singkat pada data-data yang ada, nilai-nilai
% data pada set data distribusi yang dihasilkan sangat kecil (misalnya, banyak
% medan dengan nilai piksel di orde $10^{-34}$) sehingga menyebabkan kesulitan pada
% perhitungan oleh komputer. Sehingga tahapan selanjutnya adalah peningkatan (\textit{scale-up}).
% Upaya peningkatan nilai $\rho$ untuk seluruh jumlah dataset (\texttt{len(rho)}) dilakukan
% dengan cara sebagai berikut:

% \begin{mypythoncode}
%   [Kode cuplikan peningkatan nilai rho] a = len(rho) for i in range (a): rho_max
%   = np.max(rho[i]) rho_min = abs(np.min(rho[i]))

%   if rho_max > rho_min: c = 1/rho_max else: c = 1/rho_min

%   rho[i] = rho[i]*c
% \end{mypythoncode}

% Pada kode tersebut, dicari nilai ekstrim mutlak tertinggi pada tiap medan, dan
% kemudian nilai ekstrim mutlak tersebut digunakan sebagai nilai pembagi di masing-masing
% domain.

% \subsubsection{Syarat Awal dari \textit{Solver} Gauss-Seidel}

% Setelah data distribusi partikel siap, akan dilanjutkan dengan pemecahan masalah
% menggunakan metode Gauss-Seidel. Pada subbab ini dan subbab selanjutnya, akan
% dibahas mengenai syarat awal dan algoritma dari \textit{solver} Gauss-Seidel yang
% digunakan.

% Untuk membangun \textit{solver} perhitungan potensial listrik menggunakan metode
% Gauss-Seidel ini, digunakan bahasa pemrograman C++. Bahasa pemrograman C++
% dipilih pertama-tama karena bahasa ini berkali-kali lipat lebih cepat
% dibandingkan Python, walaupun Python memiliki pustaka (\textit{library}) yang
% cukup bagus dan dapat mengoptimasi kerja komputasi, semisal Numpy \citep{lubos_brieda_2019}.
% Selain itu, Python memiliki kekurangan dalam tipe variabel yang dapat memicu
% \textit{bugs}, terlebih khusus saat pembedaan variabel \textit{integer} dan \textit{floating
% point}. Dari segi pustaka, ada banyak pustaka saintifik yang diimplementasikan
% menggunakan C++ tanpa pembungkus (\textit{wrappers}) pihak ketiga.

% Syarat awal pada \textit{solver} ini didefinisikan dalam sebuah \textit{header
% file}. Dalam \textit{header file} ini berisi konstanta yang digunakan dan juga tebakan
% awal Gauss-Seidel. Pada penelitian ini digunakan satu konstanta yaitu
% $\epsilon_{0} = 8.8541878 \times 10^{-12}$ \citep{lubos_brieda_2019}. Dan untuk
% tebakan awal, digunakan nilai tebakan awal = 0 untuk seluruh piksel pada matriks
% tebakan awal.

% \subsubsection{Algoritma \textit{Solver} Gauss-Seidel}

% Pada subbagian ini akan dibahas mengenai implementasi dari Subbagian
% \ref{gauss_seidel_silinder} pada \textit{solver} Gauss-Seidel menggunakan bahasa
% pemrograman C++. Secara umum, \textit{solver} ini akan menjalankan 2 file. Yang pertama
% adalah file untuk mendefinisikan algoritma dari Gauss-Seidel sekaligus untuk menuliskan
% hasil (dalam penelitian ini file tersebut diberi nama \texttt{Potential.cpp} (Lampiran
% \ref{kode_potential_cpp})) dan file yang kedua adalah untuk memberikan syarat
% awal atau konteks besaran domain dan juga pada file ini terjadi pemanggilan
% fungsi-fungsi yang didefinisikan pada \texttt{Potential.cpp} beserta pengisian nilainya.
% Nama file ini pada penelitian ini adalah \texttt{main.cpp} (Lampiran
% \ref{kode_main_cpp}).

% Tahap paling awal pada \texttt{Potential.cpp} adalah pemanggilan \textit{header
% file} dan pustaka yang dibutuhkan. Setelah itu, dilakukan penginisialisasian menggunakan
% \textit{constructor} \texttt{Solver} untuk matriks masukan dan matriks luaran.
% Kemudian, akan didefinisikan besaran-besaran lain yang merupakan jarak domain pada
% fungsi \texttt{setextents} dan untuk jumlah iterasi maksimal dan toleransi ralat
% diatur pada fungsi \texttt{setParam}.

% Setelah semua kondisi diatur, masuk pada perhitungan iterasi pada fungsi \texttt{writeSolveGS}.
% Di sini, argumen yang digunakan adalah \texttt{filename}, \texttt{reshape\_\texttt{\-}rows},
% \texttt{reshape\_cols}, yang nilai argumennya semuanya terletak pada \texttt{main.cpp}.
% Berikut adalah \textit{pseudocode} dari algoritma implementasi perhitungan Gauss-Seidel:
% \begin{lstlisting}[breaklines=true, breakatwhitespace=true]
% DEFINISIKAN variabel: idz, idr, idz2 (idz pangkat dua), idr2 (idr pangkat dua), crz
% SET L2 ke 0
% SET converged ke False

% FOR iterasi < nilai iterasi maksimal:
%     iterasi ditambah 1
%     FOR i < jumlah baris keseluruhan data:
%         i ditambah 1
%         FOR j < jumlah kolom:
%             j ditambah 1
%             IF i = 0 ATAU i kelipatan 100: #syarat batas paling atas
%                 CONTINUE
%             ELSE IF i modulo 100 = 99: #syarat batas paling bawah
%                 phi(i,j) = phi(i-1, j)
%             ELSE IF j = 0: #syarat batas paling kiri
%                 CONTINUE
%             ELSE IF j = 99: #syarat batas paling kanan
%                 CONTINUE
%             ELSE: #selain di syarat batas (interiornya)
%                 hitung crj
%                 hitung phi baru menggunakan rho dari file yang ada
%                 lanjutkan dengan SOR

%     IF iterasi kelipatan 100:
%         hitung L2 antara 2 iterasi

%         IF L2 < nilai toleransi
%             perhitungan konvergen atau converged = True
%             BREAK

%     IF tidak konvergen:
%         tampilkan teks: "Gauss seidel standar gagal konvergen, L2=", masukkan nilai L2

%     #penulisan hasil phi pada file csv
%     FOR i < jumlah baris keseluruhan data:
%         i ditambah 1
%         FOR j < jumlah kolom:
%             j ditambah 1
%             IF j pada indeks kolom terakhir:
%                 tulis phi(i,j) diikuti pindah baris
%             ELSE
%                 tulis phi(i,j) diikuti tanda koma (,)
% \end{lstlisting}

% Variabel-variabel yang didefinisikan di awal merupakan variabel yang akan
% digunakan pada perhitungan interior domain, yang detail perhitungannya dapat dilihat
% pada kode sumber. Untuk syarat batas pada atas, kiri, dan kanan, diimplementasikan
% syarat batas Dirichlet nol, sedangkan pada batas paling bawah diterapkan syarat
% batas Neumann. Pada perhitungan Gauss-Seidel di interior menggunakan implementasi
% penyelesaian persamaan Poisson menggunakanm metode Gauss-Seidel seperti pada Subbagian
% \ref{gauss_seidel_silinder} dan parameter SOR 1,4 untuk mempercepat konvergensi seperti
% pada \cite{lubos_brieda_2019}.

% Setiap 100 iterasi, dihitung nilai metrik L2 untuk menentukan nilai ralatnya. Nilai
% L2 ini merupakan nilai rerata ralat dari seluruh piksel iterasi tersebut dengan iterasi
% sebelumnya. Apabila nilai setelah iterasi ke-100 nilai L2 sudah di bawah nilai
% toleransi, maka iterasi pada medan tersebut dihentikan. Namun apabila belum,iterasi
% dilanjutkan pada iterasi kelipatan 100 lainnya dan diperiksa lagi apakah sudah
% di bawah nilai toleransi. Apabila nilai L2 pada medan tersebut belum sampai di bawah
% nilai toleransi sampai pada jumlah iterasi maksimal, maka iterasi pada medan
% tersebut dianggap gagal dan program akan mengeluarkan pesan galat "Gauss seidel standar
% gagal konvergen, L2=" diikuti dengan nilai L2 terakhir.

% Untuk memanggil fungsi-fungsi yang ada pada program Lampiran
% \ref{kode_potential_cpp}, digunakan program lain, yaitu program pada Lampiran
% \ref{kode_main_cpp}. Dalam Lampiran \ref{kode_main_cpp} ditampilkan contoh kasus
% saat program menghitung 2.000 data. Variabel \texttt{nz} adalah jumlah baris keseluruhan
% data, sehingga nilainya $2.000 \times 100 = 200.000$, dan variabel \texttt{nr}
% adalah jumlah kolom domain, yaitu 100.

% Variabel \texttt{totalData} merupakan jumlah seluruh piksel yang dilibatkan dalam
% perhitungan ini. File \textit{comma separated value} (CSV) dari data distribusi yang
% digunakan dipanggil pada variabel \texttt{rho} yang kemudian diubah bentuknya
% menjadi $nz \times nr$. Untuk parameter domain SPT-100 (Tabel \ref{spt_100}) didefinisikan
% pada pemanggilan fungsi \texttt{setextents}. Jumlah iterasi maksimal yang digunakan
% pada perhitungan ini sebesar 100.000 dan nilai toleransi yang diizinkan adalah 0,01.
% Kedua parameter tersebut didefinisikan pada fungsi \texttt{setParam}. Tidak ada referensi
% baku mengenai nilai maksimal iterasi dan nilai ralat yang digunakan. Pada
% \cite{lubos_brieda_2019}, untuk ukuran domain $21 \times 21 \times 21$ sebanyak 1
% data, digunakan nilai iterasi maksimal 5000 dan nilai toleransi 0. Berdasarkan
% acuan ini, dicoba nilai iterasi maksimal dan nilai toleransi ralat yang wajar,
% dapat diandalkan, serta realtif cepat untuk domain 100 $\times$ 100 dengan 2.000
% data (digunakan jumlah 2.000 data dalam sekali perhitungan untuk mencegah agar
% jangan sampai terjadi kerusakan pada perangkat keras karena durasi \textit{runtime}
% yang terlalu lama dan penggunaan memori yang terlalu besar), maka digunakan jumlah
% iterasi maksimal sebanyak 100.000 dan toleransi ralat sebesar 0,01.

% Selain data latih, untuk melatih model yang ada, dibutuhkan data validasi (\emph{validation
% data}) untuk melihat peforma dari model tanpa bias dari data latih \citep{jason_brownlee_2017}.
% Dalam penelitian ini, data validasi diambil sebanyak 25\% dari total data latih.

% \subsection{Pengambilan Data Uji}
% Setelah model didapatkan untuk dilatih, kemudian model diuji dengan data lain
% yang tidak pernah dilihat sebelumnya oleh model. Data tersebut sebagian merupakan
% kumpulan data dengan distribusi acak dan sebagian merupakan data dengan distribusi
% partikel mengikuti aturan Gaussian (Persamaan \eqref{gaussian}):

% \begin{equation}
%   \label{gaussian}f(x,y) = A \enspace exp \left(-\left(\frac{(x-x_{0})^{2}}{2
%   \sigma^{2}_{X}}+ \frac{(y-y_{0})^{2}}{2 \sigma^{2}_{Y}}\right)\right)
% \end{equation}

% Dengan $A$ adalah amplitudo, $x_{0}$ dan $y_{0}$ merupakan koordinat titik pusat,
% dan $\sigma_{x}$, $\sigma_{y}$ merupakan standar deviasi pada sumbu $x$ dan $y$.

% \subsection{Pemrosesan data}
% Salah satu keuntungan terbesar penggunaan jaringan saraf buatan adalah, tidak perlu
% dilakukan \emph{feature engineering}. Lapisan tersembunyi (\emph{hidden layer})
% yang akan mempelajar fitur-fitur dari data yang diberikan \citep{denny2015}.
% Sehingga tidak diperlukan pemrosesan data yang membutuhkan usaha besar.

% Pemrosesan data yang dilakukan adalah pengubahan dimensi bentuk dan normalisasi.
% Perubahan bentuk dalam hal ini adalah pengubahan data-data menjadi masing-masing
% bentuk matriks persegi dalam satu kolom. Kemudian dilakukan normalisasi untuk
% mentransformasi fitur menjadi skala yang mirip, sehingga meningkatkan peforma dan
% stabilitas pelatihan model \citep{google_2022}. Normalisasi yang dilakukan adalah
% penskalaan (\emph{scaling}). Penskalaan berarti mengubah nilai titik data dari rentang
% aslinya ke rentang standar (biasanya 0 sampai 1 atau -1 sampai 1).

% \subsection{Pendefinisian arsitektur}

% Pendefinisian arsitektur dilakukan sesuai dengan bentuk data yang masuk dan bentuk
% data luaran yang dikehendaki. Penambahan regulator, \emph{pooling layer}, dan \emph{padding}
% juga dilakukan di sini. Tidak ada patokan yang baku mengenai bentuk arsitektur dan
% \emph{hyperparameter} yang akan digunakan. Penentuannya akan dilakukan dengan metode
% \emph{trial and error}.

% \subsection{Pelatihan model}
% Model atau arsitektur yang sudah ada kemudian dilatih atau di-\emph{fitting} dengan
% data latih yang disediakan dan divalidasi dengan data validasi yang terpisah dari
% data latih. Pada proses pelatihan ini menggunakan fungsi ralat MSE \ref{mse}, dengan
% \emph{batch size} 32, 200 \emph{epoch}. Kemudian dengan menggunakan pengoptimasi
% ADAM \citep{kingma2017adam} dengan laju pemelajaran awal (\emph{inital learning
% rate}) = 0.001 dan diterapkan penjadwalan laju pemelajaran dengan langkah peluruhan
% (\emph{decay steps}) = 10000 dan laju peluruhan (\emph{decay rate}) = 0.9.

% Sembari dilatih, semua model dari tiap \emph{epoch} disimpan dalam bentuk file h5
% dan kemudian dipilih model yang paling baik dengan cara melihat kurva pemelajaran
% yang muncul setelah model terlatih seutuhnya. Kriteria model yang paling baik
% adalah model dengan \emph{validation loss} paling rendah dan berada di bawah atau
% sama atau hampir sama dari \emph{training validatrion} (\emph{goodfit}).

% \subsection{Evaluasi dan pengujian}
% Masuk ke tahap evaluasi dan pengujian, pertama akan dimuat data uji yang sudah
% dibuat sebelumnya. Kemudian, muat model \emph{epoch} ke-100, \emph{epoch} ke-200,
% dan \emph{epoch} dengan model paling baik. Kemudian lakukan prediksi dengan data
% distribusi $\rho_{uji}$ untuk tiap model.

% Selanjutnya, akan dibandingkan $\phi_{pred}$ yang merupakan hasil prediksi model
% dan $\phi_{uji}$. Sebelum dilakukan pembandingan, data $\phi_{uji}$ dilakukan normalisasi
% terlebih dahulu. Uji yang pertama adalah uji visual pada matriks yang diuji. Di sini
% akan dilihat secara visual seberapa mirip antara $\phi_{uji}$ dan $\phi_{pred}$.

% Selanjutnya akan dipilih beberapa piksel secara acak untuk diketahui nilainya,
% baik pada $\phi_{uji}$ maupun $\phi_{pred}$. Data piksel tersebut kemudian dibandingkan
% dengan cara dihitung MSE dan MAPE-nya.

% Uji selanjutnya adalah pengujian menggunakan MAPE. Akan dipilih nilai-nilai yang
% jauh dari nol. Kemudian dihitung MAPE dari nilai tersebut. Dari pengujian
% tersebut ditentukan nilai MAPE tertinggi, MAPE terendah, dan rata-rata dari
% keseluruhan MAPE.

% =========================

% Please add the following required packages to your document preamble:
% \usepackage{lscape}
% \usepackage{longtable}

% Note: It may be necessary to compile the document several times to get a multi-page table to line up properly

% \begin{landscape}
%   \subsection{Rencana Kegiatan Penelitian}
%   \begin{longtable}[c]{|l|l|l|l|l|l|l|l|l|l|l|l|}
%     \caption{Tabel Rencana Kegiatan November 2023 -- Mei 2024}
%     \label{table:plan}                                                                                                                                                                                 \\
%     \hline
%     No.                                                                                                                   & Kegiatan & November  & Desember & Januari & Februari & Maret & April & Mei \\ \hline
%     \endfirsthead
%     %
%     \multicolumn{12}{c}%
%     {{\bfseries Table \thetable\ continued from previous page}}                                                                                                                                        \\
%     \hline
%     No.                                                                                                                   & Kegiatan & Novermber & Desember & Januari & Februari & Maret & April & Mei \\ \hline
%     \endhead
%     %
%     1                                                                                                                     &
%     \begin{tabular}[c]{@{}l@{}}Eksplorasi Bahasa Julia dan pustaka nya\end{tabular}                                       &
%     \checkmark                                                                                                            &
%     \checkmark                                                                                                            &
%                                                                                                                           &
%                                                                                                                           &
%                                                                                                                           &
%                                                                                                                           &
%     \\ \hline
%     %
%     2                                                                                                                     &
%     \begin{tabular}[c]{@{}l@{}}Penulisan Proposal\end{tabular}                                                            &
%     \checkmark                                                                                                            &
%     \checkmark                                                                                                            &
%     \checkmark                                                                                                            &
%     \checkmark                                                                                                            &
%                                                                                                                           &
%                                                                                                                           &
%     \\ \hline
%     %
%     3                                                                                                                     &
%     \begin{tabular}[c]{@{}l@{}}Penulisan Kode dan Pengeksekusian \\ Operasi Penjumlahan Matriks\end{tabular}              &
%                                                                                                                           &
%                                                                                                                           &
%                                                                                                                           &
%     \checkmark                                                                                                            &
%                                                                                                                           &
%                                                                                                                           &
%     \\ \hline
%     %
%     4                                                                                                                     &
%     \begin{tabular}[c]{@{}l@{}}Penulisan Kode dan Pengeksekusian \\ Operasi Pengurangan Matriks\end{tabular}              &
%                                                                                                                           &
%                                                                                                                           &
%                                                                                                                           &
%     \checkmark                                                                                                            &
%                                                                                                                           &
%                                                                                                                           &
%     \\ \hline
%     %
%     5                                                                                                                     &
%     \begin{tabular}[c]{@{}l@{}}Penulisan Kode dan Pengeksekusian \\ Operasi Perkalian Matriks dengan Skalar\end{tabular}  &
%                                                                                                                           &
%                                                                                                                           &
%                                                                                                                           &
%     \checkmark                                                                                                            &
%     \checkmark                                                                                                            &
%                                                                                                                           &
%     \\ \hline
%     %
%     6                                                                                                                     &
%     \begin{tabular}[c]{@{}l@{}}Penulisan Kode dan Pengeksekusian \\ Operasi Perkalian Matriks dengan Matriks\end{tabular} &
%                                                                                                                           &
%                                                                                                                           &
%                                                                                                                           &
%     \checkmark                                                                                                            &
%     \checkmark                                                                                                            &
%                                                                                                                           &
%     \\ \hline
%     %
%     7                                                                                                                     &
%     \begin{tabular}[c]{@{}l@{}}Penulisan Kode dan Pengeksekusian \\ Operasi Inverse Matriks\end{tabular}                  &
%                                                                                                                           &
%                                                                                                                           &
%                                                                                                                           &
%                                                                                                                           &
%     \checkmark                                                                                                            &
%                                                                                                                           &
%     \\ \hline
%     %
%     8                                                                                                                     &
%     \begin{tabular}[c]{@{}l@{}}Penulisan Kode dan Pengeksekusian \\ Nilai Eigen \end{tabular}                             &
%                                                                                                                           &
%                                                                                                                           &
%                                                                                                                           &
%                                                                                                                           &
%     \checkmark                                                                                                            &
%                                                                                                                           &
%     \\ \hline
%     %
%     9                                                                                                                     &
%     \begin{tabular}[c]{@{}l@{}} Pengumpulan dan Analisis Semua data \end{tabular}                                         &
%                                                                                                                           &
%                                                                                                                           &
%                                                                                                                           &
%                                                                                                                           &
%     \checkmark                                                                                                            &
%     \checkmark                                                                                                            &
%     \\ \hline
%     10                                                                                                                    &
%     \begin{tabular}[c]{@{}l@{}} Penulisan Hasil dan Pembahasan \end{tabular}                                              &
%                                                                                                                           &
%                                                                                                                           &
%                                                                                                                           &
%                                                                                                                           &
%     \checkmark                                                                                                            &
%     \checkmark                                                                                                            &
%     \checkmark                                                                                                                                                                                         \\ \hline
%   \end{longtable}
% \end{landscape}

\chapter{HASIL DAN PEMBAHASAN}

Kajian mengenai Pemanfaatan General Purpose GPU dengan Bahasa Pemrograman Julia untuk Komputasi Berunjuk Kerja Tinggi ini diawali dengan melakukan pengukuran kecepatan eksekusi secara series oleh CPU dan secara paralel oleh GPU pada operasi penjumlahan, kemudian diakhiri dengan operasi pencarian nilai eigen. Operasi penjumlahan dinilai operasi paling sederhana sehingga dilakukan di paling awal. Sedangkan operasi pencarian nilai eigen dinilai sebagai operasi paling kompleks sehingga dilakukan di paling akhir.

\section{Hasil Simulasi Operasi Penjumlahan}

\begin{figure}[H]
	\centering
	\includegraphics[width=14cm, scale=1]{images/penelitian/addition.png}
	\caption{Hasil dari Operasi Penjumlahan Matriks}
	\label{img:result_addition}
\end{figure}

\begin{table}[H]
	\centering
	\caption{Hasil dari Operasi Penjumlahan Matriks}
	\label{tab:result_addition}
	\begin{tabular}{ccc}
		\toprule
		Variasi & CPU (s)  & GPU (s)  \\
		\midrule
		1       & 0.000002 & 0.000047 \\
		2       & 0.000002 & 0.000022 \\
		3       & 0.000009 & 0.000029 \\
		4       & 0.000423 & 0.000115 \\
		5       & 0.025585 & 0.009535 \\
		6       & 0.068290 & 0.008593 \\
		7       & 0.102998 & 0.034617 \\
		\bottomrule
	\end{tabular}
\end{table}

Perbandingan kecepatan eksekusi operasi penjumlahan secara series oleh CPU dan secara paralel oleh GPU menghasilkan Gambar \ref{img:result_addition}. Terlihat bahwa eksekusi series oleh CPU berhasil lebih cepat daripada eksekusi paralel oleh GPU pada ukuran matriks yang kecil. Namun, seiring bertambahnya ukuran matriks, kecepatan eksekusi paralel oleh GPU berhasil lebih cepat daripada kecepatan eksekusi series oleh CPU.

Meskipun secara grafik terlihat bahwa terdapat perbedaan kecepatan eksekusi yang signifikan, namun jika dilihat dari Tabel \ref{tab:result_addition}, semua eksekusi memerlukan waktu kurang dari 1 detik. Sehingga, operasi penjumlahan matriks masih bisa dianggap berjalan dengan baik di CPU maupun di GPU.

\section{Hasil Simulasi Operasi Pengurangan}

Berdasarkan penjelasan pada bagian \ref{Operasi Pengurangan}, operasi pengurangan merupakan modifikasi sederhana dari operasi penjumlahan. Sehingga diperkirakan akan memiliki hasil yang hampir sama dengan operasi penjumlahan. Hasil dari operasi pengurangan pada eksekusi series oleh CPU dan eksekusi paralel oleh GPU dapat dilihat pada Tabel \ref{tab:result_substraction} yang mana dapat dibentuk grafik pada Gambar \ref{img:result_substraction}.

\begin{figure}[H]
	\centering
	\includegraphics[width=14cm, scale=1]{images/penelitian/substraction.png}
	\caption{Hasil dari Operasi Pengurangan Matriks}
	\label{img:result_substraction}
\end{figure}

\begin{table}[H]
	\centering
	\caption{Hasil dari Operasi Pengurangan Matriks}
	\label{tab:result_substraction}
	\begin{tabular}{ccc}
		\toprule
		Variasi & CPU (s)              & GPU (s)              \\
		\midrule
		1       & $6.2 \times 10^{-7}$ & $2.0 \times 10^{-5}$ \\
		2       & $2.9 \times 10^{-6}$ & $2.8 \times 10^{-5}$ \\
		3       & $6.5 \times 10^{-6}$ & $2.1 \times 10^{-5}$ \\
		4       & $2.4 \times 10^{-4}$ & $9.6 \times 10^{-5}$ \\
		5       & $7.2 \times 10^{-4}$ & $3.9 \times 10^{-4}$ \\
		6       & $5.8 \times 10^{-2}$ & $8.6 \times 10^{-3}$ \\
		7       & $1.3 \times 10^{-1}$ & $3.5 \times 10^{-2}$ \\
		\bottomrule
	\end{tabular}
\end{table}

Berdasarkan Gambar \ref{img:result_substraction}, terlihat bahwa eksekusi series oleh CPU lebih cepat daripada eksekusi paralel oleh GPU pada matriks berukuran kecil. Seiring bertambahnya ukuran matriks, kecepatan eksekusi paralel oleh GPU berhasil lebih cepat daripada kecepatan eksekusi series oleh CPU.

Meskipun seiring bertambah nya ukuran matriks kecepatan eksekusi paralel oleh GPU lebih cepat daripada kecepatan eksekusi series oleh CPU, perbedaan antara kecepatan eksekusi yang diperoleh pada seperti yang terlihat pada Tabel \ref{tab:result_substraction}, tidak terlalu berbeda dan secara semua nya dibawah 1 detik. Untuk itu, operasi pengurangan matriks masih bisa dianggap berjalan dengan baik di CPU maupun di GPU.

\section{Hasil Simulasi Operasi Perkalian Skalar dengan Matrik}

Berdasarkan \ref{Operasi Perkalian dengan Skalar} merupakan modifikasi yang lebih kompleks dari operasi penjumlahan. Untuk itu, diperkirakan akan diperoleh hasil yang sedikit berbeda daripada operasi penjumlahan dan operasi pengurangan. Hasil dari perkalian skalar dengan matriks pada eksekusi series oleh CPU dan eksekusi paralel oleh GPU dapat dilihat pada Tabel \ref{tab:result_scalar_matrix_multiplication} yang mana hasil tersebut dapat dibentuk grafik Gambar \ref{img:result_scalar_matrix_multiplication}.

\begin{figure}[H]
	\centering
	\includegraphics[width=14cm, scale=1]{images/penelitian/scalar-matrix-multiplication.png}
	\caption{Hasil dari Operasi Perkalian Skalar dengan Matriks}
	\label{img:result_scalar_matrix_multiplication}
\end{figure}

\begin{table}[H]
	\centering
	\caption{Hasil dari Operasi Perkalian Skalar dengan Matriks}
	\label{tab:result_scalar_matrix_multiplication}
	\begin{tabular}{ccc}
		\toprule
		Variasi & CPU (s)  & GPU (s)  \\
		\midrule
		1       & 0.000001 & 0.000036 \\
		2       & 0.000016 & 0.000071 \\
		3       & 0.000034 & 0.000032 \\
		4       & 0.000913 & 0.000141 \\
		5       & 0.002931 & 0.000456 \\
		6       & 0.286256 & 0.007874 \\
		7       & 0.418853 & 0.031236 \\
		\bottomrule
	\end{tabular}
\end{table}

Berdasarkan Gambar \ref{img:result_scalar_matrix_multiplication}, terlihat bahwa kecepatan eksekusi series oleh CPU lebih cepat daripada kecepatan eksekusi paralel oleh GPU pada matriks yang berukuran kecil. Namun, seiring bertambah nya ukuran matriks, diperoleh kecepatan eksekusi paralel oleh GPU lebih cepat daripada kecepatan eksekusi series oleh CPU.

Jika dilihat dari Tabel \ref{tab:result_scalar_matrix_multiplication}, meskipun terdapat perbedaan kecepatan eksekusi series oleh CPU dan eksekusi paralel oleh GPU, waktu yang diperlukan untuk menjalankan eksekusi tersebut dibawah 1 detik. Untuk itu, dapat dikatakan bahwa operasi perkalian skalar dengan matriks dapat berjalan dengan baik di GPU dan di CPU.

\section{Hasil Simulasi Operasi Perkalian Antar Matrik}

Berdasarkan \ref{Operasi Perkalian Matriks dengan Matriks}, operasi perkalian antar matriks merupakan modifikasi operasi penjumlahan dan operasi perkalian, sehingga dapat dibilang bahwa operasi perkalian antar matriks memiliki kompleksitas yang lebih tinggi daripada operasi perkalian skalar dengan matriks. Untuk itu, diperkirakan akan diperoleh hasil yang sangat berbeda jika dibandingkan dengan operasi - operasi sebelum nya. Hasil dari operasi perkalian antar matriks ini dapat dilihat pada Tabel \ref{tab:result_matrix_matrix_multiplication} dan Gambar \ref{img:result_matrix_matrix_multiplication}.

\begin{figure}[H]
	\centering
	\includegraphics[width=14cm, scale=1]{images/penelitian/matrix-matrix-multiplication.png}
	\caption{Hasil dari Operasi Perkalian antar Matriks}
	\label{img:result_matrix_matrix_multiplication}
\end{figure}

\begin{table}[H]
	\centering
	\caption{Hasil dari Operasi Perkalian antar Matriks}
	\label{tab:result_matrix_matrix_multiplication}
	\begin{tabular}{ccc}
		\toprule
		Variasi & CPU (s)   & GPU (s)  \\
		\midrule
		1       & 0.000002  & 0.000060 \\
		2       & 0.000029  & 0.000097 \\
		3       & 0.000459  & 0.000098 \\
		4       & 0.003283  & 0.000416 \\
		5       & 0.017756  & 0.003405 \\
		6       & 1.231584  & 0.297797 \\
		7       & 25.754718 & 2.211518 \\
		\bottomrule
	\end{tabular}
\end{table}

Berdasarkan Gambar \ref{img:result_matrix_matrix_multiplication}, kecepatan eksekusi series oleh CPU berhasil lebih cepat daripada kecepatan eksekusi paralel oleh GPU pada matriks yang berukuran kecil. Seiring meningkatnya ukuran matriks, kecepatan eksekusi paralel oleh GPU berhasil lebih cepat daripada kecepatan eksekusi series oleh CPU.

Jika dilihat dari Tabel \ref{tab:result_matrix_matrix_multiplication}, pada variasi ke-7, kecepatan eksekusi paralel oleh GPU mempunyai durasi 2,2 detik sedangkan kecepatan eksekusi series oleh CPU mempunyai durasi 25,8 detik. Perbedaan ini cukup signifikan, dan diperkirakan akan semakin jauh perbedaannya jika ukuran matriks nya bertambah. Untuk itu, operasi perkalian antar matriks berjalan baik di CPU pada ukuran matriks yang tidak terlalu besar. Sehingga, jika terdapat operasi perkalian matriks, bisa menggunakan CPU jika ukuran matriks nya kecil, dan bisa menggunakan GPU jika ukuran matriks nya besar.

\section{Hasil Simulasi Operasi Inverse}

Berdasarkan bagian \ref{Operasi Inverse}, operasi pencarian matriks memiliki kompleksitas yang jauh lebih tinggi daripada operasi - operasi sebelumnya. Untuk itu, diperkirakan operasi pencarian inverse matriks ini akan memperoleh hasil yang berbeda dari operasi - operasi sebelumnya. Hasil simulasi operasi inverse pada sistem series dan sistem paralel dapat dilihat pada Tabel \ref{tab:result_inverse} dan Gambar \ref{img:result_inverse}.

\begin{figure}[H]
	\centering
	\includegraphics[width=14cm, scale=1]{images/penelitian/inverse.png}
	\caption{Hasil dari Operasi Inverse}
	\label{img:result_inverse}
\end{figure}

Berdasarkan Gambar \ref{img:result_inverse}, pada ukuran matriks yang kecil, kecepatan eksekusi series oleh CPU lebih cepat daripada kecepatan eksekusi paralel oleh GPU. Kemudian seiring meningkatnya ukuran matriks, kecepatan eksekusi paralel oleh GPU berhasil lebih cepat daripada kecepaan eksekusi series oleh CPU.

\begin{table}[H]
	\centering
	\caption{Hasil dari Operasi Inverse}
	\label{tab:result_inverse}
	\begin{tabular}{ccc}
		\toprule
		Variasi & CPU (s)    & GPU (s)   \\
		\midrule
		1       & 0.000014   & 0.000197  \\
		2       & 0.000136   & 0.000398  \\
		3       & 0.002382   & 0.000571  \\
		4       & 0.013442   & 0.017912  \\
		5       & 0.038086   & 0.034512  \\
		6       & 2.823527   & 1.068747  \\
		7       & 210.205265 & 10.711182 \\
		\bottomrule
	\end{tabular}
\end{table}

Jika dilihat dari Tabel \ref{tab:result_inverse} pada variasi ke-7, kecepatan eksekusi paralel oleh GPU mempunyai durasi kurang dari 11 detik, sedangkan kecepatan eksekusi series oleh CPU mempunyai durasi lebih dari 3 menit. Jika ukuran matriks nya ditambahkan, diperkirakan akan semakin jauh durasi ekeskusi paralel oleh GPU dan durasi eksekusi series oleh CPU. Untuk itu, jika terdapat operasi pencarian inverse matriks, bisa menggunakan CPU untuk ukuran matriks yang kecil, dan bisa menggunakan GPU untuk ukuran matriks yang besar.

\section{Hasil Simulasi Pencarian Nilai Eigen}

Berdasarkan bagian \ref{Nilai Eigen}, operasi pencarian nilai eigen matriks merpakan operasi yang paling kompleks pada kajian ini. Untuk itu, diperkirakan hasil yang diperoleh akan berbeda dari operasi operasi sebelumnya. Hasil dari simulasi operasi pencarian nilai eigen dapat dilihat pada Tabel \ref{tab:result_eigenvalue} dan Gambar \ref{img:result_eigenvalue}.

\begin{figure}[H]
	\centering
	\includegraphics[width=14cm, scale=1]{images/penelitian/eigenvalue.png}
	\caption{Hasil dari Pencarian Nilai Eigen}
	\label{img:result_eigenvalue}
\end{figure}

Berdasarkan Gambar \ref{img:result_eigenvalue}, pada ukuran matriks yang kecil, kecepatan eksekusi series oleh CPU lebih cepat daripada kecepatan eksekusi paralel oleh GPU. Namun, seiring bertambahnya ukuran matriks, kecepatan eksekusi paralel oleh GPU berhasil lebih cepat daripada kecepatan eksekusi series oleh CPU.

\begin{table}[H]
	\centering
	\caption{Hasil dari Pencarian Nilai Eigen}
	\label{tab:result_eigenvalue}
	\begin{tabular}{ccc}
		\toprule
		Variasi & CPU (s)    & GPU (s)   \\
		\midrule
		1       & 0.000035   & 0.000188  \\
		2       & 0.000718   & 0.000885  \\
		3       & 0.007700   & 0.001865  \\
		4       & 0.200065   & 0.027872  \\
		5       & 0.731179   & 0.121742  \\
		6       & 65.235088  & 5.941858  \\
		7       & 403.052508 & 42.954285 \\
		\bottomrule
	\end{tabular}
\end{table}

Jika dilihat dari \ref{tab:result_eigenvalue} pada variasi ke-7, kecepatan eksekusi paralel oleh GPU mempunyai durasi rata-rata kurang dari 43 detik, sedangkan kecepatan eksekusi series oleh CPU mempunyai durasi rata-rata lebih dari 6 menit. Jika ukuran matriks nya ditambah, maka diperkirakan akan menghasilkan perbedaan kecepatan yang jauh lebih berbeda antara eksekusi series oleh CPU dan eksekusi paralel oleh GPU. Untuk itu, jika terdapat operasi pencarian nilai eigen dari suatu matriks, dapat menggunakan CPU untuk ukuran matriks yang kecil, dan dapat menggunakan GPU untuk ukuran matriks yang besar.

\section{Pembahasan Secara Umum}

Pada kajian ini, dilakukan simulasi operasi matriks yang dijalankan pada sistem series oleh CPU dan sistem paralel oleh GPU. Simulasi ini dilakukan dari operasi sederhana hingga operasi yang kompleks. Eksekusi sistem series oleh CPU secara umum dapat menjalankan operasi matriks secara cepat pada matriks berukuran kecil dan pada operasi matriks yang tidak kompleks. Sedangkan eksekusi sistem paralel oleh GPU secara umum mampu menjalankan operasi matriks secara cepat pada matriks berukuran besar dan pada operasi matriks yang lebih kompleks.

Jika dilihat dari hasil gambar grafik pada masing-masing simulasi, terdapat titik dimana kecepatan eksekusi sistem paralel oleh GPU \textbf{mulai} lebih cepat daripada kecepatan eksekusi sistem series oleh CPU. Titik ini penulis sebut dengan \emph{titik balik}. Pada operasi penjumlahan, pengurangan, dan perkalian skalar dengan matriks, titik balik nya adalah di ukuran matriks $500 \times 500$ atau pada variasi ke-4. Kemudian pada operasi yang lebih kompleks, yakni operasi perkalian antar matriks, pencarian inverse, dan pencarian nilai eigen, titik balik nya ada pada ukuran matriks $100 \times 100$ atau variasi ke-3. Hasil ini dapat disimpulkan bahwa seiring meningkatnya kompleksistas operasi, eksekusi menggunakan sistem paralel oleh GPU dapat lebih efektif daripada eksekusi pada sistem series oleh CPU.

Jika dilihat dari hasil tabel pada masing-masing simulasi, semakin tinggi kompleksitas operasi maka semakin tinggi pula perbedaan durasi eksekusi pada ukuran matriks terbesar (variasi ke-7). Pada operasi penjumlahan, pengurangan, dan perkalian skalar dengan matriks, perbedaan durasi eksekusi pada variasi ke-7 secara umum mempunyai rata-rata kurang dari 1 detik. Namun, pada operasi yang lebih kompleks, yakni operasi perkalian antar matriks, pencarian inverse, dan pencarian nilai eigen, durasi eksekusi pada variasi ke-7 antara sistem series oleh CPU dan sistem paralel oleh GPU semua nya diatas 1 detik dan perbedaan paling besar ada pada simulasi pencarian nilai eigen. Hasil ini dapat disimpulkan bahwa seiring meningkatnya kompleksitas operasi dan seiring meningkatnya ukuran matriks, eksekusi paralel oleh GPU jauh lebih efektif daripada eksekusi series oleh CPU.

Operasi matriks tidak bergantung satu sama lain, sehingga dapat dijalankan dengan baik pada sistem paralel. Namun, pada hasil yang telah diperoleh, pada matriks berukuran kecil, sistem series lebih cepat melakukan dalam menjalankan eksekusi daripada sistem paralel. Hal ini disebabkan karena semua program pada dasarnya dijalankan di CPU. Untuk sistem series, maka eksekusi program tersebut dapat langsung dijalankan. Akan tetapi, untuk sistem paralel, eksekusi program tersebut perlu dikirimkan dari CPU ke GPU, kemudian setelah selesai dikomputasikan akan dikirimkan kembali ke CPU. Proses transfer data ini akan tampak lama jika data yang ditransferkan berukuran kecil. Dengan kata lain, proses transfer data dari CPU ke GPU dan dari GPU ke CPU akan tampak lama jika ukuran data matriks nya sedikit (matriks berukuran kecil).

Dari hasil yang telah diperoleh pada masing-masing simulasi, terlihat bahwa pada ukuran matriks yang sedikit, eksekusi sistem series oleh CPU mampu lebih cepat daripada eksekusi sistem series oleh GPU. Penulis tidak memperkirakan jika CPU dapat berjalan secepat itu. Kemungkinan cepat nya eksekusi pada CPU ini disebabkan karena CPU yang penulis pakai menggunakan Intel Core i7 tipe U generasi 10, yang mana merupakan salah satu CPU tercepat di kelas nya. Untuk itu, jika simulasi dilakukan dengan menggunakan CPU lain yang lebih lambat, diperkirakan akan diperoleh \emph{titik balik} yang lebih awal.

% \section{Hasil dari Metode Gauss-Seidel}
% Untuk menghitung seluruh data menggunakan program ini, dilakukan pembagian kloter yang setiap kloternya menghitung 2000--2200 data. Hal ini dilakukan untuk menghindari perhitungan yang terlalu lama, penggunaan memori yang terlalu besar, dan juga untuk mengantisipasi kesalahan yang besar. Untuk 20200, data dibagi menjadi 10 kloter perhitungan. Pada Gambar \ref{grafik_durasi} disajikan data dari 5 perhitungan waktu program. Dari data tersebut dapat kita simpulkan bahwa rata-rata waktu perhitungan untuk 2000--2200 data adalah 6712 detik atau 1,8 jam. Sehingga, untuk menghasilkan perhitungan $\phi$ untuk satu medan adalah sebesar 3,356 detik.
%
% Dalam perhitungan dengan jumlah yang besar atau simulasi kontinyu, misalnya, waktu perhitungan merupakan sesuatu yang sangat berpengaruh. Waktu 3,356 detik untuk sekitar 2000 medan akan memakan waktu yang relatif lama. Metode iteratif Gauss-Seidel memang membutuhkan waktu konvergen yang lama, namun dibandingkan dengan metode iteratif lainnya seperti Jacobi, metode Gauss-Seidel memiliki keakuratan yang relatif lebih tinggi \citep{Ford2015}. Metode Gauss-Seidel juga merupakan metode yang memberikan perkenalan mengenai konsep penggunaan iterasi untuk penyelesaian sistem (linear, dan sebagainya) dan basis untuk metode yang lebih canggih\defcitealias{Ford2015}{ibid} \citepalias{Ford2015}. Hal ini sangat baik untuk pembelajaran mahasiswa sarjana karena mudah diphami dan mudah untuk diimplementasikan.
% \begin{figure}[h!]
%     \centering
%     \includegraphics[width=12cm]{gambar/grafik_durasi.png}
%     \caption{Grafik durasi perhitungan Gauss-Seidel terhadap jumlah data}
%     \label{grafik_durasi}
% \end{figure}
%
% Perhitungan tersebut menghasilkan kumpulan data $\phi$ sejumlah banyaknya data $\rho$ yang memiliki rentang nilai data dari 1814160 sampai -1794090. Apabila dibandingkan dengan nilai toleransi sebesar 0,01, nilai ekstrim dari $\phi$ yang dihasilkan oleh \textit{solver} ini  relatif sangat besar, sehingga ralat yang terjadi sangat dapat diterima. Dari histogram pada Gambar \ref{hist_rata2phi} dapat kita ketahui bahwa nilai rata-rata dari $\phi$ tersebar pada rentang yang sangat besar dan relatif jauh dari nilai toleransi 0,01. Data ini tetap dapat menjustifikasi bahwa nilai toleransi ralat 0,01 merupakan nilai yang relatif sangat kecil bagi hasil $\phi$. Pesebaran nilai ekstrim dari seluruh medan terdapat pada Gambar \ref{phi_gs_rendah_tinggi}.
% \begin{figure}[h!]
%     \centering
%     \includegraphics[width=12cm]{gambar/phi_gs_rendah_tinggi.png}
%     \caption{Kiri: Histogram nilai tertinggi pada 20200 data $\phi$. Kanan: Histogram nilai tertinggi pada 20200 data $\phi$.}
%     \label{phi_gs_rendah_tinggi}
% \end{figure}
%
% \begin{figure}[h!]
%     \centering
%     \includegraphics[width=12cm]{gambar/hist_rata2phi.png}
%     \caption{Histogram nilai rata-rata phi dari seluruh medan.}
%     \label{hist_rata2phi}
% \end{figure}
%
% \cite{cheng_illarramendi_bauerheim_cuenot_2021} menjelaskan bahwa dua puncak Gaussian yang terlihat pada medan distribusi muatan akan tersebar dan menyatu pada medan potensial yang dihasilkan. Hal ini disebutkan oleh mereka karena filter \textit{low-pass} oleh operator Laplacian terbalik. Hal yang sama dapat dilihat pada potensial yang dihasilkan dalam perhitungan ini pada Gambar \ref{rho_phi_GS}.
%
% \begin{figure}[h!]
%     \centering
%     \includegraphics[width=12cm]{gambar/rho_phi_GS.png}
%     \caption{Contoh 3 pasang medan $\rho$ dan $\phi$ hasil perhitungan Gauss-Seidel}
%     \label{rho_phi_GS}
% \end{figure}
%
% Selanjutnya, data $\phi$ yang sudah terbentuk dilakukan normalisasi untuk mendapatkan skala yang sama pada tiap medan agar jaringan saraf (\textit{neural network}) dapat belajar dengan baik \citep{cheng_illarramendi_bauerheim_cuenot_2021}. Teknik penskalaan yang digunakan adalah dengan membagi keseluruhan 20200 data dengan nilai ekstrim mutlak dari keseluruhan data, dalam hal ini adalah 1814160. Setelah dilakukan normalisir, tidak terjadi perubahan distribusi pada $\phi$, hanya nilainya saja yang berubah dengan rasio 1/1814160 (Gambar \ref{rho_phi_GS_1814160}.
%
% \begin{figure}[h!]
%     \centering
%     \includegraphics[width=12cm]{gambar/rho_phi_GS_1814160.png}
%     \caption{Gambar ke-3 dari Gambar \ref{rho_phi_GS} yang sudah dinormalisir dengan cara dibagi dengan 1814160.}
%     \label{rho_phi_GS_1814160}
% \end{figure}
%
% \section{Pelatihan Jaringan Saraf \textit{Neural Network}}
%
% \subsection{Arsitektur U-Net}
% Arsitektur yang digunakan pada penelitian ini adalah U-Net (Subbagian \ref{sub_unet}) yang terlampir pada Lampiran \ref{unies_og}. Penggambaran grafis mengenai arsitektur yang digunakan digambarkan pada Gambar \ref{arsitektur_unies}.
%
% \begin{figure}[h!]
%     \centering
%     \includegraphics[width=14cm]{gambar/unies_og.png}
%     \caption{Grafik arsitektur yang digunakan. Angka di atas filter adalah jumlah filter yang digunakan, dan angka di bawah filter merupakan ukuran filter yang digunakan.}
%     \label{arsitektur_unies}
% \end{figure}
%
% Ukuran filter atau medan penerimaan (\textit{receptive field}) serta jumlah filter yang digunakan adalah bagian dari penelitian yang mencoba-coba hingga didapatkan hasil terbaik sesuai yang diharapkan. Untuk bentuk arsitektur, tidak seperti U-Net yang awalnya dicetuskan oleh \cite{DBLP:journals/corr/RonnebergerFB15}, ada modifikasi pada bagian \textit{crop and copy} yang berfungsi untuk membuang dan menyalin bagian tepi dari medan \citep{siddique_paheding_elkin_devabhaktuni_2021}. Hal ini sesuai dengan percobaan yang dilakukan \cite{alom2018recurrent} yang memodifikasi U-Net menjadi \textit{Recurrent U-Net} (RU-Net) yang menghasilkan arsitektur yang lebih mutakhir dan memiliki peforma yang lebih baik.
%
% \subsection{Pelatihan}
%
% Pelatihan dilakukan menggunakan Google Colab Pro dengan menggunakan GPU T4. Untuk melatih 1.698.177 parameter dengan 200 \textit{epoch}, dibutuhkan waktu 2 jam 30 menit (45 detik per \textit{epoch}). Untuk mengevaluasi model, dibutuhkan sampel data validasi yang dapat digunakan untuk memberikan evaluasi kesesuaian model yang tidak bias pada set data pelatihan sambil menyetel \textit{hyperparameter} model \citep{elgendy_2020}. Set data validasi yang digunakan pada pelatihan ini menggunakan 25\% dari total set data latih (5050 data).
%
% Dalam rangka mengupayakan agar model jaringan saraf atau algoritma pembelajaran dapat berkembang atau lebih cepat beradaptasi pada bobot model terhadap set data latih, maka dilakukan konfigurasi laju pemelajaran (\textit{learning rate}). Laju pemelajaran dapat dikatakan sebagai jumlah perubahan pada model selama setiap langkah proses pencarian solusi (titik optimal global), atau bisa disebut juga sebagai ukuran langkah \citep{brownlee_2019a}. 
%
% Dalam pelatihan jaringan saraf, dapat dilakukan penurunan laju pemelajaran (\textit{learning rate decay}) selama proses pelatihan. Salah satu langkah yang dapat dilakukan adalah dengan jadwal laju pemelajaran (\textit{learning rate schedule}). Dalam penelitian ini, digunakan laju penurunan eksponensial (\textit{exponential learning rate decay}) (Gambar \ref{exp_lr}) yang memiliki bentuk $lr = lr_0 \times exp(-kt)$, dengan $lr_0$ dan $k$ adalah \textit{hyperparameter} dan $t$ adalah bilangan iterasi \citep{suki_lau_2017}. Laju penurunan eksponensial ini digunakan pada pengoptimasi ADAM \citep{kingma2017adam} dengan laju awal = 0,001.
%
% \begin{figure}[h!]
%     \centering
%     \includegraphics[width=12cm]{gambar/exp.png}
%     \caption{\textit{Exponential Learning Rate Decay.} \textit{Sumber: \citep{suki_lau_2017}}.}
%     \label{exp_lr}
% \end{figure}
%
% Model ini akan dilatih sebanyak 200 epoch dan model dari tiap epoch akan disimpan dalam format file \texttt{h5} agar dapat dipilih model dengan \texttt{val\_loss} (ralat dari set data validasi) terendah dan tidak ada indikasi \textit{overfitting}.
%
% \subsection{Hasil Pelatihan}
% Salah satu cara untuk mendiagnosa \textit{overfitting} dan \textit{underfitting} adalah dengan me-\textit{plot} ralat pelatihan dan ralat validasi \citep{elgendy_2020} atau yang disebut dengan kurva pelatihan (\textit{learning curve}). Kurva pelatihan untuk pelatihan model ini terdapat pada Gambar \ref{learning_curve}
%
% \begin{figure}[h!]
%     \centering
%     \includegraphics[width=12cm]{gambar/learning_curve.png}
%     \caption{Kurva pemelajaran model. Jumlah ralat pada set data pelatihan dan validasi terhadap jumlah \textit{epoch}}
%     \label{learning_curve}
% \end{figure}
%
% Secara umum, pelatihan ini menghasilkan diagnosa \textit{overfitting} karena model mempelajari set data latih terlalu baik dibandingkan set data validasi \citep{brownlee_2019a}. Hal ini terlihat dari kurva validasi yang berada di atas kurva latih, yang artinya, pada pelatihan terakhir, model menghasilkan ralat validasi yang lebih besar dari pada ralat latih. Walaupun perbedaan ralat di akhir pelatihan relatif kecil, namun munculnya indikasi \textit{overfitting} tidak bisa dinafikan.
%
% Hasil pelatihan ini tetap dapat digunakan dengan cara mencari model terbaik dari \textit{epochs} pelatihan. Salah satu indikasi model tersebut \textit{goodfit} adalah antara kurva validasi dan pelatihan tidak ada jarak (berhimpit) atau kurva validasi berada di bawah kurva latih. Berdasarkan hasil pengamatan, model terbaik adalah model dari \textit{epoch} ke-57 dengan ralat latih : 2.1185e-04 dan ralat validasi: 2.1775e-04.
%
% \section{Hasil Prediksi Jaringan Saraf}
% Prediksi dilakukan menggunakan dua model. Yaitu model dari \textit{epoch} ke-57 dan model dari \textit{epoch} ke-200. Kedua model akan memprediksi dua set data, yaitu set data acak sebanyak 100 data dan set data Gaussian (Persamaan \eqref{gaussian}) sebanyak 18 data.
%
% Setelah melakukan prediksi, hasil prediksi yang sudah dinormalisasi dengan nilai tertinggi dari set data latih (1814160), akan akan diperiksa \textit{mean squared error} (MSE) dan \textit{mean absolute error} (MAE) dari keseluruhan set data dan beberapa medan yang dipilih secara acak dari masing-masing set data uji.
%
% Selain MSE dan MAE, akan diperiksa pula MAPE (Persamaan \eqref{mape}) dari $>1000$ piksel yang nilainya lebih dari 0,1 yang dipilih secara acak. Pemilihan jumlah sampel sebanyak 1000 adalah sesuai dengan rekomendasi dari \cite{israel1992determining} untuk jumlah populasi data 100 $\times$ 100 = 10.000, dibutuhkan setidaknya 1.000 data untuk mendapatkan \textit{margin of error} $\pm 3\%$. Dan untuk pengambilan data di atas 0,1, hal ini karena menurut \cite{DEMYTTENAERE201638}, MAPE lebih cocok untuk mengukur ralat dengan nilai yang jauh dari nol. Salah satu penyebabnya adalah adanya pembagian dengan nilai sebenarnya, yang apabila nilai tersebut nol atau sangat kecil, ralat yang dihasilkan bisa sangat besar, atau intinya, MAPE sangat sensitif terhadap nilai nol atau yang sangat kecil.
%
% \subsection{Hasil Prediksi data Acak 100 Data}
% Dengan menggunakan \textit{solver} Gauss-Seidel C++, waktu yang dibutuhkan untuk menghitung potensial listrik dari 100 data dengan distribusi acak adalah sebesar 325 detik.
%
% \subsubsection{Hasil Prediksi Dengan Model dari \textit{Epoch} ke-200}
% Akan ditinjau terlebih dahulu hasil prediksi menggunakan model dari \textit{epoch} ke-200. Hasil umum dari prediksi 100 data acak dapat dilihat pada Tabel \ref{200acak}.
%
% \begin{table}[h!]
% \centering
% \caption{Hasil keseluruhan prediksi dengan model epoch ke-200 untuk 100 data acak.}
% \label{200acak}
% \begin{tabular}{ll}
% \hline
% \textbf{Metrik}        & \textbf{Nilai}         \\ \hline
% Waktu prediksi (detik) & 3                      \\ \hline
% MSE                    & 0.00023130870479720733 \\ \hline
% MAE                    & 0.0105920327822764     \\ \hline
% \end{tabular}
% \end{table}
%
% Kemudian disajikan hasil visual dari 3 medan yang dipilih secara acak beserta dengan rerata ralatnya
% \begin{enumerate}
%     \item Data ke-0
%     
%     mean absolute error medan =  0.011185391142800172\\
%     mean squared error medan = 0.0002198482412056808\\
%     mape sampling =  6.652328247576015
%     \begin{figure}[h!]
%     \centering
%     \includegraphics[width=12cm]{gambar/0_200_acak.png}
%     \caption{Hasil visual prediksi dengan model dari epoch ke-200 untuk set data acak, indeks ke-0}
%     \label{0_200_acak}
%     \end{figure}
%
%     \item Data ke-40
%
%     mean absolute error medan =  0.013264649605893237\\
%     mean squared error medan =  0.00027769889839297595\\
%     mape sampling =  -7.139269391922416
%     \begin{figure}[h!]
%     \centering
%     \includegraphics[width=12cm]{gambar/40_200_acak.png}
%     \caption{Hasil visual prediksi dengan model dari epoch ke-200 untuk set data acak, indeks ke-40}
%     \label{40_200_acak}
%     \end{figure}
%
%     \item Data ke-42
%     mean absolute error medan =  0.009555791057188687\\
%     mean squared error medan = 0.0001709383632965548\\
%     mape sampling =  15.947304897803454
%     \begin{figure}[h!]
%     \centering
%     \includegraphics[width=12cm]{gambar/42_200_acak.png}
%     \caption{Hasil visual prediksi dengan model dari epoch ke-200 untuk set data acak, indeks ke-42}
%     \label{42_200_acak}
%     \end{figure}
%     
% \end{enumerate}
%
% \subsubsection{Hasil Prediksi Dengan Model dari \textit{Epoch} ke-57}
% Akan ditinjau terlebih dahulu hasil prediksi menggunakan model dari \textit{epoch} ke-57. Hasil umum dari prediksi 100 data acak dapat dilihat pada Tabel \ref{57acak}.
%
% \begin{table}[h!]
% \centering
% \caption{Hasil keseluruhan prediksi dengan model epoch ke-57 untuk 100 data acak.}
% \label{57acak}
% \begin{tabular}{ll}
% \hline
% \textbf{Metrik}        & \textbf{Nilai}         \\ \hline
% Waktu prediksi (detik) & 2                      \\ \hline
% MSE                    & 0.0002451981196005638 \\ \hline
% MAE                    & 0.011251571534445289     \\ \hline
% \end{tabular}
% \end{table}
%
% Kemudian disajikan hasil visual dari 3 medan yang dipilih secara acak beserta dengan rerata ralatnya
% \begin{enumerate}
%     \item Data ke-0
%     
%     mean absolute error medan =  0.013908464504802816\\
%     mean squared error medan =  0.000344087854795096\\
%     mape sampling =  -8.33244489614455
%     \begin{figure}[h!]
%     \centering
%     \includegraphics[width=12cm]{gambar/0_57_acak.png}
%     \caption{Hasil visual prediksi dengan model dari epoch ke-57 untuk set data acak, indeks ke-0}
%     \label{0_57_acak}
%     \end{figure}
%
%     \item Data ke-40
%
%     mean absolute error medan =  0.014362610343673214\\
%     mean squared error medan =  0.00030545219599322355\\
%     mape sampling =  -7.8525725979344605
%     \begin{figure}[h!]
%     \centering
%     \includegraphics[width=12cm]{gambar/40_57_acak.png}
%     \caption{Hasil visual prediksi dengan model dari epoch ke-57 untuk set data acak, indeks ke-40}
%     \label{40_57_acak}
%     \end{figure}
%
%     \item Data ke-42
%     mean absolute error medan =  0.01258524854440124\\
%     mean squared error medan =  0.0002786093680182126\\
%     mape sampling =  21.275292145676676
%     \begin{figure}[h!]
%     \centering
%     \includegraphics[width=12cm]{gambar/42_57_acak.png}
%     \caption{Hasil visual prediksi dengan model dari epoch ke-57 untuk set data acak, indeks ke-42}
%     \label{42_57_acak}
%     \end{figure}
%     
% \end{enumerate}
%
% \subsection{Hasil Prediksi data Gaussian 18 Data}
% Dengan menggunakan \textit{solver} Gauss-Seidel C++, waktu yang dibutuhkan untuk menghitung potensial listrik dari 18 data dengan distribusi Gaussian adalah sebesar 45 detik.
%
% \subsubsection{Hasil Prediksi Dengan Model dari \textit{Epoch} ke-200}
% Akan ditinjau terlebih dahulu hasil prediksi menggunakan model dari \textit{epoch} ke-200. Hasil umum dari prediksi 18 data Gaussian dapat dilihat pada Tabel \ref{200acak}.
%
% \begin{table}[h!]
% \centering
% \caption{Hasil keseluruhan prediksi dengan model epoch ke-200 untuk 18 data Gaussian.}
% \label{200gaussian}
% \begin{tabular}{ll}
% \hline
% \textbf{Metrik}        & \textbf{Nilai}         \\ \hline
% Waktu prediksi (detik) & 0                      \\ \hline
% MSE                    & 0.0004893961671139401 \\ \hline
% MAE                    & 0.014389271830866605  \\ \hline
% \end{tabular}
% \end{table}
%
% Kemudian disajikan hasil visual dari 3 medan yang dipilih secara acak beserta dengan rerata ralatnya
% \begin{enumerate}
%     \item Data ke-6
%     mean absolute error medan = 0.009043942980161843\\
%     mean squared error medan = 0.00017162983257671422\\
%     mape sampling = 4.7754985462483495
%     \begin{figure}[h!]
%     \centering
%     \includegraphics[width=12cm]{gambar/6_200_gaussian.png}
%     \caption{Hasil visual prediksi dengan model dari epoch ke-200 untuk set data Gaussian, indeks ke-6}
%     \label{6_200_Gaussian}
%     \end{figure}
%
%     \item Data ke-10
%     mean absolute error medan = 0.021639103274408242\\
%     mean squared error medan = 0.0007739544987668907\\
%     mape sampling = 15.772617188907128
%     \begin{figure}[h!]
%     \centering
%     \includegraphics[width=12cm]{gambar/10_200_gaussian.png}
%     \caption{Hasil visual prediksi dengan model dari epoch ke-200 untuk set data Gaussian, indeks ke-10}
%     \label{10_200_Gaussian}
%     \end{figure}
%
%     \item Data ke-4
%     mean absolute error medan =  0.025968039199870924\\
%     mean squared error medan = 0.0011480142622138615\\
%     mape sampling =  29.026637482792307
%     \begin{figure}[h!]
%     \centering
%     \includegraphics[width=12cm]{gambar/4_200_gaussian.png}
%     \caption{Hasil visual prediksi dengan model dari epoch ke-200 untuk set data Gaussian, indeks ke-4}
%     \label{4_200_gaussian}
%     \end{figure}
%     
% \end{enumerate}
%
% \subsubsection{Hasil Prediksi Dengan Model dari \textit{Epoch} ke-57}
% Akan ditinjau terlebih dahulu hasil prediksi menggunakan model dari \textit{epoch} ke-57. Hasil umum dari prediksi 18 data Gaussian dapat dilihat pada Tabel \ref{57gaussian}.
%
% \begin{table}[h!]
% \centering
% \caption{Hasil keseluruhan prediksi dengan model epoch ke-57 untuk 18 data Gaussian.}
% \label{57gaussian}
% \begin{tabular}{ll}
% \hline
% \textbf{Metrik}        & \textbf{Nilai}         \\ \hline
% Waktu prediksi (detik) & 0                      \\ \hline
% MSE                    & 0.0004993306540695422 \\ \hline
% MAE                    & 0.014593957448582126     \\ \hline
% \end{tabular}
% \end{table}
%
% Kemudian disajikan hasil visual dari 3 medan yang dipilih secara acak beserta dengan rerata ralatnya
% \begin{enumerate}
%     \item Data ke-6
%     
%     mean absolute error medan =  0.011164339060049471\\
%     mean squared error medan =  0.0003275269746363096\\
%     mape sampling =  4.860758654485836
%     \begin{figure}[h!]
%     \centering
%     \includegraphics[width=12cm]{gambar/6_57_gaussian.png}
%     \caption{Hasil visual prediksi dengan model dari epoch ke-57 untuk set data Gaussian, indeks ke-6}
%     \label{0_57_acak}
%     \end{figure}
%
%     \item Data ke-10
%     mean absolute error medan = 0.020210462134109868\\
%     mean squared error medan = 0.0006902836080606566\\
%     mape sampling =  17.241594270499036
%     \begin{figure}[h!]
%     \centering
%     \includegraphics[width=12cm]{gambar/10_57_gaussian.png}
%     \caption{Hasil visual prediksi dengan model dari epoch ke-57 untuk set data Gaussian, indeks ke-10}
%     \label{10_57_acak}
%     \end{figure}
%
%     \item Data ke-4
%     mean absolute error medan =  0.02386963821168969\\
%     mean squared error medan =  0.0010337360801227914\\
%     mape sampling =  25.63144233512775
%     \begin{figure}[h!]
%     \centering
%     \includegraphics[width=12cm]{gambar/4_57_gaussian.png}
%     \caption{Hasil visual prediksi dengan model dari epoch ke-57 untuk set data Gaussian, indeks ke-4}
%     \label{4_57_Gaussian}
%     \end{figure}
%     
% \end{enumerate}
%
% \section{Pembahasan Secara Umum}
%
% Dari pengamatan visual, dapat diklaim bahwa model yang telah dilatih dapat memberikan gambaran umum yang sesuai dengan visual \textit{ground truth} yang didapat dari metode Gauss-Seidel, terutama pada set data dari data acak. Hal ini sangat masuk akal, mengingat model ini dilatih menggunakan set data latih acak. Untuk kemampuan model memprediksi data distribusi lain, perlu adanya penambahan jenis distribusi lain ke set data latih atau membangun model lain dengan set data latih yang spesifik.
%
% Untuk waktu perhitungan, misalnya waktu perhitungan yang paling tinggi, menggunakan model dari \textit{epoch} ke-200 untuk 100 data distribusi acak memakan waktu 3 detik yang berarti sama dengan perhitungan untuk 1 medan menggunakan \textit{solver} Gauss-Seidel yang digunakan untuk membuat set data latih. Secara keseluruhan, model ini 162,5 $\times$ lebih cepat dalam hal waktu perhitungan.
%
% Dari segi ralat, rerata ralat mutlak dan rerata ralat absolut sudah relatif kecil. Dari visual ralatpun sudah terlihat mayoritas berkonsentrasi rendah, dan hanya meninggi pada distribusi di syarat batas Neumann, misalnya Gambar \ref{0_57_acak}. Untuk MAPE, meski dapat diperdebatkan relativitas besar kecilnya, penulis memiliki pandangan bahwa model dapat menghasilkan MAPE yang relatif kecil. Hal ini karena, pertama, MAPE memang ralat yag tepat digunakan untuk model regresi. Namun, MAPE memang memiliki beberapa kelemahan, beberapa diantaranya adalah MAPE bergantung pada situasi, derajat variabilitas data, rentang nilai sebenarnya, serta \textit{outliers}. Data uji dan data latih yang digunakan masih memiliki beberapa kelemahan, salah satunya adalah bilangan pernormalisir yang jumlahnya tidak mewakili jumlah data yang memiliki nilai tertinggi di sekitar nilai penormalisir. Sehingga sangat memungkinkan ada pergeseran distribusi. Akan lebih memungkinkan apabila ada upaya \textit{data cleaning} seperti misalnya membuang beberapa \textit{tail} data yang nilai ekstrimnya berada jauh dari pusat distribusi sehingga bilangan penormalisirnya bisa lebih mewakili keseluruhan data. Hal tersebut tidak dilakukan karena penulis melihat pada upaya pertama, hasil yang diperoleh sudah cukup bagus dari metrik ralat yang lain dan dari pengamatan visual.
%
% Masih mengenai kelemahan MAPE, MAPE memiliki bias pada hasil prediksi yang berada di bawah nilai \textit{ground truth} \citep{roberts_2023b}. Namun meski begitu, MAPE memberikan gambaran proporsi mengenai ralat dari prediksi terhadap nilai sebenarnya. Sehingga masih relevan untuk digunakan untuk mengukur kemampuan model yang dibangun. Hanya saja memang harus ada upaya untuk memilih data yang akan dijadikan komponen pengukuran, seperti upaya sampling $>1000$ data pada medan yang diukur. 
%
% Beberapa medan hasil yang tidak ditampilkan di sini, ada beberapa medan yang memiliki nilai MAPE dari hasil sampling sebesar kira-kira 3--5\%. Memang hal ini tidak dapat sepenuhnya mengatakan bahwa model ini dapat digunakan sebagai pengganti metode numerik yang sudah mapan, namun dapat mengatakan bahwa model yang dibangun dan dilatih ini dapat mendekati hasil aslinya. Sebagai perbandingan, \cite{Ozbay2021} dapat membangun model untuk 2D dan 3D pada koordinat kartesian yang menghasilkan MAPE di bawah 3\%.
%
% Pada penelitian ini, disajikan 2 model. Berdasarkan pengamatan MSE dari 10 medan yang dihitung menggunakan masing-masing model (Gambar \ref{mse2model}, menunjukkan bahwa kedua model dapat menghasilkan dapat unggul dengan jumlah yang berimbang, artinya, model 57 unggul di 5 kali percobaan, dan model 200 unggul di 5 percobaan juga. Hal ini dapat menggiring pada kesimpulan bahwa meski model 200 memiliki indikasi \textit{overfitting}, namun karena ralat latihnya relatif lebih kecil, model 200 tetap dapat diandalkan. 
% \begin{figure}[h!]
%     \centering
%     \includegraphics[width=12cm]{gambar/mse dari dua model.png}
%     \caption{MSE dari dua model dari 10 medan yang dihitung}
%     \label{mse2model}
% \end{figure}
%
% Akhirnya, dari hasil yang didapat, tetap harus dicatat bahwa metode yang diajukan tidak dapat dipandang sebagai pengganti metode numerik klasik dalam pemecahan permasalahan PDE. Metode tersebut sudah sangat matang dan mapan setelah penggunaan oleh berbagai teknisi dan ilmuwan selama lebih dari 50 tahun dan telah teruji berbagai standar ketahanan dan efisiensi komputasi yang disyaratkan \citep{DBLP:journals/corr/abs-1711-10561}.
%
% Penulis pribadi berpendapat bahwa hasil dari perhitungan metode CNN ini belum dapat diguankan sebagai pengganti murni perhitungan PDE. Namun sangat bisa sebagai tebakan awal dalam perhitungan iteratif seperti Gauss-Seidel yang harapannya berfungsi sebagai akselerator dalam konvergensi.

\chapter{KESIMPULAN DAN SARAN}
\section{Kesimpulan}
Dari penelitian ini, dapat disimpulkan
\begin{enumerate}
	\item Cara kerja komputasi di sistem paralel oleh GPU adalah dengan menjalankan program di CPU, kemudian mengirimkan data dari CPU ke GPU, GPU melakukan komputasi pada data yang diterima, kemudian hasil kalkulasi akan dikirimkan dari GPU ke CPU. Untuk mengatur supaya CPU menunggu kalkulasi GPU, perlu ditambahkan \cw{@async} pada kode Julia nya.
	\item Pada operasi yang sederhana dan pada ukuran matriks yang kecil, eksekusi sistem series oleh CPU dapat lebih cepat daripada eksekusi sistem paralel oleh GPU. Namun, pada operasi yang kompleks dan pada ukuran matriks yang besar, eksekusi sistem paralel oleh GPU dapat lebih cepat daripada eksekusi sistem series oleh CPU.
\end{enumerate}

% \section{Saran}
% \begin{enumerate}
% \item Dilakukan \textit{data cleaning} sebelum menggunakan data latih dan data uji
% \item Membuka kemungkinan pada jenis/bentuk arsitektur lain, khususnya pemanfaatan \textit{transfer learning} dalam pembentukan arsitektur.

% \item Penulis hanya mempunyai RAM yang terbatas, sehingga hasil pengujian pada ukuran matriks yang lebih besar masih belum bisa dilakukan
% \end{enumerate}


%-----------------------------------------------------------------
%Disini akhir masukan Bab
%-----------------------------------------------------------------

%-----------------------------------------------------------------
%Disini awal masukan untuk Daftar Pustaka (ISI SESUAI DENGAN DATA ANDA!)
%-----------------------------------------------------------------

% \begin{thebibliography}{99}
\addcontentsline{toc}{chapter}{DAFTAR PUSTAKA}
\singlespacing
\bibliography{skripsi.bib}
% \bibliography{Skripsi.bib}
\onehalfspacing
% \bibitem[Anton(2005)]{anton05}
% sebagai contoh
% Anton,H., Rorres, C. 2005. \emph{Elementary Linear Algebra, Apllication Version, Ninth Edition}. John Wiley and Sons, New Jersey.

% \bibitem[Bazaraa(2006)]{bazaraa06}
% Bazaraa, M.S., dkk. 2006. \emph {Nonlinear Programming, Theory and Algorithms}, 3rd edition. John Wiley and Sons, New Jersey.

% \bibitem[Boyd(2004)]{boyd04}
% Boyd, S., Vandenberghe, L. 2004. \emph {Convex Optimization}. Cambridge University Press.

% \bibitem[Comacho(2007)]{comacho07}
% Camacho, E.F., Bordons, C. 2007. \emph{Model Predictive Control Second Edition}. Springer Verlag, London.

% \bibitem[Dalhoumi, dkk(2016)]{dalhoumi16}
% Dalhoumi, Latifa, dkk. 2016. \emph{Model Based Predictive Control For Linear Interconnected Systems}. Control and Energy Mangement Lab National school of engineering of Sfax, P.B. 1173, 3083 Sfax, Tunisia.  13th International Multi-Conference on Systems, Signal, and Devices.

% \bibitem[Fatimah(2019)] {fatimah19}
% Fatimah, K.R.N. 2019. \emph{Kendali Model Prediktif Berkendala Menggunakan Active-Set dengan Linearisasi Umpan Balik pada Sistem Rumah Kaca}. Skripsi. Departemen Matematika UGM. Yogyakarta.

% \bibitem[Maciewjowski(2000)]{maciejowski00}
% Maciewjowski, J.M., 2000.\emph{Predictive Control with Constrain}. Prentice Hall, USA.

% \bibitem[Olsder(1994)]{olsder94}
% Olsder,G.J., van der Woude,J.W..1994. \emph{Mathematical Systems Theory}. Delft University Press. The Netherlands.

% \bibitem[Ogata(1995)]{ogata95}
% Ogata, Katsuhiko. 1995. \emph{Discrete-Time Control System Second Edition}. Prentice Hall International, USA.
% \end{thebibliography}
%-----------------------------------------------------------------
%-----------------------------------------------------------------

%-----------------------------------------------------------------
%Disini awal masukan untuk Lampiran (JIKA DIPERLUKAN)
%-----------------------------------------------------------------
\appendix
\addcontentsline{toc}{chapter}{LAMPIRAN}
\chapter{PROGRAM OPERASI PENJUMLAHAN MATRIKS}
\label{appx:addition}
% \section{Kode untuk dijalankan di GPU}\label{kode_potential_cpp}

\section{\emph{Module}}

\begin{lstlisting}
using CUDA
using DataFrames
using Statistics
\end{lstlisting}

\section{Untuk dijalankan di GPU}

\begin{lstlisting}
function operation_cuda(A, B)
    C = CUDA.zeros(Float16, size(A))
    CUDA.@sync C .= A .+ B
    return C
end

function simulate_gpu(A_normal, B_normal)
    # Convert your Float64 matrices to Float32 and then to CuArray{Float32}
    A = CUDA.convert(CuArray{Float16}, A_normal)
    B = CUDA.convert(CuArray{Float16}, B_normal)
    # Mengukur waktu eksekusi
    cuda_start = CUDA.@elapsed result_gpu = operation_cuda(A, B)
    return cuda_start
end
\end{lstlisting}

\section{Untuk dijalankan di CPU}

\begin{lstlisting}
function operation_cpu(A, B)
    C = zeros(Float16, size(A))
    C .= A .+ B
    return C
end

function simulate_cpu(A_normal, B_normal)
    normal_start = @elapsed result_cpu = operation_cpu(A_normal, B_normal)
    return normal_start
end
\end{lstlisting}

\section{Simulasi}

\begin{lstlisting}
function simulate(N::Int64, n_loop::Int64)
    println("Total size: $(N*N)")
    results_gpu = []
    results_cpu = []
    for i in 1:n_loop
        A_normal = rand(Float16, N, N)
        B_normal = rand(Float16, N, N)
        benchmark_gpu = simulate_gpu(A_normal, B_normal)
        benchmark_cpu = simulate_cpu(A_normal, B_normal)
        push!(results_gpu, benchmark_gpu)
        push!(results_cpu, benchmark_cpu)
    end
    println("Mean CPU: $(mean(results_cpu))")
    println("Mean GPU: $(mean(results_gpu))")
    return DataFrame(CPU=results_cpu, GPU=results_gpu)
end
\end{lstlisting}

\chapter{PROGRAM OPERASI PENGURANGAN MATRIKS}
\label{appx:substraction}

\section{\emph{Module}}

\begin{lstlisting}
using CUDA
using DataFrames
using Statistics
\end{lstlisting}

\section{Untuk dijalankan di GPU}

\begin{lstlisting}
function operation_cuda(A, B)
    C = CUDA.zeros(Float16, size(A))
    CUDA.@sync C .= A .- B
    return C
end

function simulate_gpu(A_normal, B_normal)
    # Convert your Float64 matrices to Float32 and then to CuArray{Float32}
    A = CUDA.convert(CuArray{Float16}, A_normal)
    B = CUDA.convert(CuArray{Float16}, B_normal)
    # Mengukur waktu eksekusi
    cuda_start = CUDA.@elapsed result_gpu = operation_cuda(A, B)
    return cuda_start
end
\end{lstlisting}

\section{Untuk dijalankan di CPU}

\begin{lstlisting}
function operation_cpu(A, B)
    C = zeros(Float16, size(A))
    C .= A .- B
    return C
end

function simulate_cpu(A_normal, B_normal)
    normal_start = @elapsed result_cpu = operation_cpu(A_normal, B_normal)
    return normal_start
end
\end{lstlisting}

\section{Simulasi}

\begin{lstlisting}
function simulate(N::Int64, n_loop::Int64)
    println("Total size: $(N*N)")
    results_gpu = []
    results_cpu = []
    for i in 1:n_loop
        A_normal = rand(Float16, N, N)
        B_normal = rand(Float16, N, N)
        benchmark_gpu = simulate_gpu(A_normal, B_normal)
        benchmark_cpu = simulate_cpu(A_normal, B_normal)
        push!(results_gpu, benchmark_gpu)
        push!(results_cpu, benchmark_cpu)
    end
    println("Mean CPU: $(mean(results_cpu))")
    println("Mean GPU: $(mean(results_gpu))")
    return DataFrame(CPU=results_cpu, GPU=results_gpu)
end
\end{lstlisting}

\chapter{PROGRAM OPERASI PERKALIAN SKALAR DENGAN MATRIKS}
\label{appx:multiplication_scalar_matrix}

\section{\emph{Module}}

\begin{lstlisting}
using CUDA
using DataFrames
using Statistics
\end{lstlisting}

\section{Untuk dijalankan di GPU}

\begin{lstlisting}
function operation_gpu(A, skalar)
    C = CUDA.zeros(Float32, size(A))
    CUDA.@sync C = A * skalar
    return C
end

function simulate_gpu(A_normal, skalar_normal)
    A = CUDA.convert(CuArray{Float32}, A_normal)
    cuda_start = CUDA.@elapsed result_gpu = operation_gpu(A, skalar_normal)
    return cuda_start
end
\end{lstlisting}

\section{Untuk dijalankan di CPU}

\begin{lstlisting}
function operation_cpu(A, skalar)
    C = zeros(Float32, size(A))
    C = A * skalar
    return C
end

function simulate_cpu(A_normal, skalar)
    normal_start = @elapsed result_cpu = operation_cpu(A_normal, skalar)
    return normal_start
end
\end{lstlisting}

\section{Simulasi}

\begin{lstlisting}
function simulate(N::Int64, n_loop::Int64)
    println("Total size: $(N*N)")
    results_gpu = []
    results_cpu = []
    for i in 1:n_loop
        A_normal = rand(Float32, N, N)
        scalar_normal = rand(Float32)  # Generate a random scalar
        benchmark_gpu = simulate_gpu(A_normal, scalar_normal)
        benchmark_cpu = simulate_cpu(A_normal, scalar_normal)
        push!(results_gpu, benchmark_gpu)
        push!(results_cpu, benchmark_cpu)
    end
    println("Mean CPU: $(mean(results_cpu))")
    println("Mean GPU: $(mean(results_gpu))")
    return DataFrame(CPU=results_cpu, GPU=results_gpu)
end
\end{lstlisting}

\chapter{PROGRAM OPERASI PERKALIAN ANTAR MATRIKS}
\label{appx:multiplication_matrix_matrix}

\section{\emph{Module}}

\begin{lstlisting}
using CUDA
using DataFrames
using Statistics
\end{lstlisting}

\section{Untuk dijalankan di GPU}

\begin{lstlisting}
function operation_gpu(A, B)
    C = CUDA.zeros(Float32, size(A))
    CUDA.@sync C = A * B
    return C
end

function simulate_gpu(A_normal, B_normal)
    A = CUDA.convert(CuArray{Float32}, A_normal)
    B = CUDA.convert(CuArray{Float32}, B_normal)
    cuda_start = CUDA.@elapsed result_gpu = operation_gpu(A, B)
    return cuda_start
end
\end{lstlisting}

\section{Untuk dijalankan di CPU}

\begin{lstlisting}
function operation_cpu(A, B)
    C = zeros(Float32, size(A))
    C = A * B
    return C
end

function simulate_cpu(A_normal, B_normal)
    normal_start = @elapsed result_cpu = operation_cpu(A_normal, B_normal)
    return normal_start
end
\end{lstlisting}

\section{Simulasi}

\begin{lstlisting}
function simulate(N::Int64, n_loop::Int64)
    println("Total size: $(N*N)")
    results_gpu = []
    results_cpu = []
    for i in 1:n_loop
        A_normal = rand(Float32, N, N)
        B_normal = rand(Float32, N, N)
        benchmark_gpu = simulate_gpu(A_normal, B_normal)
        benchmark_cpu = simulate_cpu(A_normal, B_normal)
        push!(results_gpu, benchmark_gpu)
        push!(results_cpu, benchmark_cpu)
    end
    println("Mean CPU: $(mean(results_cpu))")
    println("Mean GPU: $(mean(results_gpu))")
    return DataFrame(CPU=results_cpu, GPU=results_gpu)
end
\end{lstlisting}


\chapter{PROGRAM OPERASI INVERSE MATRIKS}
\label{appx:inverse}

\section{\emph{Module}}

\begin{lstlisting}
using CUDA
using DataFrames
using Statistics
\end{lstlisting}

\section{Untuk dijalankan di GPU}

\begin{lstlisting}
function operation_matrices_cuda(gpu_matrix)
    inverted_gpu_matrix = inv(gpu_matrix)  
    return inverted_gpu_matrix  
end

function simulate_gpu(A_normal)
    A = CUDA.cu(A_normal)  
    result_gpu, cuda_start = CUDA.@timed operation_matrices_cuda(A)
    return cuda_start 
end
\end{lstlisting}

\section{Untuk dijalankan di CPU}

\begin{lstlisting}
function operation_matrices_normal(A)
    C = zeros(Float32, size(A))
    C = inv(A)
    return C
end

function simulate_cpu(A_normal)
    normal_start = @elapsed result_cpu = operation_matrices_normal(A_normal)
    return normal_start
end
\end{lstlisting}

\section{Simulasi}

\begin{lstlisting}
function simulate(N::Int64, n_loop::Int64)
    results_gpu = []
    results_cpu = []
    for i in 1:n_loop
        A_normal = rand(Float32, N, N)
        benchmark_gpu = simulate_gpu(A_normal)
        benchmark_cpu = simulate_cpu(A_normal)
        push!(results_gpu, benchmark_gpu)
        push!(results_cpu, benchmark_cpu)
    end
    println("Mean CPU: $(mean(results_cpu))")
    println("Mean GPU: $(mean(results_gpu))")
    return DataFrame(CPU=results_cpu, GPU=results_gpu)
end
\end{lstlisting}

\chapter{PROGRAM PENCARIAN NILAI EIGEN}
\label{appx:eigenvalue}

\section{\emph{Module}}

\begin{lstlisting}
using LinearAlgebra
using SparseArrays
using CUDA
using Statistics
using DataFrames
\end{lstlisting}

\section{Fungsi pembuat matriks persamaan schrodinger melalui pendekatan beda-hingga}

\begin{lstlisting}
function build_matrix(n::Int)
    dx = 2.0f0 / n  # Menggunakan literal Float32
    gamma = 50.0f0  # Menggunakan literal Float32
    b = (gamma * dx) ^ 2
    y = LinRange(-2.0f0, 2.0f0, n)  # Memastikan y adalah Float32

    main_diag = fill(2.0f0, n)  # Isi dengan Float32
    off_diag = fill(-1.0f0, n-1)  # Isi dengan Float32
    a_sparse = spdiagm(0 => main_diag, 1 => off_diag, -1 => off_diag)

    # Penyesuaian elemen diagonal berdasarkan y
    for j in 1:n
        if y[j] <= -1.0 || y[j] >= 1.0
            a_sparse[j, j] += b
        end
    end

    return Matrix{Float32}(a_sparse)  # Konversi eksplisit matriks ke Float32
end
\end{lstlisting}

\section{Untuk dijalankan di GPU}

\begin{lstlisting}
function simulate_gpu(A)
    gpu_start = CUDA.@elapsed begin
        eigenvalues, eigenvectors = CUDA.@sync eigen(A)
        idx = sortperm(eigenvalues)
        eigenvalues = eigenvalues[idx]
        eigenvectors = eigenvectors[:, idx]
    end
    return gpu_start
end
\end{lstlisting}

\section{Untuk dijalankan di CPU}

\begin{lstlisting}
function simulate_cpu(A)
    cpu_start = @elapsed begin
        eigenvalues, eigenvectors = eigen(A)
        idx = sortperm(eigenvalues)
        eigenvalues = eigenvalues[idx]
        eigenvectors = eigenvectors[:, idx]
    end

    return cpu_start
end
\end{lstlisting}

\section{Simulasi}

\begin{lstlisting}
function simulate(n::Int, n_loop::Int)
    println("Matrix size: $n x $n")
    results_gpu = Float32[]
    results_cpu = Float32[]

    A_cpu = build_matrix(n)
    A_gpu = CuArray(A_cpu)

    for _ in 1:n_loop
        benchmark_gpu = simulate_gpu(A_gpu)
        benchmark_cpu = simulate_cpu(A_cpu)
        push!(results_gpu, benchmark_gpu)
        push!(results_cpu, benchmark_cpu)
    end

    println("Mean CPU: $(mean(results_cpu))")
    println("Mean GPU: $(mean(results_gpu))")
    return DataFrame(CPU=results_cpu, GPU=results_gpu)
end
\end{lstlisting}

\chapter{PROGRAM VALIDASI NILAI EIGEN}
\label{appx:eigenvalue_validation}

\section{\emph{Module}}

\begin{lstlisting}
using LinearAlgebra
using SparseArrays
using Plots
using CUDA
using Statistics
using Plots
using DataFrames
\end{lstlisting}

\section{Konstanta dan Fungsi Pembantu}

\subsection{Konstanta}

\begin{lstlisting}
hbar = 1.0545718e-34  # Konstanta Planck tereduksi (Joule sekon)
m = 9.10938356e-31    # Massa elektron (kg)
L = 2.0               # Panjang sumur potensial dalam meter
\end{lstlisting}

\subsection{Fungsi Pembantu}

\begin{lstlisting}
# Untuk mendapatkan nilai analitik nya
function analytics_eigenvalues(n::Int)
    return (n^2 * pi^2 * hbar^2) / (2 * m * L^2)
end

# Untuk mengonversi dari nilai eigen ke bentuk energi
function convert_eigenvalue_to_energy(n::Int64, eigenvalue::Float32)
    return (hbar^2 * eigenvalue) / (2 * m * ((L/n)^2))
end
\end{lstlisting}

\section{Fungsi pembuat matriks persamaan schrodinger melalui pendekatan beda-hingga}

\begin{lstlisting}
function build_matrix(n::Int)
    dx = 2.0f0 / n  # Menggunakan literal Float32
    y = LinRange(-2.0f0, 2.0f0, n)  # Memastikan y adalah Float32

    main_diag = fill(2.0f0, n)  # Isi dengan Float32
    off_diag = fill(-1.0f0, n-1)  # Isi dengan Float32
    a_sparse = spdiagm(0 => main_diag, 1 => off_diag, -1 => off_diag)

    return Matrix{Float32}(a_sparse)  # Konversi eksplisit matriks ke Float32
end
\end{lstlisting}

\section{Untuk dijalankan di GPU}

\begin{lstlisting}
function solve_gpu(A)
    eigenvalues, eigenvectors = CUDA.@sync eigen(A)
    idx = sortperm(eigenvalues)
    eigenvalues = eigenvalues[idx]
    eigenvectors = eigenvectors[:, idx]
    return Array(eigenvalues), Array(eigenvectors)
end
\end{lstlisting}

\section{Untuk dijalankan di CPU}

\begin{lstlisting}
function solve_cpu(A)
    eigenvalues, eigenvectors = eigen(A)
    idx = sortperm(eigenvalues)
    eigenvalues = eigenvalues[idx]
    eigenvectors = eigenvectors[:, idx]
    return eigenvalues, eigenvectors
end
\end{lstlisting}

\section{Simulasi}

\begin{lstlisting}
function simulate(n::Int)
    A_cpu = build_matrix(n)
    A_gpu = CuArray(A_cpu)
    eigenvalues_gpu, eigenvectors_gpu = solve_gpu(A_gpu)
    eigenvalues_cpu, eigenvectors_cpu = solve_gpu(A_gpu)

    return DataFrame(
        CPU=convert_eigenvalue_to_energy.(n, eigenvalues_cpu[1:10]), 
        GPU=convert_eigenvalue_to_energy.(n, eigenvalues_gpu[1:10]), 
        analitik=eigenvalues_analytics[1:10], 
        perbedaan=((abs.(convert_eigenvalue_to_energy.(n, eigenvalues_cpu[1:10]) .- eigenvalues_analytics[1:10])) ./ eigenvalues_analytics[1:10]) .* 100
        )
end
\end{lstlisting}

\end{document}
