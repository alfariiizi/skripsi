\(\pagebreak\)

\hypertarget{pendahuluan}{%
	\section{Pendahuluan}\label{pendahuluan}}

\hypertarget{latar-belakang-dan-rumusan-masalah}{%
	\subsection{Latar Belakang dan Rumusan
		Masalah}\label{latar-belakang-dan-rumusan-masalah}}

Di era saat ini, hampir semua hal berupa data dapat dibuat maupun
dimanipulasi dengan menggunakan komputer. Hal hal seperti menulis paper,
menganalisis suatu data, dan bermain game, dapat dilakukan secara
langsung pada komputer. Kemampuannya tersebut berasal dari komponen
\emph{Central Processing Unit} (CPU) yang mana dapat memproses berbagai
macam tugas yang diinginkan oleh pengguna. CPU menjadi inti dari
kecepatan proses penanganan tugas yang diberikan. Meskipun sebenarnya
ada peluang terjadi \emph{bottleneck} pada komponen lain, namun secara
umum, CPU yang berkecepatan cepat akan dapat menangani banyak tugas
dalam waktu yang singkat. Kecepatan CPU itu sendiri didasari oleh
kecepatan komputasi pada \emph{core} didalam CPU yang cukup cepat.
Sehingga, karena proses penanganan komputasi yang cukup baik inilah yang
membuat Komputer juga digunakan sebagai alat untuk komputasi, tidak
hanya komputasi secara umum, tapi juga komputasi dalam bentuk metode
komputasi.

Pada penggunaan metode komputasi, kerap kali muncul
\emph{computational cost}, yakni jika ingin hasil dari komputasi
merupakan hasil yang mempunyai akurasi tinggi, maka secara umum,
diperlukan perhitungan yang banyak dalam proses pencapaian akurasi
tinggi tersebut. Hal ini tentunya dengan asumsi bahwa metode yang
digunakan telah teruji efektifitas nya, sehingga perhitungan yang banyak
pun bukan merupakan perhitungan yang sia sia. Hal ini tentunya secara
umum sejalan dengan faktor koreksi yang diberikan, yakni, semakin banyak
faktor koreksi yang diberikan, maka akan semakin tinggi kemungkinan
hasil tersebut akan mempunyai hasil yang akurat. Metode seperti
Runge-Kutta orde 4 akan dianggap mempunyai akurasi yang lebih tinggi
daripada Metode Runge-Kutta orde 2 ataupun Metode Euler pada proses
diferensiasi.

\emph{Computational cost} inilah yang menghambat akurasi data jika
data yang digunakan sangatlah banyak. Misalnya saja, jika diinginkan
iterasi sebanyak 1 juta iterasi, maka agar proses kalkulasi tidak
memakan waktu sangat lama, digunakanlah metode diferensiasi yang lebih
sederhana seperti Metode Euler ataupun Metode Runge-Kutta orde 2.

Hambatan dari penggunaan metode komputasi akibat \emph{computational
	cost} ini tidak serta merta dikarenakan oleh penggunaan metode komputasi
yang digunakan, tapi juga penggunaan hardware yang digunakan. Secara
umum, proses komputasi dilakukan pada komputer dengan menggunakan
bantuan dari CPU. Akan tetapi, CPU pada dasarnya didesain untuk
mengerjakan tugas yang kompleks, bukan untuk mengerjakan komputasi yang
berat. Akibatnya, sering terjadi keterlambatan dalam perhitungan
komputasi yang berat, dan penggunaan metode sering menjadi alasan dari
keterlambatan ini.

Jika melihat pada industri game, banyak sekali game saat ini yang
mempunyai grafis yang sangatlah bagus. Padahal, dibelakang proses
rendering game tersebut, terdapat berbagai macam kalkulasi yang
diperlukan, seperti kalulasi refleksi cahaya, kalkulasi model dalam
sudut pandang tertentu, dsb. Proses kalkulasi tersebut tentunya tidak
menggunakan CPU untuk mengerjakannya, melainkan menggunakan
\emph{Graphics Processing Unit} (GPU). GPU digunakan dalam proses
rendering game dikarenakan GPU mempunyai banyak sekali \emph{core}
ketimbang \emph{core} yang dimiliki oleh CPU. Meskipun begitu,
masing-masing \emph{core} pada GPU berkinerja cukup lambat, sehingga
lebih cocok untuk digunakan dalam proses yang membutuhkan \emph{1 tugas
	yang berat} daripada \emph{berbagai macam tugas yang ringan}.

\hypertarget{tujuan}{%
	\subsection{Tujuan}\label{tujuan}}

Tujuan dari penelitian ini adalah agar menunjukkan bahwa penggunaan
jenis metode yang digunakan bukanlah suatu hal yang perlu diperdebatkan
agar \emph{computational cost} dapat dikurangi. Melainkan penggunaan
hardware lah yang menjadi tolak ukur kecepatan komputasi suatu metode.
Hardware yang dimaksud adalah hardware yang umum dijumpai pada pasar
komputer.

Karena \emph{Graphics Processing Unit} (GPU) adalah salah satu
hardware yang juga sering dijumpai pada komputer, maka digunakanlah
Vulkan API agar GPU bisa digunakan untuk proses komputasi, dengan tujuan
untuk membuktikan bahwa penggunaan metode Runge-Kutta orde 4 dan
Runge-Kutta orde 2 akan mempunyai kecepatan yang hampir sama pada
iterasi sebanyak 1 juta dan dengan langkah per iterasi sebesar
\(h = 0.0000001\).

Karena terkadang terdapat penurunan kinerja dari GPU akibat daya
baterai yang rendah, maka proses komputasi metode Runge-Kutta orde 4 dan
orde 2 tersebut dilakukan dengan daya baterai penuh + dilakukan
\emph{charging}, daya baterai berkisar 50\% hingga 60\% tanpa dilakukan
charging, dan daya baterai berkisar 20\% hingga 25\% tanpa dilakukan
charging.

\hypertarget{manfaat}{%
	\subsection{Manfaat}\label{manfaat}}

Dari penelitian ini, diharapkan \emph{barrier} mengenai
\emph{computational cost} dalam menggunakan metode komputasi tertentu
(pada mesin komputer biasa) dapat dikurangi.

\(\pagebreak\)

\hypertarget{tinjauan-pustaka}{%
	\section{Tinjauan Pustaka}\label{tinjauan-pustaka}}

\hypertarget{metode-runge-kutta}{%
	\subsection{Metode Runge-Kutta}\label{metode-runge-kutta}}

Dengan menggunakan notasi bahwa

\[
	\begin{aligned}
		y'     & = f(y) = f_y           \\
		y''    & = f'(y) y' = f'_y f    \\
		y(x_n) & = y_n                  \\
		f_n    & = f(y_n) = f(x_n, y_n)
	\end{aligned}
\]

Suryaningrat, dkk. {[}-@suryaningratOrderRungeKuttaExtended2020{]}
menyatakan bahwa metode runge-kutta orde 3, orde 4, dan orde 5, yang
telah diformulasikan oleh Geoken dan Johnson
{[}-@goekenFifthorderRungeKuttaHigher1999{]}, secara umum dapat
dinyatakan sebagai berikut:

\[
	\tag{2.1}
	y_{n+1} = y_n + h\sum_{i=1}^s b_i k_i
\]

dimana

\[
	\begin{aligned}
		k_i              & = f\left( y_n + c_i h, y_n + h \sum_{j}^{i-1} a_{ij}k_i \right ) \\
		c_i              & = \sum_{j=1}^N a_{ij}                                            \\
		\sum_{i=1}^N b_i & = 1
	\end{aligned}
\]

dimana koefisien \(a_{ij}\) mempunyai batasan iterasi
\(1 \leqslant j < i \leqslant s\); dan koefisien \(b_i\) dan \(c_i\)
mempunyai batasan iterasi untuk \(i = 1, 2, 3, \dotsc, s\).

\hypertarget{metode-runge-kutta-orde-2-rk-2}{%
	\subsubsection{Metode Runge-Kutta Orde 2 (RK
		2)}\label{metode-runge-kutta-orde-2-rk-2}}

Untuk kasus khusus, yakni pada Runge-Kutta orde 2, atau disebut juga
dengan Goeken-Johnson orde 3 (GJ3)
{[}@goekenFifthorderRungeKuttaHigher1999{]} dapat diformulasikan sebagai
berikut:

\[
	\tag{2.2}
	y_{n+1} = y_n + h \frac{1}{4} (k_1 + 3k_2)
\]

dimana nilai \(k_1\) dan \(k_2\) adalah

\[
	\begin{aligned}
		k_1 & = f(y_n)                                                     \\
		k_2 & = f \left (y_n + \frac{2}{3}k_1 + \frac{2}{9}h^2 y''\right )
	\end{aligned}
\]

\hypertarget{metode-runge-kutta-orde-4-rk-4}{%
	\subsubsection{Metode Runge-Kutta Orde 4 (RK
		4)}\label{metode-runge-kutta-orde-4-rk-4}}

Untuk kasus Runge-Kutta orde 4 ini dilakukan pendekatan oleh Goeken
dan Johnson {[}-@goekenFifthorderRungeKuttaHigher1999{]} yang
menghasilkan Persamaan (2.3), yang mana Persamaan (2.3) ini juga disebut
dengan Goeken-Johnson orde-5 (GJ 5)

\[
	\tag{2.3}
	y_{n+1} = y_n + h \left ( \frac{5}{48} k_1 + \frac{27}{56} k_2 + \frac{125}{336}k_3 + \frac{1}{24}k_4 \right )
\]

dimana nilai \(k_1\), \(k_2\), \(k_3\), dan \(k_4\) nya adalah

\[
	\begin{aligned}
		k_1 & = f(y_n)                                                                                                \\
		k_2 & = f \left ( y_n + \frac{1}{3}k_1 + \frac{1}{18}h f_y k_1 \right )                                       \\
		k_3 & = f \left ( y_n - \frac{152}{125}k_1 + \frac{252}{125}k_2 - \frac{44}{125}h f_y k_1 \right )            \\
		k_4 & = f \left ( y_n + \frac{19}{2}k_1 - \frac{72}{7}k_2 + \frac{25}{14}k_3 + \frac{5}{2} h f_y k_1 \right )
	\end{aligned}
\]

\hypertarget{vulkan-api}{%
	\subsection{Vulkan API}\label{vulkan-api}}

Application Programming Interface (API) merupakan suatu standarisasi
yang dibuat agar kode yang dituliskan dapat secara umum dipahami oleh
pengembang lain dan juga dapat mempermudah pengembang dalam membuat
suatu aplikasi. Dengan adanya API, pengembang tidak perlu tahu bagaimana
API tersebut diimplementasikan, yang perlu diketahui adalah bagaimana
cara menggunakan API tersebut {[}@ibmWhatApplicationProgramming2021{]}.
Misalnya saja, pada Windows terdapat \emph{Win32 API} yang mana
ditujukan agar pengembang yang ingin membuat aplikasi di Windows dengan
Bahasa C/C++ dapat secara langsung membuat aplikasi di Windows tanpa
diperlukannya pengetahuan mengenai kode-kode penyusun Windows itu
sendiri. API yang dapat mengontrol GPU disebut dengan \emph{Graphics
API} {[}@armGraphicsAPIs{]}. Beberapa contoh Graphics API adalah OpenGL,
Vulkan, Directx 3D, dll.

Vulkan merupakan graphics Application Programming Interface (API)
dari hasil kerjasama AMD dan Khronos Group. Vulkan pertama kali
diperkenalkan secara publik pada penghelatan \emph{Game Developers
	Conference} (GDC) pada tahun 2015 {[}@khronosMoreVulkanSPIRV2015{]}.
Vulkan didesain agar \emph{low overhead}, yakni pengembang mampu untuk
mengontrol berbagai macam hal mengenai GPU, seperti performa dan
efisiensi pada GPU {[}@amdVulkan{]}. Disisi lain, \emph{low overhead}
merupakan suatu kelebihan, disisi lain \emph{low overhead} menuntut
eksiplisitas dalam menulis kode API nya, sehingga pengembang dituntut
untuk mampu memahami kompleksitas arsitektur API ini.

Vulkan tersusun dari beberapa prosesor dan \emph{fixed-function} yang
berguna untuk mempercepat proses rendering maupun komputasi pada suatu
device. Prosesor pada device/GPU yang terdapat vulkan nya, pada umumnya,
mempunyai jumlah prosesor atau core yang cukuplah banyak. Hal ini lah
yang membuat Vulkan menyelesaikan suatu tugas dengan menggunakan
komputasi parallel {[}@sellersVulkanProgrammingGuide2017{]}.

\hypertarget{general-purpose-gpu-gpgpu}{%
	\subsection{General Purpose GPU
		(GPGPU)}\label{general-purpose-gpu-gpgpu}}

Fungsi dari \emph{Graphics Processing Unit} (GPU) pada awalnya hanya
sebatas untuk meningkatkan bagian-bagian tertentu pada \emph{graphics
	pipeline}\footnote{Proses - proses tertentu untuk merender suatu scene
	atau model.} pada saat merender scene atau model suatu game. Hal ini
dikarenakan pada era awal GPU, tidak terlalu banyak model maupun scene
yang membutuhkan kinerja tinggi dari GPU, sehingga pada umumnya yang
menggunakan GPU hanya digunakan untuk bermain game maupun untuk
pengembangan game. Seiring berjalannya waktu, muncul bidang bidang yang
membutuhkan perhitungan dan komputasi yang cukup tinggi, seperti pada
bidang Film, yakni muncul Computer-Generated Imagery (CGI) yang mana
diawali dengan munculnya film Toy Story; kemudian pada bidang \emph{Machine
	Learning} juga muncul berbagai macam pemodelan yang membutuhkan
perhitungan yang berat, dsb. Bidang - bidang baru yang muncul tersebut
tidak semuanya membutuhkan rendering scene maupun model, namun bisa
hanya sebatas kalkulasi berat. Penggunaan GPU untuk memproses
\emph{non-graphical entities}\footnote{Seperti scene maupun model.},
dikenal dengan \emph{General Purpose GPU} (GPGPU)
{[}@ghorpadeGPGPUProcessingCUDA2012{]}.

Pada dasarnya, tidak semua \emph{Application Programming Interface}
(API) yang mampu untuk melakukan GPGPU. Hal ini dikarenakan penggunaan
GPU untuk GPGPU ini muncul hanya beberapa tahun kebelakang, sehingga
untuk API yang dibuat dari awal tahun 2000an atau akhir abad 20an.
Vulkan dipublikasikan pada tahun 2015, sehingga tentunya Vulkan dapat
digunakan untuk GPGPU.

\(\pagebreak\)

\hypertarget{dasar-teori}{%
	\section{Dasar Teori}\label{dasar-teori}}

Keterbatasan core pada \emph{Central Processing Unit} (CPU), dapat
membuat pemilihan metode suatu penelitian, khusus nya metode
diferensiasi, menjadi terbatas seiring peningkatan iterasi yang
digunakan. Untuk mengatasi hal itu, digunakanlah \emph{Graphics
	Processing Unit} (GPU) dengan menggunakan \emph{Application Programming
	Interface} (API) yang mendukung penggunaan GPGPU agar proses iterasi
yang tinggi pada metode komputasi yang lebih kompleks, seperti metode
diferensiasi Runge-Kutta orde-2 dan Runge-Kutta orde 4, dapat dilakukan
dengan cepat dan akurat.

Untuk mempermudah penggunaan Vulkan pada GPGPU, terdapat Module Vuh
{[}@savenkoVuhVulkanbasedGPGPU2021{]}, yang mana dapat digunakan untuk
mengimplementasikan suatu model agar dapat dikalkulasi oleh GPU.

\(\pagebreak\)

\hypertarget{metode-penelitian}{%
	\section{Metode Penelitian}\label{metode-penelitian}}

\hypertarget{waktu-dan-tempat}{%
	\subsection{Waktu dan Tempat}\label{waktu-dan-tempat}}

Penilitian ini bisa dilakukan di berbagai tempat, dengan waktu yang
sudah dijelaskan seperti yang tertuliskan pada Tabel 5.1.

\hypertarget{alat-dan-bahan}{%
	\subsection{Alat dan Bahan}\label{alat-dan-bahan}}

Berikut merupakan alat dan bahan yang akan digunakan pada penelitian
ini:

\begin{enumerate}
	\def\labelenumi{\arabic{enumi}.}
	\tightlist
	\item
	      Sebuah Laptop, dengan spesifikasi:

	      \begin{itemize}
		      \tightlist
		      \item
		            Operating System (OS): \textbf{Manjaro} Linux
		      \item
		            RAM: \textbf{8 gb}
		      \item
		            Processor: \textbf{Intel i7 gen-10th}
		      \item
		            GPU: \textbf{Nvidia GeForce MX250}
	      \end{itemize}
	\item
	      \emph{Text editor}: \textbf{VS Code}
	\item
	      \emph{Build system} dan \emph{generator}: \textbf{CMake}
	\item
	      Module \textbf{Vulkan-tools} dan Module \textbf{Vuh}.
\end{enumerate}

\hypertarget{prosedur-kerja}{%
	\subsection{Prosedur Kerja}\label{prosedur-kerja}}

Berikut merupakan prosedur kerja yang dilakukan pada penelitian ini:

\begin{enumerate}
	\def\labelenumi{\arabic{enumi}.}
	\tightlist
	\item
	      Dilakukan penulisan kode Vulkan API berdasarkan module Vuh.
	\item
	      Memastikan bahwa baterai device dalam kondisi penuh dan sedang
	      dilakukan charging.
	\item
	      Iterasi sebanyak 1 juta dengan \(h = 0.0000001\) dijalankan dengan
	      menggunakan metode Runge-Kutta orde 2 dan orde 4 pada kode yang sudah
	      ditulis.
	\item
	      Mencatat hasil dari Runge-Kutta orde 2 dan orde 4, serta mencatat juga
	      waktu yang diperlukan untuk melakukan proses komputasi tersebut.
	\item
	      Mengulangi langkah 3 dan 4 dengan kondisi daya baterai 50\%-60\% tanpa
	      dilakukan charging, dan kondisi daya baterai 20\%-25\% yang juga tanpa
	      dilakukan charging.
\end{enumerate}

Jika muncul banyak masalah ketika melakukan komputasi dengan
menggunakan module Vuh, maka langkah pertama pada prosedur kerja diatas
diganti menjadi penulisan secara langsung kode Vulkan API tanpa
penggunaan module Vuh.

\(\pagebreak\)

\hypertarget{jadwal-penelitian}{%
	\section{Jadwal Penelitian}\label{jadwal-penelitian}}

Penelitian ini dilakukan pada Tahun 2022 dengan jadwal seperti yang
tertera pada Tabel 5.1.

Table 5.1: Rencana Jadwal Penelitian

\begin{longtable}[]{@{}
	>{\centering\arraybackslash}p{(\columnwidth - 2\tabcolsep) * \real{0.1984}}
	>{\raggedright\arraybackslash}p{(\columnwidth - 2\tabcolsep) * \real{0.8016}}@{}}
	\toprule
	\begin{minipage}[b]{\linewidth}\centering
		Tanggal
	\end{minipage} & \begin{minipage}[b]{\linewidth}\raggedright
		                 Kegiatan yang Dilakukan
	                 \end{minipage}                            \\
	\midrule
	\endhead
	1 Januari - 7 Januari                     & Menuliskan kode vulkan dengan menggunakan
	Vuh.                                                                                   \\
	10 Januari - 12 Januari                   & Mencoba untuk melakukan komputasi dengan
	kondisi daya baterai penuh                                                             \\
	13 Januari - 20 Januari                   & Dilakukan penulisan kode Vulkan API secara
	langsung tanpa menggunakan module Vuh (jika diperlukan).                               \\
	22 Januari - 24 Januari                   & Mencoba untuk melakukan komputasi dengan
	kondisi daya baterai 50\%-60\%                                                         \\
	26 Januari - 28 Januari                   & Mencoba untuk melakukan komputasi dengan
	kondisi daya baterai 20\%-25\%                                                         \\
	\bottomrule
\end{longtable}

\(\pagebreak\)

\hypertarget{daftar-pustaka}{%
	\section*{Daftar Pustaka}\label{daftar-pustaka}}
\addcontentsline{toc}{section}{Daftar Pustaka}
